\mysection{Concluding Remarks}
\label{section:conclusions}

The increasing ubiquity and capability of smart devices has driven society to
create restricted spaces. We presented a mechanism for hosts to remotely
inspect and control devices within restricted spaces. Our mechanism achieves
this goal via a narrow interface that permits remote memory operations. Our
prototype implements the mechanism for TrustZone-enabled devices.

While technically feasible, our approach must overcome certain hurdles before
it can be practically adopted. The foremost among these is end-user willingness
to subject their devices to regulation in restricted spaces. It is possible
that many users would simply choose not to use their devices within the
restricted space rather than give the host control over their devices (assuming
they do not resort to using the devices covertly). However, given our
increasing reliance on smart devices and the ways in which they are becoming
integral parts of our daily lives, it is unclear going forward whether such
simple ``opt-out'' solutions would even be a possibility. For example,
opting-out would not be a practical solution for smart devices integrated with
health monitoring and assistive functionality. Even if end-users were willing
to subject their devices to regulation, it is unclear to what extent they will
trust the host's control over their device. Solutions such as our vetting
service could ameliorate these concerns. We plan to explore these
user-interaction and user-perception issues in future work.


% We see this paper as making two main conceptual advances. The first is the
% notion of restricted spaces. As argued in the paper, the increasing ubiquity
% and capability of smart devices has driven society to create such restricted
% spaces. To promote the regulated use of smart devices in such spaces, we need
% systematic ways to enforce host-defined policies on guest devices.
% 
% The second conceptual contribution of this paper is the design of a mechanism
% for hosts to remotely inspect and control guest devices.  We showed that this
% can be achieved via a narrow interface that permits two simple operations,
% remote memory reads and writes. These operations provide an effective way for
% hosts to analyze and configure guest devices in accordance with their
% policies.  We also showed that with hardware support from the ARM TrustZone,
% we can bootstrap the security of remote memory operations.
%
% Again, we feel that this is largely a matter of user-perception. For example,
% users already place implicit trust in app writers and market vendors when they
% download apps on their smart devices, even though it is well-known that apps
% routinely violate this implicit trust and compromise user privacy by sending
% sensitive user data to unauthorized third parties.  
%
% Thus, a practical deployment of our solution would also likely require
% supporting trust infrastructure, \eg~analysis tools that can vet the host's
% operations or present them to the end-user in a human-readable format for
% approval.
