\section{Related Work}
\label{section:related}

\myparagraph{TrustZone-based Frameworks}
%
Since the introduction of TrustZone, a number of projects have used it to build
novel security applications.  TrustDump~\cite{trustdump:esorics14} is a
TrustZone-based mechanism to reliably acquire memory pages from the normal
world of a device.  While similar in spirit to remote reads, TrustDump's focus
is to be an alternative to virtualized memory introspection solutions for
malware detection. Unlike our work, TrustDump is not concerned with restricted
spaces, authenticating the host, or remotely configuring guest devices.

Samsung Knox~\cite{knox:ccs14} and \textsc{Sprobes}~\cite{sprobes:most14}
leverage TrustZone to protect the normal world in real-time from kernel-level
rootkits. These projects harden the normal world kernel by making it perform a
world switch when it attempts to perform certain sensitive operations to kernel
data. A reference monitor in the secure world checks these operations, thereby
preventing rootkits. In our work, remote reads allow the host to detect
infected devices, but we do not attempt to provide real-time protection from
malware. As discussed in \sectref{section:threat}, our work can be used in
conjunction with Knox to improve the security of the normal world.

TrustZone has also been used to improve the security of user applications.
Microsoft's TLR~\cite{tlr:asplos14} and Nokia's ObC~\cite{obc:asiaccs09}
leverage TrustZone to provide a secure execution environment for user
applications, even in the presence of a compromised kernel. Other applications
include ensuring trustworthy sensor readings from
peripherals~\cite{tenor:mobisys12} and securing mobile payments~\cite{proxama}.

\myparagraph{Enterprise Security} With the growing emphasis on BYOD, a number
of projects have developed enterprise security solutions for smart devices.
Many of these projects have focused on enabling multiple persona
(\eg~\cite{asm:sec14,flaskdroid:sec13,cells:sosp11}) or enforcing mandatory
access control policies on smart devices
(\eg~\cite{deepdroid:ndss15,seandroid:ndss13,flaskdroid:sec13,asm:sec14}).
Prior work has also explored access control for devices based on the context of
their use
(\eg~\cite{crepe:isc10,conucon:securecomm10,saint:acsac09,Covington2002,Damiani2007,conxsense:asiaccs14}).

There are three differences between these projects and ours. First, they
typically exercise their policies at the level of user apps and not
peripherals.  Our work is synergistic with these efforts, complementing the
app-level control that they provide. Second, most of these techniques are
typically implemented by enhancing the runtime platform on the guest device. In
our work, the policy enforcement mechanism is agnostic to the guest platform
because it works at the level of memory reads and writes. The platform-specific
portions (\ie~analyzing memory images and designing the write operations) are
performed at the host.  This approach keeps the TCB that executes on the guest
minimal in functionality.  Third, a minor point of difference is that these
prior projects are not integrated with secure hardware, although it may be
possible to integrate them with TrustZone.

\myparagraph{App Security}
%
It is now well-known that many popular apps exfiltrate sensitive user data from
smart devices~\cite{taintdroid:osdi10}. Moreover, a significant fraction of
apps (on Android) are over-privileged~\cite{stowaway:ccs11,pscout:ccs12} and
end-users are poor at understanding the meaning of app
permissions~\cite{ep:ubicomp12,felt:soups12}.  Such apps can leverage the
increasing array of sensors on modern smart devices in novel and dangerous
ways~\cite{soundcomber:ndss11,placeraider:ndss13}. These threats will amplify
in the future as we see an increasing number of augmented reality apps that
continuously monitor sensor feeds and extract data from the device's
environment.

Some projects have attempted to rectify the situation by offering improved app
permission models~\cite{howtoask:hotsec12} or modifying the execution
environment on the device to return ``fake'' sensor data to
apps~\cite{mockdroid:hotmobile10}. However, such techniques are usually
ineffective when the device itself is compromised (\eg~via kernel rootkits), or
if the user unintentionally installs a malicious app. Researchers have also
investigated defenses tailored toward improving privacy in the presence of
augmented reality apps~\cite{ar:sec13,darkly:oak13}. Our work can complement
these efforts by giving hosts the ability to control peripherals below the app
layer. 

% \myparagraph{Hardening Smart Devices}
% 
% Finally, the research community has addressed techniques to harden the
% software stack of smart devices. Samsung Knox~\cite{knox:ccs14}, as
% previously discussed, provides the ability to detect certain classes of
% kernel-level rootkits in real time.  MOCFI~\cite{mocfi:ndss12} enhances the
% mobile operating system by enforcing control-flow integrity properties,
% thereby mitigating the effect of attacks such as buffer overflow-based
% exploits.  Airbag~\cite{airbag:ndss14} employs lightweight virtualization to
% isolate user apps and prevent them from infecting the device's firmware or
% leaking sensitive information. At the app level,
% RetroSkeleton~\cite{retro:mobisys13} rewrites Android apps to improve their
% security on commodity devices. These techniques can help improve the
% resilience of smart devices to attack. Our work allows hosts to remotely
% analyze smart devices via remote memory operations and verify that they are
% free of malware infection.


% \myparagraph{Defending from Malware}
% \myparagraph{Context Awareness}



% * Clear difference from SAMSUNG KNOX.
% * List the things that we don't want to do (because SAMSUNG KNOX is our competition)
%     - Intercept critical operations.
%     - They cannot detect memory overflow errors.
%     - Ours is asynchronous, theirs is synchronous.
%     - No writes/uninstalling devices.
% * KNOX is not enforcing any enterprise security policy. Ours is enforcing specific enterprise policies.
% * KNOX checks the integrity of certain kernel memory pages.
%
% Prior work has developed numerous examples of security primitives that use
% remote memory operations. Examples include memory forensics~\cite{}, kernel
% malware detection~\cite{}, and operating system repair~\cite{}.  
% However, the bulk of these techniques have usually been developed for
% server-class systems or personal computers with larger form factors, such as
% desktops, that typically use the x86 architecture and where methods to isolate
% the target from the monitor, \ie~virtualization or co-processors, are readily
% available.
%
