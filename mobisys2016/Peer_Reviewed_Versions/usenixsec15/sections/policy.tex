\section{Device Analysis and Control}
\label{section:policy}

The ability to remotely inspect and configure guest devices gives hosts the
power to enforce a wide array of policies. In this section, we present a number
of applications of such remote device analysis and control.

\subsection{Kernel Analysis}
\label{section:policy:analysis}

The host can perform a variety of analyses on the guest device by remotely
reading its memory contents.

\myparagraph{Kernel-malware Detection} The host can analyze memory pages to
identify malicious software in the end-user's work environment. Such
\textit{kernel-level rootkits} allow attackers to retain long-term control over
infected devices, and are a realistic threat for mobile
devices~\cite{hotmobile10,mobisys11}. It is well-accepted by the security
community that rootkits cannot reliably be detected by user-space anti-malware
tools on the infected device because these tools rely on the services provided
by the operating system, which is itself infected. Rootkits can reliably be
detected only using external methods, which inspect the infected kernel without
relying on its services.

There are a number of ways to detect kernel-level malware by analyzing
raw-memory pages. In the past, simple kernel-level rootkits operated by
maliciously modifying code pages of the kernel. The host can easily detect such
rootkits by checking that all executable code pages of the kernel belong to a
whitelist of known-good pages.  Modern rootkits work by injecting malicious
code into the kernel, and modifying key kernel data structures to force their
execution~\cite{sbcfi:ccs07}. For example, a rootkit could inject a device
driver, and modify function pointers within the kernel to invoke functions from
this driver. More generally, a rootkit can modify an arbitrary kernel data
structure to implement its malicious functionality. Such data structure
modifications can be used to hide malicious user processes by modifying process
lists, disable firewalls by modifying kernel hooks, and weaken cryptographic
functions by modifying entropy pools~\cite{shadows:oakland07}.

Prior work~\cite{sbcfi:ccs07,gib:tdsc11,kop:ccs09,kop:sec12,strider:tr14}
has developed a generic approach to detect malicious data modifications by
analyzing raw memory pages. The main idea is for the host to externally
reconstruct the kernel's data structures, and use this view to reason about the
integrity of kernel data. We assume that the host has access to the type
declarations of the data structures used by the kernel being analyzed, \eg~the
sizes, layouts, and fields of every data structure.  The host starts the
reconstruction process from some well-known entrypoints into the system's
memory, \eg~the addresses of the entities in \texttt{System.map}, and
recursively traversing kernel data structures. When the traversal process
encounters a pointer, it must fetch the data member referenced by the pointer.
The host therefore iteratively requests virtual pages from the guest device
based upon the needs of this recursive traversal. Having reconstructed data
structures, the host can then determine whether they have been maliciously
modified. For example, it could check that function pointers in the kernel
point to pre-determined addresses in the kernel's code
space~\cite{sbcfi:ccs07}. Similarly, the host can check that the kernel's data
structures satisfy invariants that must typically hold in an uncompromised
kernel~\cite{gib:tdsc11}.

\myparagraph{Reverse-engineering Configurations} The host can use the memory
pages obtained from the guest to identify the version of the kernel, and to
determine if security patches have been applied. The host can also use it to
check whether the configuration of the kernel disallows certain well-known
attack surfaces (\eg~that access to \texttt{/dev/kmem} and dynamic module
loading are disabled).  The recursive traversal method described above can also
be used to reconstruct kernel configurations.  For example, it can be used to
identify addresses at which the functions of a peripheral's driver are loaded,
where they are hooked into the kernel, and the addresses that store
memory-mapped peripheral configurations. The host uses this information to
create remote write requests that can uninstall peripherals from the device. 

\myparagraph{Check Running Processes and Apps} Once the host has ensured that
the kernel is not infected by rootkits, it can use the clean kernel to
bootstrap security at the user level. For example, it can require the
end-user's work environment to execute an anti-malware app that relies on the
clean kernel to detect sources of infection at user-space, \eg~infected apps
that reside on the file system. The host can ensure that this anti-malware app
is executing on the guest device during check-in by traversing the kernel's
list of running processes. At check-out, it can ensure that the same app is
still executing by comparing its process identifier to the value obtained
at check-in.


\subsection{Control over Peripherals}
\label{section:policy:control}

The host can remotely configure peripherals on the guest device by writing to
suitable memory locations. We consider various peripherals and describe
scenarios where control over them would be useful.
 
\myparagraph{Camera} The use of the camera is perhaps the most obvious way for
guests to violate privacy and confidentiality
(\eg~\cite{placeraider:ndss13,placeavoider:ndss14}). In the federal and
enterprise settings, employees may photograph or videotape sensitive documents
and meetings, while in social settings, the camera can be used to record
conversations. In such settings, the camera can be disabled at check-in.

\myparagraph{Microphone} Much like the camera, the microphone can also be used
to violate privacy and confidentiality. However, disabling the microphone on
devices such as smartphones may not be acceptable to guests because it also
prevents them from having phone conversations. It may be possible for the host
to configure the microphone's driver so that the microphone is activated only
when a call is placed, and is de-activated otherwise. However, this facility
can potentially be misused by malicious guests to record meetings by simply
placing a call during the meeting. Therefore, it may be desirable to simply
disable the microphone except when the guest places an outgoing call to an
emergency number, and require guests to check-out if they wish to place other
phone calls.

\myparagraph{Networking} Enterprises (and federal institutions) today often
disallow employees and visitors from connecting their personal devices to the
corporate network. This is typically to avoid exfiltrating information outside
the enterprise. With our approach, enterprises can allow employees to use the
network after inspecting their devices to ensure that they are free from
malicious software. To prevent exfiltration, the enterprise can disable the use
of 3G/4G, and restrict the device to only WiFi. Because the enterprise controls
the WiFi network, it can therefore regulate what data is accessible to the
device and the external hosts to which the device can connect. In an
examination setting, for example, the proctor/university can similarly restrict
networking interfaces when students check-in their devices.

\myparagraph{Detachable Storage} USB dongles and flash drives are extensively
used to copy files across devices. However, they have also served as the
vehicle for malware infections (\eg~the Conficker worm~\cite{conficker:tifs12})
and can be used by malicious guests to exfiltrate sensitive data from the host.
Many enterprises only have informal guidelines discouraging their employees
from using dongles and flash drives to copy files. With our approach, the host
can simply disable the drivers for USB and flash drives at check-in, thereby
preventing the use of detachable storage media within the restricted space.

\myparagraph{Bluetooth} Smart glasses and smart watches rely on Bluetooth to
pair with a more powerful hub, such as a smart phone, via which they connect to
the external world. While it is generally possible to control their connection
to the outside world by constraining the networking interfaces on the hub
device, in some scenarios it may be necessary to prevent the device from
pairing with the hub. For example, a student wearing a prescription smart glass
may pair the device with his smart phone and use it to access class notes
stored locally on the phone. In such cases, disabling Bluetooth prevents this
channel, but still allows the student to use the glass for vision correction.

% \myparagraph{Fine-grained Control over Apps} While our approach gives hosts
% the power to control peripherals on the guest device, it cannot offer hosts
% fine-grained control over apps within the restricted space. Control over apps
% may be desirable in certain settings. For example, an enterprise may wish to
% constrain the set of apps visible on employees' devices when they are at
% work.  Our approach works by directly modifying device memory, and therefore
% operates at a lower level than is necessary to exercise control over apps. To
% enforce such control, it is generally necessary to work higher in the
% software stack.  For example, the Android Security Modules (ASM) framework
% enforces app-level controls by leveraging modifications to the Android
% runtime~\cite{asm:sec14}.  Our approach can be used in tandem with ASM to
% provide a layered approach to policy enforcement. Hosts can leverage remote
% memory operations to ensure that the guest device executes the ASM framework,
% and then rely on the framework to enforce app-level controls.

