\mysection{Concluding Remarks}
\label{section:conclusions}

% To date, restricted space hosts have largely employed \adhoc\ methods to ensure
% that guest devices are policy-compliant. Going forward, it is not clear that
% \adhoc\ methods alone will suffice. From the guest's perspective, simple
% solutions such as opting not to use devices within restricted spaces are also
% not an option, given our increasing reliance on smart devices. 

This paper develops mechanisms that allow hosts to analyze and regulate ARM
TrustZone-based guest devices using remote memory operations.  These mechanisms
can be implemented with only a small amount of trusted code running on guest
devices. The use of the TrustZone allows our approach to provide strong
guarantees of guest policy-compliance to hosts. Our vetting service allows
guests to identify conflicts between their privacy goals and the hosts' usage
policies. 

While this paper demonstrates technical feasibility of our approach, questions
about its adoptability in real-world settings remain to be answered. For
example, we can imagine our solution to be readily applicable in settings such
as federal or corporate offices and examination halls, where restricted spaces are
clearly demarcated and the expectations on guest device usage are clearly outlined.
Will it be equally palatable in less stringent settings, such as social
gatherings or restaurants? A meaningful answer to this question will require a
study of issues such as user-perception and willingness to allow their devices
to be remotely analyzed and controlled by hosts. We hope to pursue these
questions in follow-on research.


% We see this paper as making two main conceptual advances. The first is the
% notion of restricted spaces. As argued in the paper, the increasing ubiquity
% and capability of smart devices has driven society to create such restricted
% spaces. To promote the regulated use of smart devices in such spaces, we need
% systematic ways to enforce host-defined policies on guest devices.
% 
% The second conceptual contribution of this paper is the design of a mechanism
% for hosts to remotely inspect and control guest devices.  We showed that this
% can be achieved via a narrow interface that permits two simple operations,
% remote memory reads and writes. These operations provide an effective way for
% hosts to analyze and configure guest devices in accordance with their
% policies.  We also showed that with hardware support from the ARM TrustZone,
% we can bootstrap the security of remote memory operations.
%
% Again, we feel that this is largely a matter of user-perception. For example,
% users already place implicit trust in app writers and market vendors when they
% download apps on their smart devices, even though it is well-known that apps
% routinely violate this implicit trust and compromise user privacy by sending
% sensitive user data to unauthorized third parties.  
%
% Thus, a practical deployment of our solution would also likely require
% supporting trust infrastructure, \eg~analysis tools that can vet the host's
% operations or present them to the end-user in a human-readable format for
% approval.
