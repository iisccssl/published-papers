\mysection{Guest Privacy and Security}
\label{section:vetting}

We built a vetting service trusted by guests to determine the safety of a
host's request. We built it as a cloud-based server, to which the guest device
forwards the host's memory updates together with a copy of its normal world
memory image (via the UI app). We assume that the device and the vetting
service have authenticated each other as in \figref{figure:authentication} or
use SSL/TLS to obtain a communication channel with end-to-end confidentiality
and integrity guarantees. It may also be possible to implement vetting within
the secure world itself. However, we chose not to do so to avoid bloating the
secure world.

The vetting server checks the host's requests against its safety policies and
returns a \textsc{Safe} or \textsc{Unsafe} response to the device. The response
is bound with a random nonce and an HMAC to the original request in the
standard way to prevent replay attacks. The secure world performs the
operations only if the response is \textsc{Safe}. Guests can configure the
vetting server with domain-specific policies to determine safety.  Our
prototype vetting service, which we built as a plugin to the Hex-Rays IDA
toolkit~\cite{ida-pro}, analyzes memory images and checks for the following
safety policies. Although simple and based on conservative whitelisting, in our
experiments, the policies could prove safety without raising false positives.

\myparagraph{Read-safety} For each request to read from address
\textit{vaddr}$_i$, we return \textsc{Safe} only if \textit{vaddr}$_i$ falls in
a pre-determined range of virtual addresses. In our prototype, acceptable
address ranges only include pages that contain kernel code and kernel data
structures.  The vetting server returns \textsc{Unsafe} if the read request
attempts to fetch any addresses from kernel buffers that store user app data,
or virtual address ranges that lie in app user-space memory.

\myparagraph{Write-safety} Our prototype currently only allows write requests
to \textsc{null}ify peripheral interfaces or install dummy drivers that disable
peripherals. We use the following safety policy for dummy drivers.  For each
function $f$ implemented in the dummy driver, consider its counterpart $f_{\it
orig}$ from the original driver, which the vetting service obtains from the
device's memory image. We return \textsc{Safe} only if the function $f$ is
identical to $f_{\it orig}$, or $f$'s body consists of a single return
statement that returns a \textit{valid} error code (\eg~\textsc{-enomem}). We
define an error code as being valid for $f$ if and only if the same error code
is returned along at least one path in $f_{\it orig}$. The intuition behind
this safety check is that $f$ does not modify the memory state of the device or
introduce new and possibly buggy code, but returns an error code that is
acceptable to the kernel and client user apps.  For more complex dummy drivers
that introduce new code, the vetting service could employ a traditional malware
detector or more complex program analyses to scan this code for safety.
