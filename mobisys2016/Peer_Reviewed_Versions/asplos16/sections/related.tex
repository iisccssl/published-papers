\mysection{Related Work and Design Alternatives}
\label{section:related}

\myparagraph{TrustZone Support}
%
A number of projects have used TrustZone to build novel security applications.
TrustDump~\cite{trustdump:esorics14} is a TrustZone-based mechanism to reliably
acquire memory pages from the normal world of a device (Android
LiME~\cite{lime} and similar tools~\cite{dmd,ddms,recoverymode} do so too, but
without the security offered by TrustZone).  While similar in spirit to remote
reads, TrustDump's focus is to be an alternative to virtualized memory
introspection solutions for malware detection. Unlike our work, TrustDump is
not concerned with restricted spaces, authenticating the host, or remotely
configuring guest devices.

Samsung Knox~\cite{knox:ccs14} and \textsc{Sprobes}~\cite{sprobes:most14}
leverage TrustZone to protect the normal world in real-time from kernel-level
rootkits. These projects harden the normal world kernel by making it perform a
world switch when it attempts to perform certain sensitive operations to kernel
data. A reference monitor in the secure world checks these operations, thereby
preventing rootkits. In our work, remote reads allow the host to detect
infected devices, but we do not attempt to provide real-time protection from
malware. Our work can also leverage Knox to enhance the security of the normal
world (\sectref{section:threat}).

TrustZone has also been used to improve the security of user applications.
Microsoft's TLR~\cite{tlr:asplos14} and Nokia's ObC~\cite{obc:asiaccs09} use
TrustZone to provide a secure execution environment for user apps, even in the
presence of a compromised kernel. Other applications include ensuring
trustworthy sensor readings from peripherals~\cite{tenor:mobisys12} and
securing mobile payments~\cite{proxama}.

\myparagraph{Enterprise Security} With the growing ``bring your own device''
(BYOD) trend, a number of projects have developed enterprise security solutions
that enable multiple persona
(\eg~\cite{asm:sec14,flaskdroid:sec13,cells:sosp11}) or enforce mandatory
access control policies on smart devices
(\eg~\cite{deepdroid:ndss15,seandroid:ndss13,flaskdroid:sec13,asm:sec14}).
Prior work has also explored context-based access control and techniques for
restricted space objects to push usage policies onto guest devices
(\eg~\cite{saint:acsac09,Covington2002,conxsense:asiaccs14,worlddriven:ccs14,blindspot:2009,markit:upside14}).

These projects tend to use one of two techniques. One is to require guest
devices to run a software stack enhanced with a policy enforcement mechanism.
For instance, ASM~\cite{asm:sec14} introduces a set of security hooks in
Android, which consult a security policy (installed as an app) that can be used
to create multiple persona on a device. Each persona is customized with a view
of apps and peripherals that it can use. Another approach is to require
virtualized guest devices
\cite{cells:sosp11,cox:hotmobile07,vmwareverizon,kvmarm:asplos14}. In this
approach, a trusted hypervisor on the guest device enforces isolation between
VMs implementing different persona.

The main benefit of these techniques over our work is the greater app-level
control that they provide. For example, they can be used to selectively block
sensitive audio and blur faces by directly applying policies to the
corresponding user apps~\cite{worlddriven:ccs14,ar:sec13}. These techniques
are able to do so because they have a level of semantic visibility into
app-level behavior that is difficult to achieve at the level of raw memory
operations.

On the other hand, our approach enjoys two main benefits over prior work.
First, our approach simplifies the design of the trusted policy-enforcing code
that runs on guest devices to a TCB of just a few thousand lines of code. In
contrast, security-enhanced OSes and virtualized solutions required hundreds of
thousands of lines of trusted policy-enforcement code to execute on guest
devices.  Prior research has investigated ways to reduce the TCB, \eg~by
creating small hypervisors~\cite{nova:eurosys09}. However, the extent to which
such work on small hypervisors applies to smart devices is unclear, given that
any such hypervisor must support a variety of different virtualization modes,
guest quirks, and hardware features on a diverse set of personal devices.

The second benefit of our approach is that it provides security guarantees that
are rooted in trusted hardware. Prior projects have generally trusted guest
devices to correctly implement the host's policies. This trust can easily be
violated by a guest running a maliciously-modified OS or hypervisor.  It is
also not possible for a host to obtain guarantees that the policy was enforced
for the duration of the guest's stay in the restricted space. We leverage the
TrustZone to offer such guarantees using verification tokens and REM-suspend.

\myparagraph{Other Hardware Interfaces}
%
Hardware interfaces for remote memory operations were originally investigated
for the server world to perform remote DMA as a means to bypass the performance
overheads of the TCP/IP stack~\cite{mellanox,infiniband}.  This work has since
been repurposed to perform kernel malware detection~\cite{copilot:sec04} and
remote repair~\cite{backdoor:icac04}. These systems use a PCI-based
co-processor on guests via which the host can remotely transfer and modify
memory pages on the guest.

On personal devices, the closest equivalent to such a hardware interface is the
IEEE 1394 (Firewire), which is available on some laptops. However, it is not
currently available on smaller form-factor devices.  Another possibility is to
use the JTAG interface~\cite{jtag}, which allows read/write access to memory
and CPU registers via a few dedicated pins on the chipset.  However, the JTAG
is primarily used for debugging and is not easily accessible on consumer
devices.  One drawback of repurposing these hardware interfaces is that they
cannot authenticate the credentials of the host that initiates the memory
operation. Moreover, to use these hardware interfaces on guest devices, the
host needs physical access to plug into them.  Thus, these interfaces are best
used when the guest can physically authenticate the host and trust it to be
benign.


% \myparagraph{App Security}
% 
% It is now well-known that many popular apps exfiltrate sensitive user data
% from smart devices~\cite{taintdroid:osdi10}. Moreover, a significant fraction
% of apps (on Android) are over-privileged~\cite{stowaway:ccs11,pscout:ccs12}
% and end-users are poor at understanding the meaning of app
% permissions~\cite{ep:ubicomp12,felt:soups12}.  Such apps can leverage the
% increasing array of sensors on modern smart devices in novel and dangerous
% ways~\cite{soundcomber:ndss11,placeraider:ndss13}. These threats will amplify
% in the future as we see an increasing number of augmented reality apps that
% continuously monitor sensor feeds and extract data from the device's
% environment.

% Some projects have attempted to rectify the situation by offering improved
% app permission models~\cite{howtoask:hotsec12} or modifying the execution
% environment on the device to return ``fake'' sensor data to
% apps~\cite{mockdroid:hotmobile10}. However, such techniques are usually
% ineffective when the device itself is compromised (\eg~via kernel rootkits),
% or if the user unintentionally installs a malicious app.  Our work can
% complement these efforts by allowing hosts to control peripherals below the
% app layer. 

% \myparagraph{Hardening Smart Devices}
% 
% Finally, the research community has addressed techniques to harden the
% software stack of smart devices. Samsung Knox~\cite{knox:ccs14}, as
% previously discussed, provides the ability to detect certain classes of
% kernel-level rootkits in real time.  MOCFI~\cite{mocfi:ndss12} enhances the
% mobile OS by enforcing control-flow integrity properties,
% thereby mitigating the effect of attacks such as buffer overflow-based
% exploits.  Airbag~\cite{airbag:ndss14} employs lightweight virtualization to
% isolate user apps and prevent them from infecting the device's firmware or
% leaking sensitive information. At the app level,
% RetroSkeleton~\cite{retro:mobisys13} rewrites Android apps to improve their
% security on commodity devices. These techniques can help improve the
% resilience of smart devices to attack. Our work allows hosts to remotely
% analyze smart devices via remote memory operations and verify that they are
% free of malware infection.


% \myparagraph{Defending from Malware}
% \myparagraph{Context Awareness}



% * Clear difference from SAMSUNG KNOX.
% * List the things that we don't want to do (because SAMSUNG KNOX is our competition)
%     - Intercept critical operations.
%     - They cannot detect memory overflow errors.
%     - Ours is asynchronous, theirs is synchronous.
%     - No writes/uninstalling devices.
% * KNOX is not enforcing any enterprise security policy. Ours is enforcing specific enterprise policies.
% * KNOX checks the integrity of certain kernel memory pages.
%
% Prior work has developed numerous examples of security primitives that use
% remote memory operations. Examples include memory forensics~\cite{}, kernel
% malware detection~\cite{}, and OS repair~\cite{}.  
% However, the bulk of these techniques have usually been developed for
% server-class systems or personal computers with larger form factors, such as
% desktops, that typically use the x86 architecture and where methods to isolate
% the target from the monitor, \ie~virtualization or co-processors, are readily
% available.
%
