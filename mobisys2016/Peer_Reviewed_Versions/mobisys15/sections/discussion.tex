\section{Discussion}
\label{section:discussion}

% In this section, we discuss alternative ways to implement the
% framework, especially on guest devices that may not support the ARM TrustZone.
% We also consider other applications of our policy enforcement framework. 

% \subsection{Design Alternatives} 

\label{section:discussion:alternatives}
%
The design of our enforcement mechanism relies on the ARM TrustZone to isolate
the TCB and to provide a root of trust on the guest device. While devices
featuring the ARM TrustZone hardware are becoming increasingly available
(millions of deployed devices, according to the Samsung Knox
team~\cite{knox:ccs14}), we consider alternatives that can be used by
hosts to regulate guest devices.

\myparagraph{Virtualization} 
%
Historically, on x86 server-class machines, virtualization is by far the most
popular technique to perform introspection~\cite{chennoble:hotos01,vmi:ndss03}.
There have been impressive recent efforts to build virtualization
infrastructure for smart
phones~\cite{cells:sosp11,cox:hotmobile07,vmwareverizon} and other devices
based on the ARM architecture~\cite{kvmarm:asplos14}.

It is possible to regulate the use of peripherals on virtualized smart devices
using virtual machines (VMs) to implement various persona. For example, for an
employee to use his personal device at work, he may be required to download a
work VM from enterprise. The virtual peripherals in this work VM can be
pre-configured to reflect the enterprise policy. Thus, if the enterprise
disallows the use of the camera, the work VM is simply configured to not export
a virtual camera peripheral to apps executing within the VM. Virtualized
platforms also support remote memory operations---the hypervisor simply
reads/modifies the virtual memory pages belonging to a VM to perform the
operations. Thus, the enterprise can even dynamically reconfigure its work VM
via such operations.

However, in virtualized solutions, the TCB typically includes the hypervisor.
Although prior work has developed small hypervisors containing only a few
thousand lines of code (\eg~Nova~\cite{nova:eurosys09}), modern production
hypervisors are must support different virtualization modes, guest quirks, and
hardware features. This makes them large and complex, and has the unfortunate
consequence of bloating the TCB. Moreover, without a hardware root of trust, it
is hard to establish whether the guest has maliciously bypassed the host's
enforcement. For example, the employee in the above example could suspend the
work VM, and switch to a different VM that allows the use of the camera. The
hypervisor must include additional functionality to prevent the employee from
suspending the work VM, which further increases the size of the TCB.

TCB size-concerns aside, we also feel that virtualization may not be applicable
to large majority of personal computing devices. Many classes of smart devices
such as smart glasses and watches are personal, single-user devices, and
specialized in their function. Judging by their typical use-cases,
virtualization may be an overkill for most of these devices except those at the
highest end of the form factor (\eg~enterprise-class smart phones and tablets).
This reduces the motivation to invest the manpower needed to virtualize them.

\myparagraph{Alternative Network Interfaces}
%
As an alternative to virtualization, it may be possible to use hardware
interfaces to perform remote memory operations on a guest system. Such
interfaces were investigated for the server world by the high-performance
computing community to perform remote DMA as a means to bypass the performance
overheads of the TCP/IP stack~\cite{mellanox,infiniband}. This work has since
been repurposed to perform kernel malware detection~\cite{copilot:sec04} and
remote repair~\cite{backdoor:icac04}. These systems use a PCI-based
co-processor on the guest system via which the host can remotely transfer and
modify memory pages on the guest.

On personal devices, the closest equivalent to such a hardware interface is the
IEEE 1394, popularly called the Firewire. The host can leverage the Firewire
interface on a guest device to directly read and modify its memory pages.
However, a large majority of small form-factor mobile devices available today
are not equipped with the Firewire interface. For instance, to our knowledge,
except for some laptop computers, most other mobile devices do not have a
Firewire port. Another possibility is to use the JTAG interface, defined in the
IEEE 1149.1 standard by the Joint Test Action Group. Via a small number of
dedicated pins on the chip, this interface allows testing functions such as
reading and writing to the memory and registers of the chip. However, the JTAG
is primarily used as a debugging aid.

The main drawback of these alternative network interfaces typically cannot
authenticate the credentials of the host that initiates the memory operation.
This shortcoming can be exploited by a malicious host to compromise the
security and privacy of the guest. Moreover, to configure guest devices via
these network interfaces, the host must physically plug into the network
interfaces of these devices. Thus, these interfaces are best used when the
guest can physically authenticate the host and trust it to be benign.

\myparagraph{OS-based Mechanisms}
%
Operating systems can be modified to dump the contents of their memory. For
example, on Android, the LiME framework~\cite{lime} and similar
tools~\cite{dmd,ddms,recoverymode} can be used to acquire memory from a running
system. However, without trusted hardware, the host has no way to check the
integrity of these memory dumps.

Finally, it is possible to regulate the use of guest devices by requiring the
user's work environment to run a security-enhanced operating system. The
Android Security Modules (ASM) framework~\cite{asm:sec14}, for instance,
introduces a set of security hooks in Android.  These hooks consult a security
policy, installed as a user-space app, to enforce fine-grained control over the
device. As an example, policies in the the ASM system can be used to create
persona that can determine the set of apps, files, and peripherals that are
visible to the end-user. This approach works at a higher level of abstraction
than remote memory operations, and therefore has the benefit of offering
greater visibility into application-level context and finer-grained control
over apps.

However, as with virtualization, the approach presents a large TCB because the
security-enhanced operating system must be trusted. Moreover, the security
mechanism itself is closely integrated the operating system. In contrast, in
approaches that use remote memory operations, the mechanism is agnostic to
the operating system executing on the guest device. The burden of analyzing
memory and determining which locations to modify is left to the host, thereby
allowing a minimal, platform-agnostic TCB to execute on the guest.


% \subsection{Other Applications}
% \label{section:discussion:otherapps}

% \todo{VG}{To be done}
% Other applications of mechanism/policy besides the restricted space model.
% For example, personalized malware detectors using USB. Talk about the "host"
% in this case being controlled by the guest itself, and ways to implement the
% host, e.g., USB, cloud, etc.

