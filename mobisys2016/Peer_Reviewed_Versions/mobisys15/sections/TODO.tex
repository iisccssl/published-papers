\section{TODO List}

In this section, I am listing the TODOs as discussed in our telecon on
September 12th. 

The tasks marked under {\bf TU} and {\bf RU} are the immediate tasks for our
respective teams, and I foresee that we should complete these in a couple of
weeks at the most. The tasks marked under {\bf Both} are things that I added on
my own, since I feel that these are the next things to discuss and implement
for an end-to-end system. I also feel we {\bf Both} need to discuss additional
tasks to strengthen the paper.

\begin{mylist}

\item{\bf TU:} Detecting kernel malware using remote memory reads.
%
\begin{itemize}
%
\item Fetch memory pages and demonstrate and end-to-end case study that shows
the power of remote memory fetches to detect kernel malware.
\begin{itemize}
    \item Develop kernel malware: 3 days
    \item Apply memory forensic: 4 days
\end{itemize}
%
\item Also develop the components of the cross-view detection approach needed
to detect towelroot. This would use a kernel module together with a user-space
malware detector to detect malicious disk or process-level activity, for
instance.
\begin{itemize}
    \item Develop SW part: 3 days
    \item Develop kernel module: 5 days
    \item Develop user module: 4 days
\end{itemize}
%
\end{itemize}

\item{\bf RU:} Uninstalling peripherals using remote memory writes.
%
\begin{itemize}
%
\item Show an end-to-end case study (on the evaluation board) of uninstalling a
peripheral such as the camera or the microphone using remote memory writes.
\begin{itemize}
 \item Uninstalling the camera: - 9/23
\end{itemize}

%
\item Develop the protocol/technique to ensure that this uninstalling operation
cannot be undone by a simple reboot of the phone. Or if it is undone, it should
be detectable to the enterprise.
\begin{itemize}
 \item Develop the enterprise part: 9/24 - 09/29
 \item Develop the SW part: 09/30 - 10/03
\end{itemize}

%
\end{itemize}

\item{\bf Both:} End-to-end system integration.
%
\begin{itemize}
%
\item \textit{Handshake and Bootstrap.} Build the components in the SW and NW
to engage in the ``handshake'' that will perform mutual authentication of the
device and the enterprise, and bootstrap the SSL connection with the
enterprise. This is the first step before any remote memory operations are
performed. The SW must be involved in any authentication of the device (since
the entire BYOD authentication begins with a button press that puts the SW in
control), but the SSL connection is established from the NW. Protocol must
prevent man-in-the-middle attacks (even when the NW acts as the ``man in the
middle'', since the NW could be malicious).
%
\item \textit{Enterprise back-end.} Build the back-end (e.g., on the cloud) to
do security policy checking on the memory pages that we read from the device.
We can do as complex checking as we want, and many prior papers have done this.
Perhaps our prototype can keep the policies simple (e.g., just checking hashes
for code pages, and some simple invariants for data pages).
%
\item \textit{Handling important corner cases.} Remote memory reads and writes
to the NW must account for the fact that NW may have changed during the remote
memory read operations. This \textit{will} happen since the NW is involved in
the transfer of the signed pages. Thus, pages transmitted by the NW may not
match the signatures created for them by the SW.  The enterprise may contact
the NW to send the pages that have changed. Figure out and implement the
end-to-end protocol for the enterprise and NW in this setting. We assume that
if the NW does not cooperate (e.g., because it is malicious), the enterprise
will not allow the device to connect. Thus, it is in the best interest of the
NW to cooperate. Thus, for example, the NW could give ``hints'' to the SW and
the enterprise that certain pages are likely to change during transmission, and
that it is okay for the enterprise to contact the device to obtain fresh copies
of these pages (or ignore them if it does not matter for the security policy
being enforced). Figure out the mechanisms to handle this corner case.
%
\end{itemize}

\item{\bf Both:} Think of additional ideas to strengthen paper. What else can
we do (e.g., additional case studies) to add meat to the paper and support our
case? I briefly thought about repair of a kernel infected by malware, but Liviu
has done this in the past and it is extremely tricky to do correctly, so I
dropped it from this list.

\end{mylist}
