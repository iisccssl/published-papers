Smart personal devices have become ubiquitous, and we increasingly rely on
their use in our daily lives. Simultaneously, we as a society are also
beginning to place restrictions on how these devices can be used in various
environments.  Examples of such \textit{restricted spaces} abound in today's
society.  For example, enterprises and federal offices that store sensitive
information place restrictions on the use of cameras and microphones in smart
devices. In the classroom, students are often disallowed from using the aid of
smart devices during examinations. And in social settings, people are often
unwilling to have their conversations recorded by their friends wearing smart
glasses. To date, the enforcement of these restrictions has been \adhoc, and
not surprisingly, easily bypassable.

In this paper, we present a systematic approach for restricted space hosts to
remotely analyze and regulate guest device use in the restricted space. Our
approach cleanly separates policy from mechanism.  Policies on device use are
decided by the hosts that control the restricted space. These policies are
enforced by a trusted mechanism that executes on the smart guest device. We
present an implementation of our mechanism on smart devices equipped with the
ARM TrustZone architecture. We also showcase a number of useful policies that
show how device use can be regulated in a variety of restricted spaces.

% Submissions MUST be no more than fourteen (14) pages excluding references.
% Provide an abstract of fewer than 250 words.
