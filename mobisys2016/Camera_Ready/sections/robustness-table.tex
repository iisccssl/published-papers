\newcommand{\apperr}{\textsc{\small AppErrMsg}}
\newcommand{\lostconn}{\textsc{\small LostConn}}
\newcommand{\anderr}{\textsc{\small AndroidErrMsg}}
\newcommand{\emptyfile}{\textsc{\small EmptyFile}}
\newcommand{\blnkscrn}{\textsc{\small BlankScreen}}
\newcommand{\im}{\it\bfseries\small}

\begin{figure*}[htbp!]
%\small
\renewcommand{\arraystretch}{0.5}
\begin{center}
%
%\begin{tabular}{|c|p{0.16\textwidth}|p{0.16\textwidth}|p{0.16\textwidth}|p{0.16\textwidth}|p{0.16\textwidth}|}
\begin{tabular}{|c|c|c|c|c|c|}
\hline
{\bf USB} & {\im MobileWebCam} & {\im Camera ZOOM FX} & {\im Retrica} & {\im Candy Camera} & {\im HD Camera Ultra}\\ 
\hline
\textit{Passive}    & \apperr      & \apperr        & \anderr & \apperr      & \anderr\\
\textit{Active}     & \apperr      & \apperr        & \apperr & \apperr      & \apperr\\
\hline
\hline
{\bf Camera} & {\im Camera for Android} & {\im Camera MX} & {\im Camera ZOOM FX} & {\im HD Camera for Android} & {\im HD Camera Ultra} \\
\hline
\textit{Passive}       & \anderr            & \apperr   & \apperr        & \anderr               & \anderr\\
\textit{Active}        & \blnkscrn          & \apperr   & \anderr        & \blnkscrn             & \blnkscrn\\
\hline
\hline
{\bf WiFi} & {\im Spotify} & {\im Play Store} & {\im YouTube} & {\im Chrome Browser} & {\im Facebook}\\
\hline
\textit{Passive} & \lostconn & \lostconn & \lostconn & \lostconn & \lostconn\\
\textit{Active}  & \lostconn & \lostconn & \lostconn & \lostconn & \lostconn\\
\hline
\hline
{\bf 3G (Data)} & {\im Spotify} & {\im Play Store} & {\im YouTube} & {\im Chrome Browser} & {\im Facebook}\\
\hline
\textit{Passive} & \lostconn & \lostconn & \lostconn & \lostconn & \lostconn\\
\textit{Active}  & \lostconn & \lostconn & \lostconn & \lostconn & \lostconn\\
\hline
\hline
{\bf 3G (Voice)} & \multicolumn{5}{c|}{\im Default call application}\\
\hline
\textit{Passive} & \multicolumn{5}{c|}{\apperr: \small Unable to place a call}\\
\textit{Active}  & \multicolumn{5}{c|}{\apperr: \small Unable to place a call}\\
% {\bf 3G (Voice)} & \multicolumn{5}{p{0.8\textwidth}|}{\im Default call application}\\
% \hline
% \textit{Passive} & \multicolumn{5}{p{0.8\textwidth}|}{\apperr: Unable to place a call}\\
% \textit{Active}  & \multicolumn{5}{p{0.8\textwidth}|}{\apperr: Unable to place a call}\\
\hline
\hline
{\bf Microphone} & {\im Audio Recorder} & {\im Easy Voice Recorder} & {\im Smart Voice Recorder} & {\im Sound and Voice Recorder} & {\im Voice Recorder}\\
\hline
\textit{Passive}           & \apperr        & \apperr             & \apperr              & \apperr    & \apperr\\
\textit{Active}            & \emptyfile     & \emptyfile          & \emptyfile           & \emptyfile & \emptyfile\\
\hline
\end{tabular}
\end{center}
%
\begin{center}
\begin{tabular}{p{0.95\textwidth}}
{\footnotesize We use \textit{Passive} to  denote experiments in which the user app
was not running when the peripheral's driver was replaced with a dummy, and the
app was started after this replacement. We use \textit{Active} to denote
experiments in which the peripheral's driver was replaced with a dummy even as
the client app was executing. 
%
\circone~\apperr\ denotes the situation where the user app starts normally, but
an error message box is displayed within the app after it starts up; 
%
\circtwo~\blnkscrn\ denotes a situation where the user app displayed a blank
screen; 
%
\circthree~\lostconn\ denotes a situation where the user app loses network
connection; 
%
\circfour~\emptyfile\ denotes a situation where no error message is displayed,
but the sound file that is created is empty; 
%
\circfive~\anderr\ denotes the situation where the user app fails to start (in
the passive setting) or a running app crashes (in the active setting), and the
Android runtime system displays an error.}\\
% \hline
\end{tabular}
\end{center}
\indent\vspace{-0.5cm}
\mycaption{Results of robustness experiments for user apps.}
{\label{table:robustness}}
%
\end{figure*}

