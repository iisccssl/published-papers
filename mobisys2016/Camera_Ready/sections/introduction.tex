\mysection{Introduction}
\label{section:introduction}

% \addtext{Task~0}{Test}
% \deltext{Test}

Personal computing devices, such as phones, tablets, glasses, watches,
assistive health monitors and other embedded devices have become an integral
part of our daily lives. We carry these devices as we go, and expect them to
connect and work with the environments that we visit.

While the increasing capability of smart devices and universal connectivity are
generally desirable trends, there are also environments where these trends may
be misused. In enterprise settings and federal institutions, for instance,
malicious personal devices can be used to exfiltrate sensitive information to
the outside world. In examination settings, smart devices may be used to
infiltrate unauthorized information~\cite{url:examcheating}, surreptitiously
collude with peers~\cite{smartwatch:fc14} and cheat on the exam.  Even in less
stringent social settings, smart devices may be used to record pictures, videos
or conversations that could compromise privacy. We therefore need to regulate
the use of smart devices in such \textit{restricted spaces}.

Society currently relies on a number of \adhoc\ methods for policy enforcement
in restricted spaces. In the most stringent settings, such as in federal
institutions, employees may be required to place their personal devices in
Faraday cages and undergo physical checks before entering restricted spaces.
In corporate settings, employees often use separate devices for work and
personal computing needs. Personal devices are not permitted to connect to the
corporate network, and employees are implicitly, or by contract, forbidden from
storing corporate data on personal devices. In examination settings, proctors
ensure that students do not use unauthorized electronic equipment.  Other
examples in less formal settings include restaurants that prevent patrons from
wearing smart glasses~\cite{url:glassban}, or privacy-conscious individuals who
may request owners to refrain from using their devices.

We posit that such \adhoc\ methods alone will prove inadequate given our
increasing reliance on smart devices. For example, it is not possible to ask an
individual with prescription smart glasses (or any other assistive health
device) to refrain from using the device in the restricted space. The right
solution would be to allow the glass to be used as a vision corrector, but
regulate the use of its peripherals, such as the camera, microphone, or WiFi.
A general method to regulate the use of smart devices in restricted spaces
would benefit both the \textit{hosts} who own or control the restricted space
and \textit{guests} who use smart devices.  Hosts will have greater assurance
that smart devices used in their spaces conform to their usage policies. On the
other hand, guests can benefit from and be more open about their use of smart
devices in the host's restricted space.\footnote{We only consider overt use of
guest devices.  Covert use must still be addressed using other methods, such as
physical checks.}

Prior research projects
(\eg~\cite{asm:sec14,flaskdroid:sec13,conxsense:asiaccs14,worlddriven:ccs14,blindspot:2009,markit:upside14})
and enterprise mobile-device management (MDM) solutions to address this problem
(\eg\ Samsung Knox~\cite{knox:mdm}, Microsoft Intune~\cite{ms:intune} and
Blackberry EMM~\cite{blackberry:emm}) have typically assumed that guest devices
are benign. These solutions outfit the guest device with a security-enhanced
software stack that is designed to accept and enforce policies supplied by
restricted space hosts. A host must trust the software running on a guest
device to correctly enforce its policies, and generally has no means to obtain
guarantees that a guest device is policy-compliant.  Clearly, malicious guest
devices with a suitably-modified software stack can easily bypass policy
enforcement.

Our vision is to enable restricted space hosts to enforce usage policies on
guest devices with provable security guarantees. Simultaneously, we also wish
to reduce the amount of trusted policy-enforcement code (\ie~the size of the
security-enhanced software stack) that needs to execute on guest devices. To
that end, this paper offers a number of advances:

\begin{mylist}
%
\item \emphitem{Use of trusted hardware.} We leverage the ARM
TrustZone~\cite{armtz} on guest devices to offer provable security guarantees.
In particular, a guest device uses the ARM TrustZone to produce
\textit{verification tokens}, which are unforgeable cryptographic entities that
establish to a host that the guest is policy-compliant.  Malicious guest
devices, which may have violated the host's policies in the restricted space,
will not be able to provide such a proof, and can therefore be apprehended by
the host. Devices that use the ARM TrustZone are now commercially available and
widely deployed~\cite{knox:ccs14}, and our approach applies to these devices.
%
\item \emphitem{Remote memory operations.} We use host-initiated remote memory
operations as the core method to regulate guest devices. In this approach, a
host decides usage policies that govern how guest devices must be regulated
within the restricted space. For example, the host may require certain
peripherals on the guest device (\eg~camera, WiFi or 3G/4G) be disabled in the
restricted space. The host sends these policies to the guest device, where a
trusted policy-enforcement mechanism applies these policies by reading or
modifying device memory. 

The principal benefit of using remote memory operations as an API for policy
enforcement is that it considerably simplifies the design and implementation of
the policy-enforcement mechanism, while still offering hosts fine-grained
control over guest devices.  Remote memory operations also give hosts that use
our approach the unique ability to scan guest devices for kernel-level malware.
Combined with the ARM TrustZone, which helps bootstrap trust in the guest's
policy-enforcement code, our approach offers hosts an end-to-end assurance that
guest devices are policy-compliant.
%
\item \emphitem{Secure device checkpointing.} The downside to enforcing
policies by modifying device memory is that changes to the guest device are
ephemeral, and can be undone with a simple reboot of the guest device. We
therefore introduce \textit{REM-suspend}, a secure checkpointing scheme to
ensure that a guest device remains ``tethered'' to the host's policies even
across device reboots and other power-down events.
%
\item \emphitem{Vetting for guest device privacy and security.} The advances
above benefit hosts, but guests may be uncomfortable with the possibility of
hosts accessing and modifying raw memory on their devices. If access to raw
guest device memory is not mediated, malicious hosts may be able to use this
access to compromise the guest's privacy and security. For example, the host
can read sensitive and private app data from devices and install malicious
snooping software (\eg~keyloggers) on the guest device.  We therefore mediate
the host's access to the guest device by introducing a vetting service. The
vetting service is trusted and configurable by guests, and allows them to check
the safety of the host's memory operations before performing them on the
devices.  The vetting service ameliorates guests' privacy and security concerns
and restricts the extent to which hosts can control their devices.
%
\end{mylist}

We built and evaluated a prototype to show the benefits of our approach. We
show that a small policy-enforcing code base running on guest devices offers
hosts fine-grained policy-based control over the devices. We also show that a
vetting service with a few simple sanity checks allows guests to ensure the
safety of the host's remote memory operations.
