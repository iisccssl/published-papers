\mysection{Introduction}
\label{section:introduction}

% Problem definition.
%
\add{It is now common for Web applications (\textit{host}) to include untrusted,
third-party \js\ code (\textit{guest}) of arbitrary provenance in the form
of advertisements, libraries and widgets. Despite advances in language and
browser technology, \js\ still lacks mechanisms that enable Web application
developers to debug and understand the behavior of third-party guest
code. Using existing reflection techniques in \js, the host cannot attribute
changes in the \js\ heap and the DOM to specific guests. Further, fine-grained
context about a guest's interaction with the host's DOM and network is not
supported. For example, the host cannot inspect the behavior of guest code under
specific cookie values or decide whether to allow network requests by the
guests.}

\add{This paper proposes to enhance the \js\ language with builtin support for
introspection of third-party guest code. The main idea is to extend \js\
with a new \texttt{transaction} construct, within which hosts can speculatively
execute guest code containing arbitrary \js\ constructs. In addition to
enforcing security policies on guests, a \texttt{transaction} would allow hosts
to \textit{cleanly recover from policy-violating actions of guest code}. When
a host detects an offending guest, it simply chooses not to commit the
transaction corresponding to the guest. Such an approach neutralizes any data
and DOM modifications initiated earlier by the guest, without having to undo
them explicitly. The introspection mechanism (\texttt{transaction}) is built
within the \js\ language itself, thereby allowing guest code to contain
arbitrary \js\ constructs (unlike contemporary
techniques \cite{adsafe,fbjs,caja,wrappers:esorics09,language:csf09}).} 

\remove{The client-side of modern Web applications frequently contains \js\ code
in the form of advertisements, games, and libraries that may be developed by
untrusted third parties.  To include such third-party code (the \textit{guest}),
the author of such a Web application (the \textit{host}) must choose to
(a)~place the guest code in its own protection domain (\eg~an \code{iframe}) or
(b)~include the guest in the same protection domain as the host. The first
approach isolates the guest from the host, but has the downside of offering
limited means for guest/host interaction.  Any communication between the guest
and the host must happen via explicit \code{postMessage} commands between the
corresponding \code{iframes}.  The second approach eases guest/host interaction,
but also exposes the host to malicious guests. Hosts must therefore employ
protection mechanisms to confine the actions of guest code. This paper
addresses the problem of designing such protection mechanisms.

% Prior work to address the problem.
%
One way to protect against untrusted guests is to use \textit{\js\ reference
monitors}.  Reference monitors mediate the action of guest code as it executes
and ensure that any accesses by the guest to the host's data structures conform
to the host's security policies. Examples of reference monitors include
Caja~\cite{caja}, object capabilities~\cite{mmt10}, inlined reference
monitors~\cite{phung:asiaccs09,corescript:popl07,mets:hotos07}, object
wrappers~\cite{wrappers:esorics09}, and aspect-oriented policy
enforcement~\cite{conscript:oak10}. Some reference monitors also require the
guest code to be written in safe subsets of \js\ (\eg~Caja). The main purpose of
these reference monitors is to provide \textit{access control} to the host's
objects.

% What is our unique contribution to this space?
%
This paper proposes a novel approach to construct \js\ reference monitors using
speculative execution.  The main idea is to extend \js\ with a new
\textit{transaction} construct, within which hosts can speculatively execute
guest code that contains arbitrary \js\ constructs. In addition to enforcing
access control policies, reference monitors that use speculative execution
allow hosts to \textit{cleanly recover from the policy-violating actions of
guest code}. When a host detects an offending guest, it simply chooses not to
commit the transaction corresponding to the guest. Such an approach neutralizes
any data and DOM modifications initiated earlier by the guest, without having
to undo them explicitly.}

% Why is this contribution important? (i.e., motivating example).
%
\remove{To motivate the need for recovery,} Let us consider an example of a
Web-based word processor that hosts a third-party widget to display
advertisements (see \figref{figure:motivate}). During an editing session, this
widget scans the document for specific keywords and displays advertisements
relevant to the text that the user has entered.  Such a widget may modify the
host in several ways to achieve its functionality, \eg~it could install event
handlers to display advertisements when the user places the mouse over specific
phrases in the text. However, as an untrusted guest, this widget may also
contain malicious functionality, \eg~it could implement a clickjacking-style
attack by overlaying the editor with transparent HTML elements pointing to
malicious sites.

%-------------------------------------------------------------------------------
% Redid motivated example with reduced font sizes and in two-column format.
%-------------------------------------------------------------------------------
\begin{figure*}[t]%{rh}{0.60\textwidth}
%\vspace*{-3.5ex}
\setlength{\tabcolsep}{2pt}
\centering
%\scriptsize
\begin{tabular}{rl}
%\hline
\lno{1} & \htmltiny{script type="text/javascript"}\\
\lno{2} & \codetiny{var editor = new Editor(); initialize(editor);}\\
\lno{3} & \codetiny{var builtins = [], tocommit = true;}\\
\lno{4} & \codetiny{for(var prop in Editor.prototype) builtins[prop] = prop;}\\
\lno{5} & \codetiny{var tx = transaction \{} 
          \fbox{\textbf{\textrm{\tiny Guest code: Lines 6--9}}}\\
  \rowcolor[gray]{0.9}
\lno{6} & \mytab \codetiny{Editor.prototype.getKeywords = function(content)
\{...\}}\\
  \rowcolor[gray]{0.9}
        & \mytab \codetiny{...}\\
  \rowcolor[gray]{0.9}
\lno{7} & \mytab \codetiny{var elem = document.getElementById("editBox");}\\
  \rowcolor[gray]{0.9}
\lno{8} & \mytab \codetiny{elem.addEventListener("mouseover", displayAds,
false);}\\
  \rowcolor[gray]{0.9}
        & \mytab \codetiny{...}\\
  \rowcolor[gray]{0.9}
\lno{9} & \mytab \codetiny{document.write(`<div style="opacity:0.0;
z-index:0;
\textit{... size/loc params}">}\\
  \rowcolor[gray]{0.9}
         & \mytab \mytab \codetiny{<a href="http://evil.com"> Evil Link
</a></div>');}\\
\lno{10} & \codetiny{\};}\\
\lno{11} & \codetiny{tocommit = gotoIblock(tx);}
           \fbox{\textbf{\textrm{\tiny Implements the host's security
policies}}}\\
\lno{12} & \codetiny{if (tocommit) tx.commit();}\\
\lno{13} & \codetiny{... /* rest of the Host Web application's code */}\\
\lno{14} & \htmltiny{/script}\\
%\hline
\end{tabular}
\remove{
\mycaptionsquish{Motivating example. \describecaption{This example shows how a
hosting Web application can mediate untrusted guest code (lines 6--9). The
invocation to the introspection block (line 11) enforces the host's security
policies (see \figref{figure:iblock}) on the actions performed by the guest. To
ease exposition, 
%
(a)~this example omits variable hiding (a concept introduced in
\sectref{section:design:hiding}), which ensures that the guest cannot tamper
with sensitive host variables such as \code{tx}, \code{tocommit} and
\code{builtins};
%
(b)~we inlined the guest code from \code{http://untrusted.com/guest.js} in
lines~\lnobig{6}--\lnobig{9}. \sectref{section:implementation:auxiliary:script}
explains how the host can use transactions to mediate guests included using a
\code{<script>} tag. }}
}
\add{
\mycaptionsquish{Motivating example. \describecaption{This example shows how a
host can mediate an untrusted guest (lines 6--9). The 
introspection block (invoked in line 11) enforces the host's security
policies (see \figref{figure:iblock}) on the actions performed by the guest.}}
}
{\label{figure:motivate}}
%\vspace*{-3.0ex}
\end{figure*}

%-------------------------------------------------------------------------------

Traditional reference monitors \add{\cite{ulfarthesis}, which mediate the
action of guest code as it executes,} can detect and prevent such attacks.
However, such reference monitors typically only enforce access control policies,
and would have let the guest modify the host's heap and DOM (such as to install
innocuous event handlers) until the attack is detected.  When such a reference
monitor reports an attack, the end-user faces one of two unpalatable options:
(a)~close the editing session and start afresh; or (b)~continue with the
tainted editing session. In the former case, the end-user loses unsaved work.
In the latter case, the editing session is subject to the unknown and possibly
undesirable effects of the heap and DOM changes that the widget initiated
before being flagged as malicious. In our example, the event handlers
registered by the malicious widget may also implement undesirable functionality
and should be removed when the widget's clickjacking attempt is detected.

Speculative execution allows \add{hosts to introspect all actions of untrusted
guest code.} \remove{reference monitors to detect and recover from malicious
guests.} In our example, the host speculatively executes the untrusted
widget by enclosing it in a transaction. When the attack is detected, the
host simply discards all changes initiated by the widget. The end-user can
proceed with the editing session without losing unsaved work, and with the
assurance that the host is unaffected by the malicious widget.

% What are the main technical contributions of this paper?
This paper describes the \txjs\ system, that has the following novel features:
%
\begin{mylist}
%
\item \textbf{\js\ transactions.} \txjs\ allows hosting Web applications to
speculatively execute guests by enclosing them in transactions.  \txjs\
maintains \textit{read and write sets} for each transaction to record the
objects that are accessed and modified by the corresponding guest.  These sets
are exposed as properties of a \textit{transaction object} in \js.  Changes to
a \js\ object made by the guest are visible within the transaction, but any
accesses to that object from code outside the transaction return the unmodified
object.  The host can inspect such speculative changes made by the guest and
determine whether they conform to its security policies.  The host must
explicitly commit these changes in order for them to take effect; uncommitted
changes simply do not take and need not be undone explicitly.
%
\item \textbf{Transaction suspend/resume.} Guest code may attempt
operations outside the purview of the \js\ interpreter. In a browser,
these \textit{external operations} include
AJAX calls that send network requests, such as \code{XMLHttpRequest}.  \txjs\
introduces a \emph{suspend and resume} mechanism that affords unprecedented
flexibility to mediate external operations. Whenever a guest attempts an
external operation, \txjs\ suspends it and passes control to the host.
Depending on its security policy, the host can perform the action on behalf of
the guest, perform a different action unbeknownst to the guest, or buffer up
and simulate the action, before resuming this or another suspended transaction. 
%

\remove{Developers often use \js\ callbacks (or continuations) to
implement event driven or asynchronous Web 2.0 applications, which can quickly
lead to code blow-up. Suspend/resume enables a non-continuation passing style of
programming for asynchronous requests by implementing an implicit callback
invocation mechanism using resume.}
%
\item \textbf{Speculative DOM updates.} Because \js\ interacts heavily with the
DOM\@, \txjs\ provides a speculative DOM subsystem, which ensures that DOM
changes requested by a guest will also be speculative. Together with support for
\js\ transactions, \txjs's DOM subsystem allows hosts to cleanly recover from
attacks by malicious guests.
%
\end{mylist}

\txjs\ provides these features without restricting or modifying guest code in
any way. This allows reference monitors based on \txjs\ to mediate the actions
of legacy libraries and applications that contain constructs that are often
disallowed in safe \js\
subsets~\cite{adsafe,fbjs,caja,wrappers:esorics09,language:csf09}
(\eg~\code{eval}, \code{this} and \code{with}).

In the rest of the paper, we discuss the design, implementation and evaluation
of \txjs.

\remove{
We have implemented a prototype of \txjs\ by adding these features to
Firefox's \js\ interpreter.
We evaluated our prototype by using it to build
reference monitors to mediate several guests, the largest of which was a
jQuery-based application containing about $7,500$ lines of code. Our results
show that \txjs\ is expressive and effective at isolating untrusted guests. On
average, the use of \txjs\ increased the load time of these guests by only 0.16
seconds.
}

\begin{comment}
% Problem definition.
%
The client-side of modern Web applications frequently contains \js\ code in the
form of advertisements, games, and libraries that may be developed by untrusted
third parties. To include such third-party code (the \textit{guest}),
the author of such a Web application (the \textit{host}) must choose to
(a)~place the guest code in its own protection domain (\eg~an
\code{iframe}) or (b)~include the guest in the same protection domain as the
host. The first approach isolates the guest from the host, but has the downside
of offering limited means for guest/host interaction.  Any communication
between the guest and the host must happen via explicit \code{postMessage}
commands between the corresponding \code{iframes}.  The second approach eases
guest/host interaction, but also exposes the host to malicious guests. Hosts
must therefore employ protection mechanisms to confine the actions of guest
code. This paper addresses the problem of designing such protection mechanisms.

% Prior work to address the problem.
%
One way to protect against untrusted guests is to use \textit{\js\ reference
monitors}.  Reference monitors mediate the action of guest code as it executes
and ensure that any accesses by the guest to the host's data structures conform
to the host's security policies. Examples of reference monitors include
Caja~\cite{caja}, object capabilities~\cite{mmt10}, inlined reference
monitors~\cite{phung:asiaccs09,corescript:popl07,mets:hotos07}, object
wrappers~\cite{wrappers:esorics09}, and aspect-oriented policy
enforcement~\cite{conscript:oak10}. Some reference monitors also require the
guest code to be written in safe subsets of \js\ (\eg~Caja). The main purpose of
these reference monitors is to provide \textit{access control} to the host's
objects.

% What is our unique contribution to this space?
%
This paper proposes a novel approach to construct \js\ reference monitors using
speculative execution.  The main idea is to extend \js\ with a new
\textit{transaction} construct, within which hosts can speculatively execute
guest code that contains arbitrary \js\ constructs. In addition to enforcing
access control policies, reference monitors that use speculative execution
allow hosts to \textit{cleanly recover from the security-violating actions of
guest code}. When a host detects an offending guest, it simply chooses not to
commit the transaction corresponding to the guest. Such an approach neutralizes
any data and DOM modifications initiated earlier by the guest, without having
to undo them explicitly.

% Why is this contribution important? (i.e., motivating example).
%
To motivate the need for recovery, consider the example of a Web-based word
processor that hosts a third-party widget to display advertisements (see
\figref{figure:motivate}). During an editing session, this widget scans the
document for specific keywords and displays advertisements relevant
to the text that the user has entered.  Such a widget may modify the host in
several ways to achieve its functionality, \eg~it could install event handlers
to display advertisements when the user places the mouse over specific phrases
in the text. However, as an untrusted guest, this widget may also contain
malicious functionality, \eg~it could implement a clickjacking-style attack by
overlaying the editor with transparent HTML elements pointing to malicious
sites.

%-------------------------------------------------------------------------------
% Redid motivated example with reduced font sizes and in two-column format.
%-------------------------------------------------------------------------------
%\begin{figure*}[t!]
%\begin{tabular}{|ll|}
%\hline
%\begin{minipage}[b]{0.48\linewidth}
%\renewcommand{\arraystretch}{0.75}
%\begin{tabular}{rl}
%\lno{1} & \htmltiny{script type="text/javascript"}\\
%\lno{2} & \codetiny{var editor = new Editor(); initialize(editor);}\\
%\lno{3} & \codetiny{var builtins = [], tocommit = true;}\\
%\lno{4} & \codetiny{for(var prop in Editor.prototype) builtins[prop] = prop;}\\
%\lno{5} & \codetiny{var tx = transaction \{} 
%          \fbox{\textbf{\textrm{\tiny Guest code: Lines 6--9}}}\\
%  \rowcolor[gray]{0.9}
%\lno{6} & \mytab \codetiny{Editor.prototype.getKeywords = function(content)
%\{...\}}\\
%  \rowcolor[gray]{0.9}
%        & \mytab \codetiny{...}\\
%  \rowcolor[gray]{0.9}
%\lno{7} & \mytab \codetiny{var elem = document.getElementById("editBox");}\\
%  \rowcolor[gray]{0.9}
%\lno{8} & \mytab \codetiny{elem.addEventListener("mouseover", displayAds,
%false);}\\
%  \rowcolor[gray]{0.9}
%        & \mytab \codetiny{...}\\
%  \rowcolor[gray]{0.9}
%\lno{9} & \mytab \codetiny{document.write(`<div style="opacity:0.0;
%z-index:0;
%\textit{... size/loc params}">}\\
%  \rowcolor[gray]{0.9}
%         & \mytab \mytab \codetiny{<a href="http://evil.com"> Evil Link
%</a></div>');}\\
%\lno{10} & \codetiny{\};}\\
%\lno{11} & \codetiny{do \{} 
%           \fbox{\textbf{\textrm{\tiny Introspection block (iblock): Lines
%11--27}}}\\
%\lno{12} & \mytab \codetiny{var arg = tx.getArgs();~~~var obj =
%tx.getObject();}\\
%\lno{13} & \mytab \codetiny{var rs = tx.getReadSet();~var ws =
%tx.getWriteSet();}\\
%\end{tabular}
%\end{minipage} &
%\begin{minipage}[b]{0.5\linewidth}
%\renewcommand{\arraystretch}{0.75}
%\begin{tabular}{rl}
%\lno{14} & \mytab \codetiny{for(var i in builtins) \{}\\
%\lno{15} & \mytab \mytab \codetiny{if (ws.checkMembership(Editor.prototype,
%builtins[i])) tocommit = false;}\\
%\lno{16} & \mytab \codetiny{\} ... /* definition of `IsClickJacked' to go here
%*/}\\
%\lno{17} & \mytab \codetiny{if (IsClickJacked(tx.getTxDocument())) tocommit =
%false;}\\
%\lno{18} & \mytab \codetiny{... /* more policy checks go here */}\\
%  \rowcolor[gray]{0.9}
%\lno{19} & \mytab \codetiny{switch(tx.getCause()) \{}\\
%  \rowcolor[gray]{0.9}
%\lno{20} & \mytab \mytab \codetiny{case "addEventListener":}\\
%  \rowcolor[gray]{0.9}
%\lno{21} & \mytab \mytab \mytab \codetiny{var txHandler =
%MakeTxHandler(arg[1]);}\\
%  \rowcolor[gray]{0.9}
%\lno{22} & \mytab \mytab \mytab \codetiny{obj.addEventListener(arg[0],
%txHandler, arg[2]); break;}\\
%  \rowcolor[gray]{0.9}
%\lno{23} & \mytab \mytab \codetiny{case "write": WriteToTxDOM(obj, arg[0]);
%break; ... /* more cases */}\\
%  \rowcolor[gray]{0.9}
%\lno{24} & \mytab \mytab \codetiny{default: break;}\\
%  \rowcolor[gray]{0.9}
%\lno{25} & \mytab \codetiny{\}; tx = tx.resume();}\\
%  \rowcolor[gray]{0.9}
%\lno{26} & \codetiny{\} while(tx.isSuspended());}\\
%\lno{27} & \codetiny{if (tocommit) tx.commit();}\\
%\lno{28} & \codetiny{... /* rest of the Host Web application's code */}\\
%\lno{29} & \htmltiny{/script}\\
%\end{tabular}
%\end{minipage}\\
%\hline
%\end{tabular}
%\mycaptionsquish{Motivating example. \describecaption{This example shows how a
%hosting Web application can mediate untrusted guest code (lines 6--9). The
%introspection block (lines 11--27) enforces the host's security policies on the
%actions performed by the guest. To ease exposition, 
%
%(a)~this example omits variable hiding, a concept introduced in
%\sectref{section:design:hiding}, which ensures that the guest cannot tamper
%with sensitive host variables such as \code{tx}, \code{tocommit} and
%\code{builtins};
%
%(b)~we inlined the guest code from \code{http://untrusted.com/guest.js} in
%lines~\lnobig{6}--\lnobig{9}.
%\sectref{section:implementation:auxiliary:script} explains how the host can use
%transactions to mediate guests included using a \code{<script>} tag.
%}}{\label{figure:motivate}}
%\end{figure*}

%-------------------------------------------------------------------------------
% Redid motivated example with reduced font sizes and in two-column format.
%-------------------------------------------------------------------------------
\begin{wrapfigure}{rh}{0.55\textwidth}
\centering
\scriptsize
\begin{tabular}{|rl|}
\hline
\lno{1} & \htmltiny{script type="text/javascript"}\\
\lno{2} & \codetiny{var editor = new Editor(); initialize(editor);}\\
\lno{3} & \codetiny{var builtins = [], tocommit = true;}\\
\lno{4} & \codetiny{for(var prop in Editor.prototype) builtins[prop] = prop;}\\
\lno{5} & \codetiny{var tx = transaction \{} 
          \fbox{\textbf{\textrm{\tiny Guest code: Lines 6--9}}}\\
  \rowcolor[gray]{0.9}
\lno{6} & \mytab \codetiny{Editor.prototype.getKeywords = function(content)
\{...\}}\\
  \rowcolor[gray]{0.9}
        & \mytab \codetiny{...}\\
  \rowcolor[gray]{0.9}
\lno{7} & \mytab \codetiny{var elem = document.getElementById("editBox");}\\
  \rowcolor[gray]{0.9}
\lno{8} & \mytab \codetiny{elem.addEventListener("mouseover", displayAds,
false);}\\
  \rowcolor[gray]{0.9}
        & \mytab \codetiny{...}\\
  \rowcolor[gray]{0.9}
\lno{9} & \mytab \codetiny{document.write(`<div style="opacity:0.0;
z-index:0;
\textit{... size/loc params}">}\\
  \rowcolor[gray]{0.9}
         & \mytab \mytab \codetiny{<a href="http://evil.com"> Evil Link
</a></div>');}\\
\lno{10} & \codetiny{\};}\\
\lno{11} & \codetiny{tocommit = gotoIblock(tx);}
           \fbox{\textbf{\textrm{\tiny Implements the host's security
policies}}}\\
\lno{12} & \codetiny{if (tocommit) tx.commit();}\\
\lno{13} & \codetiny{... /* rest of the Host Web application's code */}\\
\lno{14} & \htmltiny{/script}\\
\hline
\end{tabular}
\mycaptionsquish{Motivating example. \describecaption{This example shows how a
hosting Web application can mediate untrusted guest code (lines 6--9). The
invocation to the introspection block (line 11) enforces the host's security
policies (see \figref{figure:iblock}) on the actions performed by the guest. To
ease exposition, 
%
(a)~this example omits variable hiding (a concept introduced in
\sectref{section:design:hiding}), which ensures that the guest cannot tamper
with sensitive host variables such as \code{tx}, \code{tocommit} and
\code{builtins};
%
(b)~we inlined the guest code from \code{http://untrusted.com/guest.js} in
lines~\lnobig{6}--\lnobig{9}. \sectref{section:implementation:auxiliary:script}
explains how the host can use transactions to mediate guests included using a
\code{<script>} tag. }}
{\label{figure:motivate}}
\end{wrapfigure}

%-------------------------------------------------------------------------------

Traditional reference monitors can detect and prevent such attacks.  However,
such reference monitors typically only enforce access control policies, and
would have let the guest modify the host's heap and DOM (such as to install
innocuous event handlers) until the attack is detected.  When such a reference
monitor reports an attack, the end-user faces one of two unpalatable options:
(a)~close the editing session and start afresh; or (b)~continue with the
tainted editing session. In the former case, the end-user loses unsaved work.
In the latter case, the editing session is subject to the unknown and possibly
undesirable effects of the heap and DOM changes that the widget initiated
before being flagged as malicious. In our example, the event handlers
registered by the malicious widget may also implement undesirable functionality
and should be removed when the widget's clickjacking attempt is detected.

Speculative execution allows reference monitors to detect and
recover from malicious guests. In our example, the host would speculatively
execute the untrusted widget by enclosing it within a transaction. When the
attack is detected, the host simply discards all changes initiated by the
widget. The end-user can proceed with the editing session without losing
unsaved work, and with the assurance that the host is unaffected by the
malicious widget.

% What are the main technical contributions of this paper?
This paper describes \txjs, a speculative execution system for \js\ with
several novel features:
%
\begin{mylist}
%
\item \textbf{\js\ transactions.} \txjs\ allows hosting Web applications to
speculatively execute guests by enclosing them in transactions.  \txjs\
maintains \textit{read and write sets} for each transaction to record the
objects that are accessed and modified by the corresponding guest.  These sets
are exposed as properties of a \textit{transaction object} in \js.  Changes to
a \js\ object made by the guest are visible within the transaction, but any
accesses to that object from code outside the transaction return the unmodified
object.  The host can inspect such speculative changes made by the guest and
determine whether they conform to its security policies.  The host must
explicity commit these changes in order for them to take effect; uncommitted
changes simply do not take and need not be undone explicitly.
%
\item \textbf{Transaction suspend/resume.} Guest code may attempt
operations outside the purview of the \js\ interpreter. In a browser,
these \textit{external operations} include
AJAX calls that send network requests, such as \code{XMLHttpRequest}.  \txjs\
introduces a \emph{suspend and resume} mechanism that affords unprecedented
flexibility to mediate external operations. Whenever a guest attempts an
external operation, \txjs\ suspends it and passes control to the host.
Depending on its security policy, the host can perform the action on behalf of
the guest, perform a different action unbeknownst to the guest, or buffer up
and simulate the action, before resuming this or another suspended transaction. 
%
\item \textbf{Speculative DOM updates.} Because \js\ interacts heavily with the
DOM\@, \txjs\ provides a speculative DOM subsystem, which ensures that DOM
changes requested by a guest will also be speculative.  Together with support
for \js\ transactions, \txjs's DOM subsystem allows hosts to cleanly recover
from attacks by malicious guests.
%
\end{mylist}

\txjs\ provides these features without restricting or modifying guest code in
any way. This allows reference monitors based on \txjs\ to mediate the actions
of legacy libraries and applications that contain constructs that are often
disallowed in safe \js\
subsets~\cite{adsafe,fbjs,caja,wrappers:esorics09,language:csf09}
(\eg~\code{eval}, \code{this} and \code{with}).

We have implemented a prototype of \txjs\ by adding these features to
Firefox's \js\ interpreter.
We evaluated our prototype by using it to build
reference monitors to mediate several guests, the largest of which was a
jQuery-based application containing about $7,500$ lines of code. Our results
show that \txjs\ is expressive and effective at isolating untrusted guests. On
average, the use of \txjs\ increased the load time of these guests by only 0.16
seconds.
\end{comment}