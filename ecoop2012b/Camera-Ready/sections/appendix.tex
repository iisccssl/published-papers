%\mysection{Event-handler Wrapper Generators}
%\label{section:appendix:events}
%%
%\begin{wrapfigure}{rh}{0.5\textwidth}
%\indent\vspace{-0.5cm}
%\begin{center}
%\indent\vspace{-0.5cm}
%\renewcommand{\arraystretch}{0.75}
%\begin{tabular}{|rl|}
%\hline
%\lno{1} & \codetiny{tx\_clkhandler = function(evt) \{}\\
%\lno{2} & \mytab \codetiny{evttx = transaction \{ node.evtH (evt); \}
%iblock\_func (evttx);}\\
%\lno{3} & \codetiny{\}}\\
%\hline
%\end{tabular}
%\end{center}
%\mycaptionsquish{Generating wrapped event handlers.}{\label{figure:evthandler}}
%\end{wrapfigure}
%
%%This section describes the implementation of the event handler wrapper
%%generator, which is included in the mandatory part of a host's iblock.
%Firefox currently supports three event-handling models~\cite{dom-level}.  For
%each model, the goal of the wrapper generator is to obtain a reference to the
%handler being registered, and wrap it suitably.  We describe the three models
%briefly and discuss how \txjs\ obtains a reference to the handler in each
%case.
%
%\noindent \textbf{(1)} The \textit{DOM-level 0/Traditional} model registers an
%event handler for a DOM node, \eg~a \html{div} element, as follows:
%\code{node.onclick = clkhandler}.  Here \code{node} represents the \html{div}
%element and \code{clkhandler} is registered as a handler for \code{onclick}
%events.  \txjs\ modifies the interpreter to suspend transactions that change
%properties containing event handlers, such as \code{onclick} and \code{onload}.
%Once the transaction suspends, the iblock obtains a reference to
%\code{clkhandler}.
%
%\noindent \textbf{(2)} The \textit{DOM-level 0/Inline} model registers an event
%handler using code such as: 
%
%\code{document.write("\html{div onclick="/*\textrm{handler code}*/}")}, 
%
%which sets an attribute (\eg~\code{onclick}) of the DOM node.  \txjs\ handles
%such cases by suspending the execution of \code{document.write}.  The argument
%to this call is HTML code, which the transaction's iblock parses to obtain a
%reference to the event handler.
%
%\noindent \textbf{(3)} The \textit{DOM-level 2} model registers an event
%handler as follows: \code{node.addEventListener("click", clkhandler, false)}.
%\txjs\ suspends \code{addEventListener}, thereby allowing the iblock to obtain
%a reference to \code{clkhandler}.
%
%
%In addition to obtaining a reference to the event handler, the iblock also
%obtains a reference to \code{node}, which is the DOM node for which the handler
%was being registered. The iblock then initializes a new property
%\code{node.evtH} with \code{clkhandler}, and defines a new function
%\code{tx\_clkhandler} as shown in \figref{figure:evthandler}, which it
%registers as the event handler.
%Here, \code{iblock\_func} is a function that
%contains the iblock itself, while \code{evt} is a \js\ object that
%the browser uses to denote the event. As a result of this transformation,
%\code{tx\_clkhandler} is invoked when the \code{onclick} event is triggered,
%which then executes \code{clkhandler} within a transaction, thereby allowing
%\txjs\ to mediate its operation as well.
%
%
\mysection{Non-tail-recursive Interpreters}
\label{section:appendix:tail-call}

%
\begin{wrapfigure}{rh}{0.5\textwidth}
%\vspace*{-6.0ex}
%
\tikzset{>=stealth,
    frame/.style={minimum width=6.5em, text height=1.7ex, text depth=.2ex, draw,
outer ysep=0}}
%
\begin{tabular}{|c|}
%
\hline
\begin{minipage}[b]{0.45\textwidth}
\begin{center}
\renewcommand{\arraystretch}{0.75}
\begin{tabular}{rl}
\lno{1} & \codetiny{function f() \{ document.body.appendChild(...); \}}\\
\lno{2} & \codetiny{var tx = transaction \{ f(); \}} \\
\lno{3} & \codetiny{g(tx);} \\
\end{tabular}
\end{center}
% \smallskip
\scriptsize 
\begin{tabular}{p{0.95\textwidth}}
\textbf{(a)} Problematic code for an interpreter with non-tail
recursion.
\end{tabular}
\end{minipage}\\
%
%\hline
%
%\begin{minipage}[b]{0.45\textwidth}
%\begin{center}
%\scriptsize
%\begin{tikzpicture}
%    \node (top) [frame,inner xsep=1em,outer xsep=0] {call to
%\textsf{JS\_interpret}};
%    \draw [dashed] (top.south west) -- ++(0,-.5);
%    \draw [dashed] (top.south east) -- ++(0,-.5);
%    \path (top.north)++(0,1ex) node [anchor=base] {\textit{Native (C++)
%stack}};
%    \path (top)++(11em,0) node (js1) [frame] {\textbf{\code{tx} delimiter}};
%    \path (js1.north)++(0,1ex) node [anchor=base] {\textit{\js\ stack}};
%    \path (js1.south) node (js0) [below,frame] {main program};
%    \draw [->] (top.mid east)++(-1ex,0) -- (js1.mid west);
%\end{tikzpicture}
%\end{center}
%\scriptsize 
%\begin{tabular}{p{0.95\textwidth}}
%\textbf{(b)} When the main \js\ program starts the transaction, the C++
%function \textsf{JS\_interpret} grows the \js\ stack but does not call itself,
%so the native stack does not grow.
%\end{tabular}
%\smallskip
%\end{minipage}\\
%
\hline
%
\begin{minipage}[b]{0.45\textwidth}
\begin{center}
\scriptsize
\begin{tikzpicture}
    \node (top) [frame,inner xsep=1em,outer xsep=0] {call to
\textsf{JS\_interpret}};
    \draw [dashed] (top.south west) -- ++(0,-.5);
    \draw [dashed] (top.south east) -- ++(0,-.5);
    \path (top.north)++(0,1ex) node [anchor=base] {\textit{Native (C++) stack}};
    \path (top)++(11em,0) node (js2) [frame] {call to \verb|f|};
    \path (js2.north)++(0,1ex) node [anchor=base] {\textit{\js\ stack}};
    \path (js2.south) node (js1) [below,frame] {\textbf{\code{tx} delimiter}};
    \path (js1.south) node (js0) [below,frame] {main program};
    \draw [->] (top.mid east)++(-1ex,0) -- (js2.mid west);
    \node (removed) [fit=(js2) (js1), right delimiter=\}, inner sep=0] {};
    \path (removed.east) node [anchor=north,rotate=90] {removed\strut};
\end{tikzpicture}
\end{center}
\smallskip
\scriptsize 
\begin{tabular}{p{0.95\textwidth}}
\textbf{(b)} When the transaction suspends, the interpreter
removes activation records from the front of the \js\ stack, up to and
including the (youngest) transaction delimiter.\\
\end{tabular}
\end{minipage}\\
%
\hline
%
\begin{minipage}[b]{0.45\textwidth}
\begin{center}
\scriptsize
\begin{tikzpicture}
    \node (top) [frame,inner xsep=1em,outer xsep=0] {call to
\textsf{JS\_interpret}};
    \draw [dashed] (top.south west) -- ++(0,-.5);
    \draw [dashed] (top.south east) -- ++(0,-.5);
    \path (top.north)++(0,1ex) node [anchor=base] {\textit{Native (C++) stack}};
    \path (top)++(11em,0) node (js1) [frame] {call to \verb|g|};
    \path (js1.north)++(0,1ex) node [anchor=base] {\textit{\js\ stack}};
    \path (js1.south) node (js0) [below,frame] {main program};
    \draw [->] (top.mid east)++(-1ex,0) -- (js1.mid west);
    \path (js0.south west)++(-8ex,-3ex)
        node (object) [draw,anchor=north west,outer ysep=0]
        {\begin{tabular}[t]{@{}c@{}}$\vdots$\\read set\\write set\end{tabular}};
    \draw [->] (js1.mid east)++(-1ex,0) .. controls +(2,-.3) and +(3,.4)
                                        .. (object.north east);
    \path (object.north east)++(2ex,0) node (tx1) [anchor=north west,frame]{call
to \verb|f|};
    \path (tx1.south) node (tx0) [below,frame] {\textbf{\code{tx} delimiter}};
    \draw [->] (tx1.mid -| object.east)++(-1ex,0) -- (tx1.mid west);
    \node [fit=(object), inner sep=0, outer xsep=1.2ex,
           left delimiter=\{,
label=left:\begin{tabular}{@{}r@{}}transaction\\object\end{tabular}\enspace] {};
\end{tikzpicture}
\end{center}
%\smallskip
\scriptsize 
\begin{tabular}{p{0.95\textwidth}}
\textbf{(c)} Before resuming the transaction, the main program may invoke other
functions, say~\texttt{g}.\\
\end{tabular}
\smallskip
\end{minipage}\\
%
\hline
%
\begin{minipage}[b]{0.45\textwidth}
\begin{center}
\scriptsize
\begin{tikzpicture}
    \node (top) [frame,inner xsep=1em,outer xsep=0] {call to
\textsf{JS\_interpret}};
    \draw [dashed] (top.south west) -- ++(0,-.5);
    \draw [dashed] (top.south east) -- ++(0,-.5);
    \path (top.north)++(0,1ex) node [anchor=base] {\textit{Native (C++) stack}};
    \path (top)++(11em,0) node (js2) [frame] {call to \verb|f|};
    \path (js2.north)++(0,1ex) node [anchor=base] {\textit{\js\ stack}};
    \path (js2.south) node (js1) [below,frame] {\textbf{\code{tx} delimiter}};
    \path (js1.south) node (jsg) [below,frame] {call to \verb|g|};
    \path (jsg.south) node (js0) [below,frame] {main program};
    \draw [->] (top.mid east)++(-1ex,0) -- (js2.mid west);
    \node (reinstated) [fit=(js2) (js1), right delimiter=\}, inner sep=0] {};
    \path (reinstated.east) node [anchor=north,rotate=90] {reinstated\strut};
\end{tikzpicture}
\end{center}
\scriptsize 
\begin{tabular}{p{0.95\textwidth}}
\textbf{(d)} When the transaction is resumed, its activation records are
reinstated onto the front of the \js\ stack.\\
\end{tabular}
\end{minipage}\\
%
\hline
\begin{minipage}[b]{0.45\textwidth}
\begin{center}
\scriptsize
\begin{tikzpicture}
    \node (top) [frame,inner xsep=1em,outer xsep=0] {call to
\textsf{JS\_interpret}};
    \path (top.south) node (bad) [below,frame,inner xsep=1em,outer xsep=0]{call
to \textsf{JS\_interpret}};
    \draw [dashed] (bad.south west) -- ++(0,-.5);
    \draw [dashed] (bad.south east) -- ++(0,-.5);
    \path (top.north)++(0,1ex) node [anchor=base] {\textit{Native (C++) stack}};
    \path (top)++(11em,0) node (js2) [frame] {call to \verb|f|};
    \path (js2.north)++(0,1ex) node [anchor=base] {\textit{\js\ stack}};
    \path (js2.south) node (js1) [below,frame] {\textbf{\code{tx} delimiter}};
    \path (js1.south) node (js0) [below,frame] {main program};
    \draw [->] (top.mid east)++(-1ex,0) -- (js2.mid west);
    \draw [->] (bad.mid east)++(-1ex,0) -- (js1.mid west);
    \node (to remove) [fit=(js2) (js1), right delimiter=\}, inner sep=0, outer
xsep=1.1ex] {};
    \path (to remove.east) node [anchor=north,rotate=90] {to remove\strut};
\end{tikzpicture}
\end{center}
\scriptsize 
\begin{tabular}{p{0.95\textwidth}}
\textbf{(e)} If \textsf{JS\_interpret} were to implement \js\
function calls by calling itself recursively (as happens in the implementation
of certain constructs, such as \code{eval}), an older call to
\textsf{JS\_interpret} (the lower one in this diagram) would need to return
before a younger one does. Control flow in C++ is not flexible enough to enable
this.\\
\end{tabular}
\end{minipage}\\
\hline
\end{tabular}
\mycaptionsquish{Native versus \js\ call stacks.  %\describecaption{The
%implementation of suspend/resume assumes an interpreter without non-tail
%recursion.}
}
{\label{figure:stacks}}
%\vspace*{-8.5ex}
\end{wrapfigure}
%

%
A key challenge in enhancing a legacy \js\ interpreter, such as SpiderMonkey,
with support for transactions is in how the interpreter uses recursion.  To
support the suspend/resume mechanism for switching control flow between a
transaction and its iblock, the interpreter must not accumulate any activation
records in its native stack (\eg~the C++ stack, for SpiderMonkey) between when
a transaction starts and when it suspends.  In particular, the interpreter must
not represent \js\ function calls by C++ function calls. The same issue also
arises when a compiler or JIT interpreter is used to turn \js\ code into
machine code.

To illustrate this point, consider SpiderMonkey, which implements the bytecode
interpreter in C++.  The main entry point to the bytecode interpreter is the
C++ function \textsf{JS\_interpret}, which maintains the \js\ stack as a linked
list of activation records, each of which is a C++ structure. When one
function calls another in \js,
%
 the \textsf{JS\_interpret} function does not
call itself in C++; instead, it adds a new activation record to the front of
the linked list and continues with the same bytecode interpreter loop as
before.  Similarly, when a function returns to another in \js,
\textsf{JS\_interpret} does not return in C++; instead, it removes an old
activation record from the front of the linked list and continues with the same
bytecode interpreter loop as before.  For the most part, SpiderMonkey does not
represent \js\ calls by C++ calls.

The fact that SpiderMonkey does not represent \js\ calls by native calls helps
us add transactions to it without making invasive changes, as the following
example illustrates.  Suppose a transaction invokes a function \code{f} that
suspends for some reason, \eg~in \figref{figure:stacks}(a), the function
\code{f} calls \code{appendChild}.  If the C++ call to \textsf{JS\_interpret}
that executes the transaction were not same as the one that executes the called
function \code{f}, then the former, although older, would have to return before
the latter returns. As detailed in \figref{figure:stacks}, the former has to
return when suspending the transaction, whereas the latter has to return when
resuming the transaction.  Even exception handling in C++ does not allow such
control flow.

Unfortunately, \textsf{JS\_interpret} in SpiderMonkey does call itself in a few
situations.  For example, it handles the \code{eval} construct in this way, and
the problem of the C++ stack in Figure \ref{figure:stacks}(e) does arise if we
replace the \code{document.body.appendChild(...)} of
\figref{figure:stacks}(a) by 
%
\code{eval\linebreak[0]("document.\linebreak[0]body.\linebreak[0]%
appendChild\linebreak[0](...)")}.
%
One way to solve this problem requires applying the
continuation-passing-style transformation to the interpreter to put it into tail
form, \ie~convert all recursive calls to \textsf{JS\_interpret} to tail calls.
However, this transformation is invasive, especially if done manually on legacy
interpreters.

\txjs\ uses a less invasive mechanism to enable suspend/resume in SpiderMonkey.
This mechanism is similar in functionality to gluing (see
\sectref{section:design:iblock:gluing}), and we explain it with an example.
Consider the \code{eval} construct, whose functionality is to parse its input
string, compile it into bytecode, and then execute the bytecode as usual.
Because only the last step, \ie~that of executing the bytecode, can suspend, we
simply changed the behavior of \code{eval} so that, if invoked inside a
transaction, it suspends the transaction right away.  The iblock of the
transaction can then compile the string into bytecode and include the bytecode
into the execution of the transaction. This is achieved by adding a new
activation record to the front of the transaction's \js\ stack and modifying
the program counter to execute this code when the transaction resumes.  When
the suspended transaction resumes, it transfers control to the \code{eval}ed
code, which can freely suspend.  Besides \code{eval}, our current \txjs\
prototype also implements gluing for \code{document.write} (as discussed in
\sectref{section:design:iblock:gluing}) and \js\ builtins \code{call} and
\code{apply}, which make non-tail recursive calls to \textsf{JS\_interpret}.
