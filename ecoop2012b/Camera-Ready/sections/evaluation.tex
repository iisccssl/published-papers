\mysection{Evaluation}
\label{section:evaluation}

We evaluated four aspects of \txjs. First, in \sectref{section:evaluation:cases}
we study the applicability of \txjs\ to real-world guests, which varied in size
from about $1,\!400$ to $7,\!500$ lines of code. Second, we show in
\sectref{section:evaluation:recovery} that a host that uses \txjs\ can protect
itself and recover gracefully from malicious and buggy guests. Third, we report
 a performance evaluation of \txjs\ in \sectref{section:evaluation:performance}.
Last, in \sectref{section:evaluation:policy}, we study the complexity of
writing policies for \txjs. All experiments were performed with Firefox
v3.7a4pre on a 2.33Ghz Intel Core2 Duo machine with 3GB RAM and running Ubuntu
7.10.
\remove{We turned off \js\ JIT compilation support in the browser during
these experiments.}

\mysubsection{Case studies on guest benchmarks} 
\label{section:evaluation:cases}

\begin{wrapfigure}{rh}{0.45\textwidth}
%\vspace*{-6.0ex}
\setlength{\tabcolsep}{2pt}
\centering
\scriptsize
\begin{tabular}{|rl|c|c|}
\hline
& \textbf{Benchmark}      & \textbf{Size (LoC)} & \textbf{\html{script} tags}\\
%  & \textbf{Examples of policies enforced} \\ 
\hline
%
\lno{1}  & \js\ Menu~\cite{jsmenu}       & $1,\!417$  & $1$ \\
%& Disallow all network and cookie access \\
%
\lno{2}  & Picture Puzzle~\cite{picpuz}  & $1,\!709$      & $3$ \\
%  & Disallow attaching event handlers for keys events \\ 
%
\lno{3}  & GoogieSpell~\cite{gspell}     & $2,\!671$      & $4$ \\
%  & Disallow \code{XMLHttpRequest} if domain cookies were read \\ 
%
\lno{4}  & GreyBox~\cite{gbox}           & $2,\!338$      & $7$ \\
%  & Create iframes to whitelisted URLs only \\ 
%
\lno{5}  & Color Picker~\cite{cpicker}   & $7,\!543$      & $6$ \\
%  & Prevent modification to \code{innerHTML} property of arbitrary \html{div}
%elements\\ 
\hline
\end{tabular}
\mycaption{Guest benchmarks. \describecaption{We used transactions
to isolate each of these benchmarks from a simple hosting Web page.}}
{\label{figure:macrobenchmarks}}%\indent\vspace{-0.5cm}
%\vspace*{-5.0ex}
\end{wrapfigure}

%------------------------------------------------------------------------------

To evaluate \txjs's applicability to real-world guests, we experimented
with five \js\ applications, shown in \figref{figure:macrobenchmarks}.  For
each guest benchmark in \figref{figure:macrobenchmarks}, we played the role of
a host developer attempting to include the guest into the host, \ie~we created
a Web page and included the code of the guest into the page using \html{script}
tags. Most of the guests were implemented in several files; the \html{script}
column in \figref{figure:macrobenchmarks} shows the number of \html{script}
tags that we had to use to include the guest into the host. We briefly describe
these guest benchmarks and the domain-specific policies that were implemented
for each iblock.

\begin{figure*}[t]
%\vspace*{-6.0ex}
\setlength{\tabcolsep}{2pt}
\centering
\renewcommand{\arraystretch}{0.65}
\begin{tabular}{|rl||rl|}
\hline
\lno{1}  & \htmltiny{script src="jsMenu.js" func="menu"}\htmltiny{/script} &
\lno{5}  & \mytab \codetiny{var tx = transaction \{
e(getFunctionBody(menu));\}}\\
\lno{2}  & \htmltiny{script src="libTranscript.js}\htmltiny{/script} & \lno{6} &
\mytab \codetiny{to\_commit = gotoIblock(tx);}\\
\lno{3}  & \htmltiny{script}\codetiny{(function () \{} & \lno{7} & \mytab
\codetiny{if(to\_commit) tx.commit();}\\
\lno{4}  & \mytab \codetiny{var to\_commit = true, e = eval; \mytab // indirect
eval} & \lno{8} & \codetiny{\})(); }\htmltiny{/script}\\
\hline
\end{tabular}
\mycaptionsquish{Confining \js\ Menu.
\describecaption{%This figure illustrates several concepts:
%
(a)~lines~\lnobig{1} and \lnobig{5} demonstrate the enhanced \html{script} tag
and the host's use of indirect \code{eval} to include the guest, which is
compiled into a function (called \code{menu}; line~\lnobig{1})
(\sectref{section:implementation:auxiliary:script}). \code{getFunctionBody}
extracts the code of the function \code{menu};
%
(b)~line~\lnobig{3} implements variable hiding
(\sectref{section:design:hiding}), making \code{tx} invisible to the guest;
%
(c)~our supporting library \code{libTranscript} (line~\lnobig{2}) implements
%\code{processDOMEvents} (called on line~\lnobig{7}). This function processes
%the guest's DOM operations that trap into the iblock.
the mandatory part of the iblock and is invoked from \code{gotoIblock}.
}}
{\label{figure:txwebapp}} 
%\vspace*{-6.0ex}
%
\end{figure*}

%\begin{wrapfigure}{rh}{0.45\textwidth}
%\vspace*{-6.0ex}
%\setlength{\tabcolsep}{2pt}
%\centering
%\renewcommand{\arraystretch}{0.65}
%\begin{tabular}{|rl|}
%\hline
%\lno{1}  & \htmltiny{script src="jsMenu.js" func="menu"}\htmltiny{/script}\\
%\lno{2}  & \htmltiny{script src="libTranscript.js}\htmltiny{/script}\\
%\lno{3}  & \htmltiny{script}\codetiny{(function () \{} \\
%\lno{4}  & \mytab \codetiny{var to\_commit = true, e = eval; \mytab // indirect
%eval} \\
%\lno{5}  & \mytab \codetiny{var tx = transaction \{
%e(getFunctionBody(menu));\}}\\
%%\lno{6} &  \mytab \codetiny{do \{ ... <application-specific-policies> ... }\\
%%\lno{7} &  \mytab \mytab \codetiny{processDOMEvents(tx);}\\
%%\lno{8} & \mytab \mytab \codetiny{tx = tx.resume();}\\
%%\lno{9} & \mytab \codetiny{\} while(tx.isSuspended());}\\
%\lno{6} & \mytab \codetiny{to\_commit = gotoIblock(tx);}\\
%\lno{7} & \mytab \codetiny{if(to\_commit) tx.commit();}\\
%\lno{8} & \codetiny{\})(); }\htmltiny{/script}\\
%\hline
%\end{tabular}
%\mycaptionsquish{Confining \js\ Menu.
%\describecaption{%This figure illustrates several concepts:
%%
%(a)~lines~\lnobig{1} and \lnobig{5} demonstrate the enhanced \html{script} tag
%and the host's use of indirect \code{eval} to include the guest, which is
%compiled into a function (called \code{menu}; line~\lnobig{1})
%(\sectref{section:implementation:auxiliary:script}). \code{getFunctionBody}
%extracts the code of the function \code{menu};
%%
%(b)~line~\lnobig{3} implements variable hiding
%(\sectref{section:design:hiding}), making \code{tx} invisible to the guest;
%%
%(c)~our supporting library \code{libTranscript} (line~\lnobig{2}) implements
%%\code{processDOMEvents} (called on line~\lnobig{7}). This function processes
%%the guest's DOM operations that trap into the iblock.
%the mandatory part of the iblock and is invoked from \code{gotoIblock}.
%}}
%{\label{figure:txwebapp}} 
%\vspace*{-6.0ex}
%%
%\end{wrapfigure}

\noindent
\textit{(1) \js\ Menu} is a standalone widget that implements pull-down
menus. \figref{figure:txwebapp} shows how we confined \js\ Menu using \txjs.
The iblock for \js\ menu enforced a  policy that disallowed the guest
from accessing the network (\code{XMLHttpRequest}) or domain cookies.

\js\ Menu makes extensive use of \code{document.write} to build menus,
with several of these calls used to register event handlers, as shown below
%in the snippet in \figref{figure:jsmenu-evthdlr} 
(event handler registrations are shown in bold). Each \code{document.write}
call causes the transaction to suspend and pass control to the iblock. The
iblock uses \code{document.parse} to (a)~parse the arguments to identify the
HTML element(s) being created; (b)~identify whether any event handlers are being
registered and wrap them; and (c)~write resulting HTML to the transaction's
speculative DOM.
\remove{
%
%\begin{figure}[t!]
\begin{center}
%\vspace*{-1.0ex}
\setlength{\tabcolsep}{2pt}
\renewcommand{\arraystretch}{0.65}
\begin{tabular}{|p{0.6\textwidth}|}
\hline
\lno{1}~\codetiny{document.write("<" + blInfo.divType + ' id="' + mName);}\\
\lno{2}~\codetiny{document.write('" class="' + mClass + '" ');}\\
\lno{3}~\codetiny{document.write(setStyle(blInfo.divType, blInfo.fontSize, ... ,
"inherit"));}\\
\lno{4}~\codetiny{document.write(' \textbf{onclick="' + mAction + '"} ');}\\
\lno{5}~\codetiny{... // more code elided ...}\\
\lno{6}~\codetiny{document.write('\textbf{onmouseover="popMenu('+mLevel+",
'"+mName+"','"+mPopup+"'"+');"} ');}\\
\lno{7}~\codetiny{document.write('\textbf{onmouseout="setColorPassive(' + "'" +
mName + "'" + ');"}>');}\\
\hline
\end{tabular}
\end{center}
}
%\vspace*{-1.0ex}
%\mycaptionsquish{Code from \js\ Menu.}{\label{figure:jsmenu-evthdlr}}
%\end{figure}

\noindent
\textit{(2) Picture Puzzle} uses the drag-and-drop features provided by the
AJS \js\ library~\cite{ajs} to build an application that prompts the user to
arrange jumbled pieces of a picture within a $3\times3$ grid (we adapted this
benchmark from \cite{picpuz}).  We ran the benchmark within a transaction and
enforced a domain-specific security policy that prevented the transaction from
committing its changes if it attempted to install a handler to capture the
user's keystrokes (\eg~any event with \code{onkey} as a substring).

\noindent
\textit{(3) GoogieSpell} extends the AJS library to provide a spell-checking
service. When a user clicks the ``check spelling'' button, GoogieSpell sends an
\code{XMLHttpRequest} to a third-party server to fetch suggestions for
misspelled words. We created a transactional version of GoogieSpell, whose
iblock implemented a domain-specific policy that prevents an
\code{XMLHttpRequest} once the benchmark has read domain cookies or if the
target URL of \code{XMLHttpRequest} does not appear on a
whitelist.\footnote{Such \textit{cross-origin resource sharing} permits
cross-site \code{XMLHttpRequest}s, and is supported by Firefox-3.5 and
higher~\cite{cross-origin-xhr}.}

\noindent
\textit{(4) GreyBox} is content-display application that also extends the AJS
library. It can be used to display external pages, build image galleries,
receive file uploads and even show video or Flash content. The application
creates an \html{iframe} to load new content. Our transactional version of the
GreyBox application encoded a domain-specific iblock policy that only allowed
the creation of \html{iframe}s to whitelisted URLs.

\noindent
\textit{(5) Color Picker} builds upon the popular jQuery library~\cite{jquery}
and lets a user pick a color by moving sliders depicting the intensities of
red, blue and green. We executed the entire benchmark (including all the
supporting jQuery libraries) as a transaction and encoded an iblock that
disallowed modifications to the \code{innerHTML} property of arbitrary
\html{div} nodes.

However, for this guest, it turns out that an iblock that disallows any changes
to the sensitive \code{innerHTML} property of \textit{any} \html{div} element
is overly restrictive. This is because Color Picker modified the
\code{innerHTML} property of a \html{div} element that it created. We therefore
loosened our policy into a history-based policy that let the benchmark change
\code{innerHTML} properties of \html{div} elements that it created. The iblock
determines whether a \html{div} element was created by the transaction by
querying its write set. The relevant snippet from the iblock is shown below;
the \code{tx} variable denotes the transaction:
%
\begin{center}
%\vspace*{-1.0ex}
\setlength{\tabcolsep}{2pt}
\renewcommand{\arraystretch}{0.55}
\begin{tabular}{|l|}
\hline
\lno{1}~\codetiny{var ws = tx.getWriteSet(); ...}\\
\lno{2}~\codetiny{if (tx.getCause().match("innerHTML") 
                    \&\& ws.checkMembership(tx.getObject(), "*")}\\
          \mytab \codetiny{\&\& !(tx.getObject() instanceof
HTMLBodyElement))}\\
\lno{3}~\mytab \codetiny{// perform action on behalf of untrusted code}\\
\hline
\end{tabular}
\renewcommand{\arraystretch}{1}
%\vspace*{-2.0ex}
\end{center}


%------------------------------------------------------------------------------
\mysubsection{Fault injection and recovery} 
\label{section:evaluation:recovery}
%
To evaluate how \txjs\ can help hosts detect and debug malicious guest
activity, we performed a set of fault-injection experiments on a real Web
application that allows integration of untrusted guest code. We used the Bigace
Web content management system~\cite{bigace} running on our Web server as the
host, and created a Web site that mashed content from Bigace with content
provided by untrusted guests (each guest was included into the mashup using the
\html{script} tag).  We wrote guests that emulated known attacks and studied
host behavior when the host (1)~directly included the guest in its protection
domain; and (2)~used \txjs\ to isolate the guest.

Our experiments show that with appropriate iblock policies, speculative
execution ensured clean recovery; neither the \js\ heap nor the DOM of the host
was affected by the misbehaving guest.

\begin{mylist}
%
\item \textit{Misplaced event handler.} \js\ provides a \code{preventDefault}
method that suppresses the default action normally taken by the browser as a
result of the event. For example, the default action on clicking a link is to
fetch the
page corresponding to the URL referenced in the link. Several sites use
\code{preventDefault} to encode domain-specific actions instead, \eg~displaying
a popup when a link is clicked. 

In this experiment, we created a buggy guest that displays an advertisement
within a \html{div} element. This guest mistakenly registers an \code{onClick}
event handler that uses \code{preventDefault} with the \code{document} object
instead of with the \html{div} element. The result of including this guest
directly into the host's protection domain is that all hyperlinks on the Web
page are rendered unresponsive.  We then modified the host to isolate the guest
using a policy that disallows a transaction to commit if it attempts to
register an \code{onClick} handler with the \code{document} object. This
prevented the advertisement from being displayed, \ie~the \html{div} element
containing the misbehaving guest was not even created, but otherwise allowed
the host to function correctly. \js\ reference monitors proposed in prior work
can prevent the registration of the \code{onClick} handler, but leave
the \code{div} element of the misbehaving guest on the host's Web page.
%
\item \textit{Prototype hijacking.} We implemented a prototype hijacking attack
by writing a guest that set the \code{Array.prototype.slice} function to
\code{null}.  To illustrate the ill-effects of this attack, we modified the
host to include two popular (and benign) widgets, namely Twitter \cite{twitter}
and AddThis \cite{addthis}, in addition to the malicious guest. The prototype
hijacking attack prevented both the benign widgets from functioning properly.

However, when the malicious guest is enclosed within a transaction whose iblock
prevents a commit if it detects prototype hijacking attacks, the host and both
benign widgets worked normally. We further inspected the transaction's write
set and verified that none of the heap operations attributed to the malicious
guest were actually applied to the host. Although traditional \js\ reference
monitors can detect and prevent prototype hijacking attacks by blocking further
\html{script} execution, they do not allow the hosts to cleanly recover from
all heap changes.

\item \textit{Oversized advertisement.} We created a guest that displayed an
interactive \js\ advertisement within a \html{div} element. In an unprotected
host, this advertisement expands to occupy the full screen on a
\code{mouseover} event, \ie~the guest registered a misbehaving event-handler
that modifies the size of the \html{div}.  We modified the host to isolate this
guest using a transaction and an iblock that prevents a \code{commit} if the
size of the \html{div} element increased beyond a pre-specified limit. With
this policy, we observed that the host could successfully prevent the undesired
\html{div} modification by discarding the speculative DOM and \js\ heap changes
made by the event handler executing within the transaction.

% \item \textit{Logic error.} We wrote a buggy guest that emulated a
% malfunction on the New York Times web site, which displayed a blank screen
% for a couple of days in November 2008~\cite{opera_bug}. In that case, the
% culprit was a buggy guest that injected another \html{script} using
% \code{document.write} after the page had finished loading.  This caused the
% browser to replace the contents of the existing page (\ie~the New York Times
% page) with the injected \html{script}, which resulted in a blank screen.  We
% replicated this scenario by including the buggy guest in our host's
% protection domain, and verified that it displayed a blank screen. When
% confined using an iblock that skips the execution of \code{document.write} if
% the host page has been loaded completely.
%
\end{mylist}

%------------------------------------------------------------------------------

\mysubsection{Performance} 
\label{section:evaluation:performance}
%
We measured the overhead imposed by \txjs\ both using guest benchmarks, to
estimate the overall cost of using transactions, and with
microbenchmarks, to understand the impact on specific \js\ operations. 

\mysubsubsection{Guest benchmarks}
%
To evaluate the overall performance impact of \txjs, we measured the increase
in the load time of each guest benchmark.  Recall that each benchmark is
included in the Web page using a set of \html{script} tags; the version that
uses \txjs\ executes the corresponding \js\ code within a single transaction
using modified \html{script} tags.  The \code{onload} event fires at the end of
the document loading process, \ie~when all scripts have completed execution. We
therefore measured the time elapsed from the moment the page is loaded in the
browser to the firing of the \code{onload} event.

\begin{figure*}[t]
%\vspace*{-7.0ex}
\setlength{\tabcolsep}{2pt}
\centering
\includegraphics[keepaspectratio=true,width=0.6\textwidth]{figures/perf.png}
\remove{
\mycaptionsquish{Performance of guest benchmarks. \describecaption{This
chart
compares the time to load the unmodified version of each guest benchmark
against the time to load the transactional version in the two variants of
\txjs. On average, \txjs\ (JS only) increases the load time by just 0.11s,
while \txjs\ (full) increases the load time by 0.16s.}}
}
\add{
\mycaptionsquish{Performance of guest benchmarks. \describecaption{This
chart compares the time to load the unmodified version of each guest benchmark
against the time to load the transactional version in the two variants of
\txjs.}}
}
{\label{figure:onloadperf}}
%\vspace*{-5.0ex}
\end{figure*}

%\begin{wrapfigure}{rh}{0.6\textwidth}
%%\vspace*{-7.0ex}
%\setlength{\tabcolsep}{2pt}
%\centering
%\includegraphics[keepaspectratio=true,width=0.6\textwidth]{figures/perf.png}
%\remove{
%\mycaptionsquish{Performance of guest benchmarks. \describecaption{This
%chart
%compares the time to load the unmodified version of each guest benchmark
%against the time to load the transactional version in the two variants of
%\txjs. On average, \txjs\ (JS only) increases the load time by just 0.11s,
%while \txjs\ (full) increases the load time by 0.16s.}}
%}
%\add{
%\mycaptionsquish{Performance of guest benchmarks. \describecaption{This
%chart compares the time to load the unmodified version of each guest benchmark
%against the time to load the transactional version in the two variants of
%\txjs.}}
%}
%{\label{figure:onloadperf}}
%%\vspace*{-5.0ex}
%\end{wrapfigure}

To separately assess the impact of speculatively executing \js\ and DOM
operations, each experiment involved executing the benchmarks on two separate
variants of \txjs, namely \txjs\ (full) which supports both speculative DOM and
\js\ operations and \txjs\ (JS only) which only supports speculative \js\
operations (and therefore does not isolate DOM operations of the guest).
\figref{figure:onloadperf} presents the results averaged over 25 runs
of this experiment. On average, \txjs\ (JS only) increased load time by just
$0.11$ seconds while \txjs\ (full) increased the load time by $0.16$ seconds.
These overheads are typically imperceptible to end users. Only Color Picker
had above-average overheads. This was because (a)~the guest heavily
interacted with the DOM, causing frequent suspension of its transaction; and
(b)~the guest had several \code{Array} operations that referenced the
\code{length} of the array. Each such operation triggered a traversal of
read/write sets to calculate the array length.

Note that \txjs\ only degrades performance of \js\ code executing within
transactions (\ie~guests). The performance of code executing outside
transactions (\ie~hosts) is not affected by our prototype.

% md : not relevant any more

% However, \txjs\ increased the load time of the \js\ menu application by about
% $1.1$ seconds. We found that the bulk of the overhead was because of the HTML
% parser used in the iblock of the transaction used to confine the application.
% As shown in \figref{figure:jsmenu-evthdlr}, \js\ menu makes extensive use of
% \code{document.write} calls with incomplete HTML tags. Because the parser that
% we used in our prototype did not accept incomplete HTML tags, the iblock
% buffered arguments until it identified a complete tag before forwarding the
% \code{document.write} to the parser. This buffering, in conjunction with the
% fact that the \js-based HTML parser itself executes slower than the browser's
% natively-implemented HTML parser, contributed to the increased load time for
% the \js\ menu application.

\begin{wrapfigure}{rh}{0.41\textwidth}
%\vspace*{-5.0ex}
\setlength{\tabcolsep}{2pt}
\centering
\scriptsize
%\begin{tabular}{|l|c|c|r|}
\begin{tabular}{|l|r|}
\hline
\textbf{Microbenchmark} %& \textbf{Suspends?} & \textbf{Gluing?} 
& \textbf{Overhead} \\ 
\hline
\multicolumn{2}{|c|}{\textbf{Native Functions}}\\
\hline
\codetiny{eval("1")}                   %& \yes & \yes 
& $6.69\times$ \\ 
\codetiny{eval("if (true)true;false")} %& \yes & \yes 
& $6.87\times$ \\ 
\codetiny{fn.call(this, i)}            %& \no  & \yes 
& $1.89\times$ \\ 
\hline
\multicolumn{2}{|c|}{\textbf{External operations}}\\
\hline
\codetiny{getElementById("checkbox")}                %& \yes & \no 
& $6.78\times$ \\ 
\codetiny{getElementsByTagName("input")}             %& \yes & \no 
& $6.89\times$ \\ 
\codetiny{createElement("div")}                      %& \yes & \no 
& $3.69\times$ \\ 
\codetiny{createEvent("MouseEvents")}                %& \yes & \no 
& $3.82\times$ \\ 
\codetiny{addEventListener("click", clk, false)}     %& \yes & \no 
& $26.51\times$ \\ 
\codetiny{dispatchEvent(evt)}                        %& \yes & \no 
& $1.20\times$ \\
\codetiny{document.write("$<$b$>$Hi$<$/b$>$")}       %& \yes & \no 
& $1.26\times$
\\
\codetiny{document.write("$<$script$>$x=1;$<$/script$>$")}
                                             %& \yes & \yes 
& $2.01\times$ \\
%\footnote{We performed only 1000 runs for this experiment.} 
\hline
\end{tabular}
\mycaption{Performance of function call microbenchmarks.
%\describecaption{The table also shows whether the microbenchmark caused its
%transaction to suspend, and whether \txjs\ used gluing during its execution.}}
% \describecaption{Note: The call to \code{document.write} on line \lnobig{11}
% invokes gluing of the \html{script}'s execution context
% (\sectref{section:design:iblock:gluing}) when it suspends, while call to
% \code{document.write} on line \lnobig{10} only suspends.}
}
{\label{figure:microbenchmarkcost}}%\noindent\vspace{-0.5cm}
%\vspace*{-5.5ex}
\end{wrapfigure}

\mysubsubsection{Microbenchmarks} 
%
We further dissected the performance of \txjs\ using microbenchmarks designed
to stress specific functionalities.
\add{We used two sets of microbenchmarks:
function calls and event dispatchers.}
\remove{We used three sets of
microbenchmarks: function calls, event dispatchers, and the Sunspider suite.}
In
our experiments, we executed each microbenchmark within a transaction whose
iblock simply permitted all actions and resumed the transaction without
enforcing additional security policies, and compared its performance against the
non-transactional version.

\myparagraph{Function calls} 
%
\add{
We devised a set of microbenchmarks (\figref{figure:microbenchmarkcost}) that
}\remove{
These microbenchmarks
}stress the performance of \txjs's function call-handling code. Each benchmark
invoked the code in first column of \figref{figure:microbenchmarkcost}
$10,000$ times.

Recall that \txjs\ suspends on function calls that cause external operations
and for certain native function calls, such as \code{eval}. Each suspend
operation requires \txjs\ to save the state of the transaction, execute the
iblock, and restore the transaction state upon the execution of a \code{resume}
call. Most of the benchmarks in \figref{figure:microbenchmarkcost} trigger a
suspension, which induces significant overheads. In particular,
\code{addEventListener} had an overhead of $26.51\times$. The bulk of the
overhead was induced by code in the iblock that generates wrappers for the
event handler registered using \code{addEventListener}.
%
% The overhead includes the cost of i) setting the transaction context, ii)
% extracting the function body of the original event handler and wrapping it in
% a \code{Transaction} construct, iii) appending the iblock to the
% \code{Transaction} construct, and iv) attaching the new event handler to the
% DOM node.
%

% md: commenting out, since this is not relevant now.

% Note that \code{document.write} also incurs relatively high overheads of
% $12.3\times$ and $9.38\times$. This overhead can be attributed to the HTML
% parser, which processes the arguments to \code{document.write}. In the second
% case, \txjs\ also used gluing to patch the execution of the \code{script} tag
% to the current transaction.
\begin{comment}
\begin{figure*}
\vspace*{-3.5ex}
\setlength{\tabcolsep}{2pt}
%
\begin{minipage}[b]{0.48\textwidth}
\centering
\scriptsize %\addtolength{\tabcolsep}{-2pt}
%\begin{tabular}{|l|c|c|r|}
\begin{tabular}{|l|r|}
\hline
\textbf{Microbenchmark} %& \textbf{Suspends?} & \textbf{Gluing?} 
& \textbf{Overhead} \\ 
\hline
\multicolumn{2}{|c|}{\textbf{Native Functions}}\\
\hline
\lno{1}
\codetiny{eval("1")}                   %& \yes & \yes 
& $6.69\times$ \\ 
\lno{2}
\codetiny{eval("if (true)true;false")} %& \yes & \yes 
& $6.87\times$ \\ 
\lno{3}
\codetiny{fn.call(this, i)}            %& \no  & \yes 
& $1.89\times$ \\ 
\hline
\multicolumn{2}{|c|}{\textbf{External operations}}\\
\hline
\lno{4}
\codetiny{getElementById("checkbox")}                %& \yes & \no 
& $6.78\times$ \\ 
\lno{5}
\codetiny{getElementsByTagName("input")}             %& \yes & \no 
& $6.89\times$ \\ 
\lno{6}
\codetiny{createElement("div")}                      %& \yes & \no 
& $3.69\times$ \\ 
\lno{7}
\codetiny{createEvent("MouseEvents")}                %& \yes & \no 
& $3.82\times$ \\ 
\lno{8}
\codetiny{addEventListener("click", clk, false)}     %& \yes & \no 
& $26.51\times$ \\ 
\lno{9}
\codetiny{dispatchEvent(evt)}                        %& \yes & \no 
& $1.20\times$ \\
\lno{10}
\codetiny{document.write("$<$b$>$Hi$<$/b$>$")}       %& \yes & \no 
& $1.26\times$
\\
\lno{11}
\codetiny{document.write("$<$script$>$x=1;$<$/script$>$")}
                                             %& \yes & \yes 
& $2.01\times$ \\
%\footnote{We performed only 1000 runs for this experiment.} 
\hline
\end{tabular}
\mycaption{Performance of function call microbenchmarks.
%\describecaption{The table also shows whether the microbenchmark caused its
%transaction to suspend, and whether \txjs\ used gluing during its execution.}}
% \describecaption{Note: The call to \code{document.write} on line \lnobig{11}
% invokes gluing of the \html{script}'s execution context
% (\sectref{section:design:iblock:gluing}) when it suspends, while call to
% \code{document.write} on line \lnobig{10} only suspends.}
}
{\label{figure:microbenchmarkcost}}%\noindent\vspace{-0.5cm}
\indent\vspace{0.25cm}
\end{minipage}
%
\hfill
\begin{minipage}[b]{0.48\textwidth}
\centering
\scriptsize
\begin{tabular}{|l|c|c|}
\hline
                & \multicolumn{2}{c|}{\textbf{\underline{Overhead}}} \\ 
\textbf{Event name}  & \textbf{Normalized} & \textbf{Raw ($\mu$s)}\\
% & Normal (ms) & Transactional (ms) 
\hline
\lno{1} Drag Event     (\codetiny{drag})            & 1.71$\times$ & 97 \\ %& 136 & 233 
\lno{2} Keyboard Event (\codetiny{keypress})        & 1.16$\times$ & 150 \\
% \footnote{We performed only 100 runs for this experiment.} & 92 & 107 
\lno{3} Message Event  (\codetiny{message})         & 1.17$\times$ & 85 \\ % & 515 & 600 
\lno{4} Mouse Event    (\codetiny{click})           & 1.54$\times$ & 86 \\ % & 160 & 246 
\lno{5} Mouse Event    (\codetiny{mouseover})       & 2.05$\times$ & 88 \\ % & 84 & 172 
\lno{6} Mutation Event (\codetiny{DOMAttrModified}) & 2.14$\times$ & 88 \\ % & 77 & 165 
\lno{7} UI Event       (\codetiny{overflow})        & 1.97$\times$ & 61 \\ % & 63 & 124 
\hline
\end{tabular}
\mycaptionsquish{Performance of event dispatch microbenchmarks.}
{\label{figure:txevents}}
\indent\vspace{0.9cm}
\end{minipage}
\vspace*{-7.5ex}
\end{figure*}
\end{comment}

\begin{wrapfigure}{rh}{0.47\textwidth}
%\vspace*{-3.5ex}
\setlength{\tabcolsep}{2pt}
\centering
\scriptsize
\begin{tabular}{|l|c|c|}
\hline
                & \multicolumn{2}{c|}{\textbf{\underline{Overhead}}} \\ 
\textbf{Event name}  & \textbf{Normalized} & \textbf{Raw ($\mu$s)}\\
\hline
% & Normal (ms) & Transactional (ms) 
Drag Event     (\codetiny{drag})            & 1.71$\times$ & 97 \\ %&
%136 & 233 
Keyboard Event (\codetiny{keypress})        & 1.16$\times$ & 150 \\
% \footnote{We performed only 100 runs for this experiment.} & 92 & 107 
Message Event  (\codetiny{message})         & 1.17$\times$ & 85 \\ % &
%515 & 600 
Mouse Event    (\codetiny{click})           & 1.54$\times$ & 86 \\ % &
%160 & 246 
Mouse Event    (\codetiny{mouseover})       & 2.05$\times$ & 88 \\ % &
%84 & 172 
Mutation Event (\codetiny{DOMAttrModified}) & 2.14$\times$ & 88 \\ % &
%77 & 165 
UI Event       (\codetiny{overflow})        & 1.97$\times$ & 61 \\ % &
%63 & 124 
\hline
\end{tabular}
\mycaptionsquish{Performance of event dispatch microbenchmarks.}
{\label{figure:txevents}}
%\vspace*{-3.5ex}
\end{wrapfigure}

\myparagraph{User events} 
%
A \js\ application executing within a transaction may dispatch user events,
such as mouse clicks and key presses, which must be processed by the event
handler associated with the relevant DOM node. The promptness with which events
are dispatched typically affects end-user experience.

To measure the impact of transactions on this aspect of browser performance, we
devised a set of microbenchmarks that dispatched user events such as clicking a
checkbox, moving the mouse, pressing keys, etc. and measured the delay in
handling them (\figref{figure:txevents}). \remove{
 Each microbenchmark used \js\ code to simulate user actions, such as clicking
on a checkbox, hovering and moving the mouse, pressing keys, and events such as
\code{drag}, \code{message}, \code{DOMAttrModified} and \code{overflow}. 
}

In each case, code that generated and dispatched the event executed as a
transaction with an iblock that allowed all actions. To measure overhead, we
executed this code $1,\!000$ times and compared its performance against a native
event dispatcher. \figref{figure:txevents} presents the results, which show the
normalized overhead as well as the raw delay to process a single event. As this
figure shows, although the normalized overheads range from 16\% to 114\%, the
raw delays average about $94$ microseconds, which is imperceptible to end users.

% Read-set logging was turned off, since we believe that in a real setting, the
% cost of logging read accesses will get amortized across events. 
\remove{
\myparagraph{SunSpider} 
%
Finally, we also tested \txjs\ with the SunSpider \js\ benchmark suite by
executing each of its benchmarks within a transaction. This benchmark suite
reported an average overhead of $3.94\times$ across all benchmarks.  In
particular, we observed high overheads for benchmarks that had tight loops
operating over many array elements. The overhead primarily stems from having to
consult the write set for every read operation and updating the read set itself
even though the iblock's permissive security policy did not consult read/write
sets.
}
%------------------------------------------------------------------------------

\mysubsection{Complexity of policies} 
\label{section:evaluation:policy}

\begin{figure*}[t!]
%\begin{wrapfigure}{rh}{0.57\textwidth}
%\vspace*{-7.5ex}
\setlength{\tabcolsep}{2pt}
\centering
\scriptsize %\addtolength{\tabcolsep}{-1pt}
\begin{tabular}{|l|c|c||l|c|c|}
\hline 
\textbf{Policy}  & \textbf{T-LOC} & \textbf{C-LOC} &
  \textbf{Policy}  & \textbf{T-LOC} & \textbf{C-LOC}\\
\hline
  Conscript-\#1     & 7   & 2 
& Conscript-\#2     & 5   & 6\\
  Conscript-\#3     & 6   & 3 
& Conscript-\#4     & 9   & 7\\
  Conscript-\#5     & 9   & 9 
& Conscript-\#6     & 5   & 8\\
  Conscript-\#7     & 7   & 5 
& Conscript-\#8     & 5   & 6\\
  Conscript-\#10    & 9   & 16 
& Conscript-\#11    & 12  & 17\\
  Conscript-\#12    & 5   & 4 
& Conscript-\#13    & 4   & 6\\ 
  Conscript-\#14    & 3   & 5 
& Conscript-\#15    & 6   & 7\\
  Conscript-\#16    & 6   & 4 
& Conscript-\#17    & 7   & 5\\
\hline
\end{tabular}
\mycaptionsquish{Policy complexity. \describecaption{Comparing policies in
\txjs\ (T-LOC) and Conscript (C-LOC). Policies are numbered as in
Conscript~\cite{conscript:oak10}. We omitted Conscript-\#9 since it is
IE-specific.}}
{\label{figure:policies-LOC}}
%\vspace*{-2.5ex}
%\end{wrapfigure}
\end{figure*}

To study the complexity of writing policies in \txjs, we compared the number of
lines of code needed to write policies in \txjs\ and in
Conscript~\cite{conscript:oak10}.  We considered the policies discussed in 
Conscript and wrote equivalent policies in \txjs;
\figref{figure:policies-LOC} compares the source lines of code (counting number
of semi-colons) of policies in \txjs\ and Conscript. This shows that the
programming effort required to encode policies in both systems is comparable.


% \begin{figure}[t!]
% \centering
% \scriptsize
% \begin{tabular}{|l|c|c|}
% \hline 
% & \multicolumn{2}{c|}{\textbf{\underline{Lines of Code}}} \\ 
% \textbf{Policy Description}  & \textbf{\txjs} & \textbf{ConScript}\\
% \hline
% \lno{1} No dynamic script  & 7 & 2\\ 
% \lno{2} No string arguments to \code{setTimeout}, \code{setInterval}  & 5 & 6\\ 
% \lno{3} Restrict \code{XMLHttpRequest} to secure connections  & 5 & 8\\ 
% \lno{4} No foreign links after a cookie access  & 9 & 16\\ 
% \lno{5} Disable dynamic \html{iframe} creation  & 5 & 4\\ 
% \lno{6} Staged eval restrictions  & 7 & 5\\ 
% \hline
% \end{tabular}
% \mycaptionsquish{Comparing policy complexity in \txjs\ and Conscript.}
% {\label{figure:policies-LOC}}
% \end{figure}
