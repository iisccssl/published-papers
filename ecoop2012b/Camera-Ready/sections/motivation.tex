\mysection{Overview of \txjs}
\label{section:motivation}

\add{
\txjs\ enables hosts to understand the behavior of untrusted guests, detect
attacks by malicious guests and recover from them, and perform forensic
analysis. We briefly discuss \txjs's utility and then provide an overview of
its functionality for confining a malicious guest.
}

\add{
\begin{mylist}
%
\item \textbf{Understanding guest code.} Analysis of third-party \js\ code is
often hard due to code obfuscation. Using \txjs, a host can set watchpoints on
objects of interest. Coupled with suspend/resume, it is possible to perform a
fine grained debug analysis by inspecting the read/write sets on every guest
initiated object read/write and method invocation. \txjs's speculative execution
provides an ideal platform for concolic unit testing \cite{cute,dart} of guests.
For example, using \txjs, a host can test 
a guest's behavior under different values of domain cookies.
%
\item \textbf{Confining malicious guests.} \txjs's speculative execution permits
buffering of network I/O and writing to a speculative DOM, thereby allowing
unprecedented flexibility in confining untrusted guest code. For
example, to prevent clickjacking-style attacks, the host can simply discard
guest's modifications to the speculative DOM.
%
\item \textbf{Forensic analysis.} Since \txjs\ suspends on
external and user-defined operations, the suspend/resume mechanism is an
effective tool for forensic analysis of a suspected vulnerability exploited by
the guest. For example, code-injection attacks using DOM or host APIs
\cite{dom_xss} can be analyzed by observing the sequence of suspend calls and
their arguments.
%
\end{mylist}
}

\myparagraph{\textbf{\txjs\ in action}}

We illustrate \txjs's ability to confine untrusted guests by further elaborating
on the example introduced in \sectref{section:introduction}. Suppose that the
word processor hosts the untrusted widget using a \html{script} tag, as follows:
\html{script src="http://untrusted.com/guest.js"}.  In \figref{figure:motivate},
lines~\lnobig{6}--\lnobig{9} show a snippet from \code{guest.js}, which displays
advertisements relevant to keywords entered in the editor. Line~\lnobig{6}
registers a function to scan for keywords in the editor window by adding it to
the prototype of the \code{Editor} object. Lines~\lnobig{7} and \lnobig{8} show
the widget registering an event handler to display advertisements on certain
mouse events. While lines~\lnobig{6}--\lnobig{8} encode the core functionality
related to displaying advertisements, line~\lnobig{9} implements a
clickjacking-style attack by creating a transparent \html{div} element, placed
suitably on the editor with a link to an evil URL\@.

When hosting such a guest, the word processor can protect itself from attacks by
defining and enforcing a suitable set of security policies. These may include
policies to prevent prototype hijacks~\cite{ccc_06}, clickjacking-style attacks,
drive-by downloads, stealing cookies, snooping on keystrokes, \etc. Further, if
an attack is detected and prevented, it should not adversely affect normal
operation of the word processor.  We now illustrate how the word processor can
use \txjs\ to achieve such protection and cleanly recover from attempted
attacks.

\begin{figure*}[t]%{rh}{0.60\textwidth}
%\vspace*{-3.5ex}
\setlength{\tabcolsep}{2pt}
\centering
%\scriptsize
\begin{tabular}{cl}
%\hline
\lno{A} & \codetiny{do \{} \fbox{\textbf{\textrm{\tiny Function gotoIblock
implements the host's introspection block: Lines A--R}}}\\
\lno{B} & \mytab \codetiny{var arg = tx.getArgs();~~~var obj =
tx.getObject();}\\
\lno{C} & \mytab \codetiny{var rs = tx.getReadSet();~var ws =
tx.getWriteSet();}\\
\lno{D} & \mytab \codetiny{for(var i in builtins) \{}\\
\lno{E} & \mytab \mytab \codetiny{if (ws.checkMembership(Editor.prototype,
builtins[i])) tocommit = false;}\\
\lno{F} & \mytab \codetiny{\} ... /* definition of `IsClickJacked' to go here
*/}\\
\lno{G} & \mytab \codetiny{if (IsClickJacked(tx.getTxDocument())) tocommit =
false;}\\
\lno{H} & \mytab \codetiny{... /* more policy checks go here */}
\fbox{\textbf{\textrm{\tiny inlined code from libTranscript: Lines I--O}}}\\
  \rowcolor[gray]{0.9}
\lno{I} & \mytab \codetiny{switch(tx.getCause()) \{}\\
  \rowcolor[gray]{0.9}
\lno{J} & \mytab \mytab \codetiny{case "addEventListener":}\\
  \rowcolor[gray]{0.9}
\lno{K} & \mytab \mytab \mytab \codetiny{var txHandler =
MakeTxHandler(arg[1]);}\\
  \rowcolor[gray]{0.9}
\lno{L} & \mytab \mytab \mytab \codetiny{obj.addEventListener(arg[0],
txHandler, arg[2]); break;}\\
  \rowcolor[gray]{0.9}
\lno{M} & \mytab \mytab \codetiny{case "write": WriteToTxDOM(obj, arg[0]);
break; ... /* more cases */}\\
  \rowcolor[gray]{0.9}
\lno{N} & \mytab \mytab \codetiny{default: break;}\\
  \rowcolor[gray]{0.9}
\lno{O} & \mytab \codetiny{\};}\\
\lno{P} & \mytab \codetiny{tx = tx.resume();}\\
\lno{Q} & \codetiny{\} while(tx.isSuspended());}\\
\lno{R} & \codetiny{return tocommit;}\\
%\hline
\end{tabular}
\remove{
\mycaptionsquish{An iblock. \describecaption{The above code shows the body
of the function \code{gotoIblock} and implements the host's iblock for the guest
in \figref{figure:motivate}. An iblock consists of two parts: a host-specific
part, which encodes the host's policies to confine the guest (lines 4--8), and a
mandatory part, which contains functionality that is generic to all hosts (lines
9--15). The mandatory part of iblock is implemented as a JavaScript library
(\code{libTranscript}) and can be included within the iblock of the host.}}
}
\add{
\mycaptionsquish{An iblock. \describecaption{An iblock consists of two parts: a
host-specific part, which encodes the host's policies to confine the guest
(lines D--H), and a mandatory part, which contains functionality that is generic
to all hosts (lines I--O).}}
}
{\label{figure:iblock}}
%\vspace*{-3.0ex}
\end{figure*}

%-------------------------------------------------------------------------------
The host protects itself by embedding the guest within a \code{transaction}
construct (line~\lnobig{5}, \figref{figure:motivate}) and specifies its security
policy (lines~\lnobig{D}--\lnobig{O}, \figref{figure:iblock}). When the
transaction executes, \txjs\ records all reads and writes to \js\ objects in
per-transaction read/write sets. Any attempts by the guest to modify the host's
\js\ objects (\eg~on line~\lnobig{6}, \figref{figure:motivate}) are speculative;
\ie~these changes are visible only to the guest itself and do not modify the
host's view of the \js\ heap. To ensure that DOM modifications by the guest are
also speculative, \txjs's DOM subsystem clones the host's DOM at the start of
the transaction and resolves all references to DOM objects in a transaction to
the cloned DOM\@. Thus, references to \code{document} within the guest resolve
to the cloned DOM\@.

When the guest performs DOM operations, such as those on
lines~\lnobig{7}--\lnobig{9}, and other external operations, such as
\code{XMLHttpRequest}, \txjs\ \textit{suspends} the transaction and passes
control to the host.  This situation is akin to a system call in a user-space
program causing a trap into the operating system.  Suspension allows the host to
mediate external operations as soon as the guest attempts them.  When a
transaction suspends or completes execution, \txjs\ creates a
\textit{transaction object} in \js\ to denote the completed or suspended
transaction. In \figref{figure:motivate}, the variable \code{tx} refers to the
transaction object. \txjs\ then passes control to the host at the program point
that syntactically follows the transaction. There, the host implements an
\textit{introspection block} (or \textit{iblock}) to enforce its security policy
and perform operations on behalf of a suspended transaction.

\myparagraph{Transaction objects} 
\label{section:motivation:txobj}
%
A transaction object records the state of a suspended or completed transaction.
It stores the read and write sets of the transaction and the list of activation
records on the call stack of the transaction when it was suspended.  It
provides builtin methods, such as \code{getReadSet} and \code{getWriteSet}
shown in \figref{figure:iblock}, that the host can invoke to access read and
write sets, observe the actions of the guest, and make policy decisions. 

When a guest tries to perform an external operation and thus suspends, the
resulting transaction object contains arguments passed to the operation. For
example, a transaction that suspends due to an attempt to modify the DOM\@, such
as the call \code{document.write} on line~\lnobig{9}, will contain the DOM
object referenced in the operation (\code{document}), the name of the method
that caused the suspension (\code{write}), and the arguments passed to the
method.  (Recall that \txjs's DOM subsystem ensures that \code{document}
referenced within the transaction will point to the cloned DOM\@.)  The host can
access these arguments using builtin methods of the transaction object, such as
\code{getArgs}, \code{getObject} and \code{getCause}. 
\remove{These arguments are
analogous to system-call arguments passed from a user-space program to the
operating system.}
Depending on its policy, the host can either perform
the operation on behalf of the guest, simulate the effect of performing it,
defer the operation for later, or not perform it at all.

The host can resume a suspended transaction using the transaction object's
builtin \code{resume} method. \txjs\ then uses the activation records stored in
the transaction object to restore the call stack, and resumes control at the
program point following the instruction that caused the transaction to suspend
(akin to resumption of program execution following a system call).
Transactions can suspend an arbitrary number of times until they complete
execution. The builtin \code{isSuspended} method determines whether the
transaction is suspended or has completed. 

A completed transaction can be committed using the builtin \code{commit}
method.  This method copies the contents of the write set to the corresponding
objects on the host's heap, thereby publishing the changes made by the guest.
It also synchronizes the host's DOM with the cloned version that contains any
DOM modifications made by the guest. A completed transaction's call stack is
empty, so attempts to resume a completed transaction will have no effect. Note
that \txjs\ does not define an explicit \code{abort} operation.  This is
because the host can simply discard changes made by a transaction by choosing
not to commit them. If the transaction object is not referenced anymore, it
will be garbage-collected.

\myparagraph{Introspection blocks} 
\label{section:motivation:iblock}
%
When a transaction suspends or completes, \txjs\ passes control to the
instruction that syntactically follows the transaction in the code of the host.
At this point, the host can check the guest's actions by encoding its security
policies in an \textit{iblock}. The iblock in \figref{figure:iblock}
%spans lines~\lnobig{11}--\lnobig{27} and 
has two logical parts: a
\textit{host-specific part} that encodes host's policies
(lines~\lnobig{D}--\lnobig{H}), and a \textit{mandatory part} that performs
operations on behalf of suspended guests (lines~\lnobig{I}--\lnobig{O}). The
iblock in \figref{figure:iblock} illustrates two policies:
%
\begin{mylist}
%
\item Lines~\lnobig{D}--\lnobig{E} detect prototype hijacking attempts on the
\code{Editor} object. To do so, they check the transaction's write set for
attempted redefinitions of builtin methods and fields of the \code{Editor}
object.
%
\item Line \lnobig{G} detects clickjacking-style attempts by checking the DOM
for the presence of any transparent HTML elements introduced by the guest. (The
body of \code{IsClickJacked}, which implements the check, is omitted for
brevity).
%
\end{mylist}

\begin{comment}
%==============================================================================
% Workflow figure moved here for correct placement!
%==============================================================================
\begin{figure*}[t!]
\vspace*{-3.5ex}
\centering
%\includegraphics[keepaspectratio=true,width=0.8\textwidth]{
figures/transcript.png}
\begin{tabular}{cc}
%\hline
\begin{minipage}{0.37\textwidth}
\centering
\includegraphics[keepaspectratio=true,width=0.85\textwidth]{
figures/workflow-1.pdf}\\
\textbf{\scriptsize (a)~Locations of traps/returns to/from \txjs.}
\end{minipage} &
\begin{minipage}{0.6\textwidth}
\centering
\includegraphics[keepaspectratio=true,width=0.91\textwidth]{
figures/workflow-2.pdf}\\
\textbf{\scriptsize(b)~Corresponding actions within the \txjs\ runtime system
for a trap/return.}
\end{minipage}\\
\hline
\end{tabular}
\mycaptionsquish{Workflow of a \txjs-enhanced host.
\describecaption{Part~(a) of the figure shows a host enclosing a guest within a
transaction and an inlined introspection block, while part~(b) shows the \js\
runtime and the DOM subsystem. The labels \ding{172}-\ding{177} in the figure
show: 
%
\ding{172}~the host's DOM being cloned at the start of the transaction,
%
\ding{173}~the host's call stack before a call that suspends the transaction,
%
\ding{174}~the call stack after suspension,
%
\ding{175}~the host's call stack when the transaction is about to resume; the
speculative DOM has been updated with the requested changes,
%
\ding{176}~the host's call stack just after resumption,
%
\ding{177}~shows the transaction committing, which copies all speculative
changes to the host's DOM and \js\ heap.
%
The thick lines on the call stacks denote transaction delimiters.  Arrows show
control transfer from the transaction to the iblock and back.}}
{\label{figure:workflow}} 
\vspace*{-3.0ex}
\end{figure*}
%==============================================================================
\end{comment}

The body of the \code{switch} statement encodes the mandatory part of the
iblock and implements two key functionalities, which are further explained in
\sectref{section:design:iblock}:
%
\begin{mylist}
%
\item Lines~\lnobig{J}--\lnobig{L} in \figref{figure:iblock} create and attach
an event handler to the cloned DOM when the guest suspends on line~\lnobig{8} in
\figref{figure:motivate}. The \code{MakeTxHandler} function creates a new
\textit{wrapped} handler, by enclosing the guest's event handler
(\code{displayAds}) within a \code{transaction} construct. Doing so ensures that
the execution of any event handlers registered by the guest is also speculative,
and mediated by the host's security policies. The iblock then attaches the event
handler to the corresponding element (\code{elem}) in the cloned DOM.
%
\item Line~\lnobig{M} in \figref{figure:iblock} speculatively executes the
DOM modifications requested when the guest suspends on line~\lnobig{9} in
\figref{figure:motivate}. The \code{WriteToTxDOM} function invokes the
\code{write} call on \code{obj}, which points to the \code{document} object in
the cloned DOM.
%
\end{mylist}

If a transaction does not commit because of a policy violation, the host's DOM
and \js\ objects will remain unaffected by the guest's modifications. For
instance, when the host
% in \figref{figure:motivate} 
aborts the guest after it
detects the clickjacking attempt, the host's DOM will not contain any remnants
of the guest's actions (such as event handlers registered by the guest).  The
host's \js\ objects, such as \code{Editor}, are also unaffected. Speculatively
executing guests therefore allows hosts to cleanly recover from attack
attempts.

Iblocks offer hosts the option to postpone external operations. For example, a
host may wish to defer all network requests from an untrusted advertisement
until the end of the transaction. It can do so using an iblock that buffers
these requests when they suspend, and thereafter resume the transaction; the
buffered requests can be processed after the transaction has completed. Such
postponement will not affect the guest if the buffered requests are
asynchronous, \eg~\code{XMLHttpRequest}.

Because a transaction may suspend several times, the iblock is structured as a
loop, whose body executes each time the transaction suspends and once when the
transaction completes. This way, the same policy checks apply whether the
transaction suspended or completed.