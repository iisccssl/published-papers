\mysection{Design of Transcript}
\label{section:design}

%==============================================================================
% Workflow figure moved here for correct placement!
%==============================================================================
\begin{figure*}[t!]
%\vspace*{-3.5ex}
\centering
%\includegraphics[keepaspectratio=true,width=0.8\textwidth]{
%figures/transcript.png}
\begin{tabular}{cc}
%\hline
\begin{minipage}{0.37\textwidth}
\centering
\includegraphics[keepaspectratio=true,width=0.85\textwidth]
{figures/workflow-1.pdf}\\
\textbf{\scriptsize (a)~Locations of traps/returns to/from \txjs.}
\end{minipage} &
\begin{minipage}{0.6\textwidth}
\centering
\includegraphics[keepaspectratio=true,width=0.91\textwidth]
{figures/workflow-2.pdf}\\
\textbf{\scriptsize(b)~Corresponding actions within the \txjs\ runtime system
for a trap/return.}
\end{minipage}\\
\hline
\end{tabular}
\mycaptionsquish{Workflow of a \txjs-enhanced host.
\describecaption{Part~(a) of the figure shows a host enclosing a guest within a
transaction and an inlined introspection block, while part~(b) shows the \js\
runtime and the DOM subsystem. The labels \ding{172}-\ding{177} in the figure
show: 
%
\ding{172}~the host's DOM being cloned at the start of the transaction,
%
\ding{173}~the host's call stack before a call that suspends the transaction,
%
\ding{174}~the call stack after suspension,
%
\ding{175}~the host's call stack when the transaction is about to resume; the
speculative DOM has been updated with the requested changes,
%
\ding{176}~the host's call stack just after resumption,
%
\ding{177}~shows the transaction committing, which copies all speculative
changes to the host's DOM and \js\ heap.
%
The thick lines on the call stacks denote transaction delimiters.  Arrows show
control transfer from the transaction to the iblock and back.}}
{\label{figure:workflow}} 
%\vspace*{-3.0ex}
\end{figure*}
%==============================================================================

We now describe the design of \txjs's mechanisms using
\figref{figure:workflow}, which summarizes the workflow of a \txjs-enhanced
host. The figure shows the operation of the \txjs\ runtime system at key points
during the execution of the host, which has included an untrusted guest akin to the one in
\figref{figure:motivate} using a transaction.

When a transaction begins execution, \txjs\ first provides the transaction with
its private copy of the host's DOM tree. It does so by cloning the current
state of the host's DOM, including any event handlers associated with the nodes
of the DOM (\ding{172} in \figref{figure:workflow}). When a guest references
nodes in the host's DOM, \txjs\ redirects these references to the corresponding
nodes in the transaction's private copy of the DOM. 

Next, the \txjs\ runtime pushes a \textit{transaction delimiter} on the \js\
call stack. \txjs\ places the activation records of methods invoked within the
transaction above this delimiter. It also records the locations of \js\ objects
accessed/modified within the transaction in read/write sets.  If the
transaction executes an external operation, the runtime suspends the
transaction.  To do so, it creates a transaction object and (a)~initializes the
object with the transaction's read/write sets; (b)~pops all the activation
records on the \js\ call stack until the topmost transaction delimiter;
(c)~stores these activation records in the transaction object; (d)~saves the
program counter; and (e)~sets the program counter to immediately after the end
of the transaction, \ie~the start of the iblock (steps \ding{173} and \ding{174}
in \figref{figure:workflow}).

The iblock logically extends from the end of the transaction to the last
\code{resume} or \code{commit} call on the transaction object (\eg~lines
\lnobig{A}--\lnobig{R} in \figref{figure:iblock}). The iblock can access
the transaction object and its read/write sets to make policy decisions.  If
the iblock invokes \code{resume} on a suspended transaction, the \txjs\ runtime
(a)~pushes a transaction delimiter on the current \js\ call stack; (b)~pushes
the activation records saved in the transaction object; and (c)~restores the
program counter to its saved value. Execution therefore resumes from the
statement following the external operation (see \ding{175} and \ding{176}). If
the iblock invokes \code{commit} instead, the \txjs\ runtime updates the \js\
heap using the values in the transaction object's write set.  The \code{commit}
operation also replaces the host's DOM with the cloned DOM (step~\ding{177}).

The \txjs\ runtime behaves in the same way even when transactions are nested:
\txjs\ pushes a new delimiter on the \js\ call stack for each level of nesting
encountered at runtime. Each suspend operation only pops activation records
until the topmost delimiter on the stack.  Nesting is important when a guest
itself wishes to confine code that it does not trust. This situation arises
when a host includes a guest from a first-tier advertising agency
(\code{1sttier.com}), which itself includes code from a second-tier agency
(\code{2ndtier.com}). Whether the host confines the advertisement using an
\textit{outer} transaction, \code{1sttier.com} may itself confine code from
\code{2ndtier.com} using an \textit{inner} transaction using its own security
policies.  If code from \code{2ndtier.com} attempts to modify the DOM\@, that
call suspends and traps to the iblock defined by \code{1sttier.com}.  If this
iblock attempts to modify the DOM on behalf of \code{2ndtier.com}, the outer
transaction suspends in turn and passes control to the host's iblock. In
effect, the DOM modification succeeds only if it is permitted at \textit{each}
level of nesting.

%------------------------------------------------------------------------------
\mysubsection{Components of an iblock} 
\label{section:design:iblock}

%
As discussed in \sectref{section:motivation}, an iblock consists of two parts:
a \textit{host-specific} part, which codifies the host's policies to mediate
guests, and a \textit{mandatory} part, which contains functionality that is
generic to all hosts. In our implementation, we have encoded the second part as a
\js\ library (\code{libTranscript}) that can simply be included into the iblock
of a host. %\footnote{The body of the \code{switch} statement in
%\figref{figure:iblock} shows a snippet from this \js\ library.
%\figref{figure:txwebapp} illustrates an iblock that simply includes and invokes
%functions from the library.}
This mandatory part implements two functionalities: gluing execution contexts
and generating wrappers for event handlers.

\mysubsubsection{Gluing execution contexts} 
\label{section:design:iblock:gluing}
%
Guests often use \code{document.write} or similar calls to modify the host's
DOM, as shown on line~\lnobig{9} of \figref{figure:motivate}.  When such guests
execute within a transaction, the \code{document.write} call traps to the
iblock, which must complete the call on behalf of the guest and render the
HTML in the cloned DOM\@. However, the HTML code in \code{document.write} may
contain scripts, \eg~\code{document.write('\html{script src = code.js}')}. The
execution of \code{code.js}, having been triggered by the guest, must then be
mediated by the same security policy that governs the guest. 

Thus, \code{code.js} should be executed in the same context as the
transaction where the guest executes.  To achieve this goal, the
mandatory part of the iblock encapsulates the content of \code{code.js} into a
function and uses a builtin \code{glueresume}
method of the transaction object to
instruct the \txjs\ runtime to invoke this function when it resumes the
suspended transaction. The net effect is similar to fetching
and inlining the content of \code{code.js} into the transaction. We call this
operation \textit{gluing}, because it glues the code in
\code{code.js} to that of the guest.

To implement gluing, the iblock must recognize that the \code{document.write}
includes additional scripts. This in turn requires the iblock to
parse the HTML argument to \code{document.write}. We therefore exposed  the
browser's HTML parser through a new \code{document.parse} API to allow
HTML (and CSS) parsing in iblocks. This API accepts a
HTML string argument, such as the argument to
\code{document.write}, and parses it to recognize \html{script} elements and
other HTML content. It also recognizes inline event-handler registrations, so
that they can be wrapped as described in
\sectref{section:design:iblock:wrapper}.  When the iblock invokes
\code{document.parse} (in \figref{figure:iblock}, it is invoked within the
call to \code{WriteToTxDOM} on line~\lnobig{M}), the parser creates new
functions that contain code in \html{script} elements.  It returns these
functions to the host's iblock, which can then invoke them
by gluing. The parser also renders other (non-script) HTML
content in the cloned DOM\@. 

Guest operations involving \code{innerHTML} are handled simlarly. \txjs\
suspends a guest that attempts an \code{innerHTML} operation, parses the
new HTML code for any scripts, and glues their execution
into the guest's context.

\mysubsubsection{Generating wrappers for event handlers}
\label{section:design:iblock:wrapper}
%
Guests executing within a transaction may attempt to register functions to
handle asynchronous events.  For example, line~\lnobig{8} in
\figref{figure:motivate} registers \code{displayAds} as an \code{onMouseOver}
handler. Because \code{displayAds} is guest code, it is important to associate
it with the iblock for the transaction that registered it and to subject it to
the same policy checks.  \txjs\ does so by creating a new function
\code{tx\_displayAds} that \textit{wraps} \code{displayAds} within a
transaction guarded by the same iblock, and registering \code{tx\_displayAds}
as the event handler for the \code{onMouseOver} event.

To this end, the mandatory part of the iblock includes creating wrappers
(such as \code{tx\_displayAds}) for event handlers. When the guest executes a
statement such as \code{elem.addEventListener(...)}, it would trap to the
iblock, which can then examine the arguments to this call and create a wrapper
for the event handler.  Guests can alternatively use \code{document.write}
calls to register event handlers \eg~\code{document.write ('\html{div
onMouseOver="displayAds();"}')}.  In this case, the iblock recognizes that an
event handler is being registered by parsing the HTML argument of the
\code{document.write} call (using the \code{document.parse} API) when it
suspends, and wraps the call. Our wrapper generator handles all the event
 models  supported by Firefox~\cite{dom-level}.
%  \remove{
%We refer readers interested in the details to
%Appendix~\ref{section:appendix:events}.
%}

Besides event handlers, \js\ supports other constructs for asynchronous
execution: AJAX callbacks, which execute upon receiving network events
(\code{XMLHttpRequest}), and features such as \code{setTimeOut} and
\code{setInterval} that trigger code execution based upon timer events. The
mandatory part of the iblock also handles these constructs by wrapping
callbacks as just described.

%------------------------------------------------------------------------------
\mysubsection{Hiding sensitive variables} 
\label{section:design:hiding}

The iblock of a transaction checks the guest's actions against the host's
policies. These policies are themselves encoded in \js, and may use methods and
variables (\eg~\code{tx}, \code{tocommit} and \code{builtins} in
\figref{figure:motivate}) that must be protected from the guest. Without
precautions, the guest can use \js's extensive reflection capabilities to
tamper with these sensitive variables.  \figref{figure:selfleak} presents an
example of one such attack, a reference leak, where the malicious guest obtains
a reference to the \code{tx} object by enumerating the properties of the
\code{this} object, and redefines the method \code{tx.getWriteSet}
speculatively.  As presented, the example in \figref{figure:motivate} is
vulnerable to such a reference leak.

\begin{wrapfigure}{rh}{0.42\textwidth}
%\begin{center}
%\vspace*{-3.0ex}
\setlength{\tabcolsep}{2pt}
\centering
\renewcommand{\arraystretch}{0.75}
%%\begin{tabular}{|rl|}
%%\hline
%%\lno{1}  & \codetiny{var tx = transaction \{} \codetiny{... //code that
%%suspends ...}\\
%%\lno{2}  & \mytab \codetiny{for (var x in this) \{}\\
%%\lno{3}  & \mytab \mytab \codetiny{if (this[x] instanceof Tx\_obj) txref =
%%this[x];}\\
%%\lno{4}  & \mytab \codetiny{\}; txref.getWriteSet = function() \{~\};}\\
%%\lno{5}  & \codetiny{\}}\\
%%\hline
%%\end{tabular}
%\end{center}
\begin{tabular}{|l|}
\hline
\codetiny{var tx = transaction \{} \codetiny{... //code that suspends ...}\\
\mytab \codetiny{for (var x in this) \{}\\
\mytab \mytab \codetiny{if (this[x] instanceof Tx\_obj) txref = this[x];}\\
\mytab \codetiny{\}; txref.getWriteSet = function() \{~\};}\\
\codetiny{\}}\\
\hline
\end{tabular}
\mycaptionsquish{A guest that implements a reference leak.
\describecaption{The \code{tx} object is created and attached to \code{this}
when guest suspends.}}
{\label{figure:selfleak}}
%\vspace*{-3.0ex}
\end{wrapfigure}

To protect such sensitive variables, we adopt a defense called \textit{variable
hiding} that eliminates the possibility of leaks by construction. This
technique mandates that guests be placed outside the scope of the iblock's
variables, such as \code{tx}. The basic idea is to place the guest and the
iblock in separate, lexically scoped functions, so that variables such as
\code{tx}, \code{tocommit} and \code{builtins} are not accessible to the guest.
By so hiding sensitive variables from the guest, this defense prevents reference
leaks. \figref{figure:txwebapp} illustrates this defense %with a code snippet,
after introducing some more details of our implementation.

%------------------------------------------------------------------------------
\mysection{Security Assurances} 
\label{section:analysis}
%
\txjs's ability to protect hosts from untrusted guests depends on two factors:
(a)~the assurance that a guest cannot subvert \txjs's mechanisms, \ie~the
robustness of the trusted computing base; and (b)~host-specific policies used to
mediate guests. 

\mysubsection{Trusted computing base}
\label{section:analysis:tcb}

\txjs's trusted computing base (TCB) consists of the runtime component
implemented in the browser and the mandatory part of the host's iblock. The TCB
provides the following security properties: (a)~\textit{complete mediation},
\ie~control over all \js\ and external operations performed by a guest; and
(b)~\textit{isolation}, \ie~the ability to confine the effects of the guest. 

\begin{mylist}
%
\item \textbf{Complete mediation.} The \txjs\ runtime and the mandatory part of
the host's iblock together ensure complete mediation of guest execution. The
runtime: (a)~records all guest accesses to the host's \js\ heap in the
corresponding transaction's read/write sets; (b)~causes a trap to the host's
iblock when the guest attempts an external operation; and (c)~redirects all
guest references to the host's DOM to the cloned DOM. The mandatory part of the
iblock, consisting of wrapper generators and the HTML parser, ensures that any
additional code fetched by the guest or scheduled for later execution
(\eg~event handlers or callbacks for \code{XMLHttpRequest}) will itself be
enclosed within transactions mediated by the same iblock. This process recurs
so that the host's policies mediate all guest code, even event handlers
installed by callbacks of event handlers.

\item \textbf{Isolation.} \txjs\ isolates guest operations using speculative
execution. It records changes to the host's \js\ heap within the guest
transaction's write set, and changes to the host's DOM within the cloned DOM.
The host then has the opportunity to review these speculative changes within
its iblock and ensure that they conform to its security policies.  Observe that
a suspended/completed transaction may provide the host with references to
objects modified by the guest, \eg~in \figref{figure:motivate}, a reference to
\code{elem} is passed to the iblock via the \code{getObject} API. Speculative
execution ensures that if the transaction has not yet been committed, then
accesses to the object's methods and fields via this reference will still
resolve to their values at the beginning of the transaction.  Thus, for
instance, a call to the \code{toString} method of the \code{elem} object in the
iblock of \figref{figure:motivate} would still work as intended if even if the
guest had redefined this method within the transaction.  Note that variables
hidden from the guest cannot even be \textit{speculatively} modified, thereby
automatically isolating them from the guest.
%
\end{mylist}

Together, the above properties ensure the following invariant: At the point
when a transaction suspends or completes execution and is awaiting inspection
by the host's iblock, none of the host's \js\ objects or its DOM would have
been modified by the guest. Further, host variables hidden from the guest will
not be modified even after the transaction has committed.  Overall, executing a
transaction never incurs any side effect, and any side effect that would be
incurred by committing a transaction can be first vetted by inspecting the
transaction.

\mysubsection{Whitelisting for host policies}
\label{section:analysis:bestpractice}

Hosts can import the speculative changes made by a guest after inspecting them
against their security policies. Even though complete mediation and isolated
execution ensure that the core \textit{mechanisms} of \txjs\ cannot be
subverted by guest execution (\ie~they ensure that all of the guest's
speculative actions will be available for inspection by the host), the ability
of the host to isolate itself from the guest ultimately depends on its
\textit{policies}. 

Host policies are necessarily domain-specific and have to be written manually
in our current prototype. Though our experiments
(\sectref{section:evaluation:policy}) suggest that the effort required to write
policies in \txjs\ is comparable to that required in other systems, writing
policies is admittedly a difficult exercise and further research is needed to
develop tools for policy authors to debug/verify the completeness of their
policies. \newadd{However, iblock policies once written can be reused across
applications if applications share similar protection criteria. As a deployment
model, we envision a vendor or community-driven curated database of
commonly-used iblock policies, which hosts can use to secure untrusted guests.}

We \add{suggest that iblock authors should employ a whitelist which
specifies the host objects that can legitimately be modified by the guest and
reject attempts to modify objects outside the whitelist. This guideline may
cause false positives if the whitelist is not comprehensive. For example, both
\code{window.location} and \code{window.location.href} can be used to change the
location field of the host, but a whitelist that includes only one will reject
guests that modify guest location using the other. Nevertheless, whitelisting
allows hosts to be conservative when allowing guests to modify their objects.}

\remove{outline a set of ``best practices'' below to avoid
common pitfalls.} 

\remove{
\begin{mylist}
%
\item \textbf{Use whitelisting.} Iblocks should employ a whitelist that
specifies host objects that can legitimately be modified by the guest and
reject attempts to modify objects outside the whitelist. This guideline may
cause false positives if the whitelist is not comprehensive. For example, both
\code{window.location} and \code{window.location.href} can be used to change
the location field of the host, but a whitelist that includes only one will
reject guests that modify guest location using the other. Nevertheless,
whitelisting allows hosts to be conservative when allowing guests to modify
their objects.
%
The following two guidelines are subsumed if whitelisting is employed, but
should be followed if it is not convenient to write a whitelist-based policy.
%
\item \textbf{Validate global objects and caller arguments.} A guest executing
in a transaction can speculatively modify any objects visible in its scope.
These may include host-defined global variables, as well as arguments passed to
functions currently on the call stack (\eg~via \code{arguments.callee.caller}).
The host must therefore check whether any of its objects that are visible to the
guest appear in the write set of the transaction, and vet these changes against
its security policy.

Our prototype implementation provides a library of utilities that policy
authors can use to detect modifications to objects that are in-scope. This
library implements \code{modifGlobals}, \code{modifCallers} and
\code{modifNatives}, which an iblock author can use to obtain a list of
host-defined globals, caller arguments and native objects (\eg~\code{String},
\code{Object}, \code{Array}) that the guest has speculatively modified. These
utilities can ease the burden of writing iblocks.  As an example, the
following snippet included in an iblock would reject guests that inspect and
modify arguments on the call stack: 
%
\code{if (modifCallers(tx). length != 0) tocommit = false;}, 
%
where \code{tx} is the transaction object.
%
\item \textbf{Defend against prototype hijacks and type forgery.} A malicious
guest can redefine builtin object properties to implement prototype hijacking
and type forgery attacks. For example, a guest can redefine the \code{toString}
and \code{valueOf} methods of an object to return entities of arbitrary type.
If the host commits the transaction despite this redefinition, it will be
vulnerable to type forgery when it subsequently invokes these methods.

The host's iblocks must therefore include code to check the write set for
redefinitions of builtin methods such as \code{toString} and \code{valueOf} or
attempts to modify an object via its \code{prototype} field. We include this
guideline as a best practice rather than including it as a defense in the
mandatory part of the iblock because there may be domain-specific scenarios
where violations are acceptable, \eg~in \figref{figure:motivate}, where the
host allows limited modifications to the \code{Editor} object via its
\code{prototype} field.
%
\end{mylist} 
}