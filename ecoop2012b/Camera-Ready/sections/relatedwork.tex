\mysection{Related Work}
\label{section:relatedwork}

This paper builds upon the idea of extending \js\ with transactions, which was
proposed in a recent position paper~\cite{plas10}.  While that paper focused on
the semantics of the extended language, this paper is the first to report the
design and implementation of a complete speculative execution system for \js.

There is much prior work in the broad area of isolating untrusted guests.
\txjs\ is unique because it allows hosts to recover cleanly and easily from the
effects of malicious or buggy guests (\figref{figure:relatedwork}). In exchange
for requiring no modification to the guest, \txjs\ requires modifications both
to the host (\ie~the server side) and to the browser (\ie~the client side) to
enhance the \js\ language.

\myparagraph{Static analysis}
%
Despite the dynamic nature of \js, there have been a few efforts at statically
analyzing \js\ code. Gatekeeper~\cite{gatekeeper:sec09} presents a static
analysis to validate widgets written in a subset of \js. It does so by matching
widget source code against a database of patterns denoting unsafe programming
practices.
  Guha \etal~\cite{guha} developed static techniques to improve AJAX
security. Their work uses static analysis to enhance a server-side proxy with
models of AJAX computation on the client. The proxy then ensures that AJAX
requests from the client conform to these models. 

Chugh \etal~\cite{stagedif:pldi09} developed a staged information-flow tracking
framework for \js\ to protect hosts from untrusted guests. Its static analysis
identifies constraints on host variables that can be read or written by guests.
It validates these constraints on code loaded at runtime via \code{eval} or
\html{script} tags, and rejects such code if it violates these constraints. 
Unlike \txjs, which tracks changes to both the heap and DOM, Chugh \etal's work
only tracks changes to the heap.

%\begin{wrapfigure}{rh}{0.6\textwidth}
\begin{figure*}[t]
%\vspace*{-3.5ex}
\setlength{\tabcolsep}{1pt}
\centering
\fontsize{6.5}{6.5} \selectfont %\addtolength{\tabcolsep}{-2pt}
\begin{tabular}{|l|c|c|c|c|}
\hline
\textbf{System} & 
  \textbf{Recovery} & 
  \shortstack{\textbf{Unrestricted}\\\textbf{guest}} & 
  \shortstack{\textbf{Unmodified}\\\textbf{browser}} & 
  \shortstack{\textbf{Policy}\\\textbf{coverage}}\\
\hline
%
Transcript         
                   & \yes     & \yes      & \no      & Heap + DOM\\
%
Conscript \cite{conscript:oak10}   
                   & \no      & \yes      & \no      & Heap + DOM\\
%
AdJail \cite{adsandbox}      
                   & \no      & \yes      & \yes     & DOM$^{(1)}$\\
%
Caja \cite{caja}
                   & \no      & \no       & \yes     & Heap + DOM\\
%
Wrappers \cite{wrappers:esorics09,mmt10,views:www10}
                   & \no      & \yes$^{(2)}$ & \yes  & Heap + DOM\\
%
Info. flow \cite{stagedif:pldi09}
                   & \no      & \yes      & \yes     & Heap\\
%
IRMs \cite{phung:asiaccs09,corescript:popl07,browsershield}
                   & \no      & \yes      & \yes     & Heap + DOM\\
%
Subsetting \cite{mmt10,adsafe,fbjs} 
                   & \no      & \no       & \yes     & Static policies$^{(3)}$\\
%
\hline
\end{tabular}
\remove{
\mycaptionsquish{Techniques to confine untrusted guests.
\describecaption{(1)~Adjail executes guests in a separate \html{iframe}
from the host and disallows guests from executing in the host's context.
%
(2)~Some wrapper-based solutions~\cite{wrappers:esorics09} restrict \js\
constructs allowed in guests.
%
(3)~Subsetting is a static technique and its policies are not enforced at
runtime.}}{\label{figure:relatedwork}}
}
\add{
\mycaptionsquish{Techniques to confine untrusted guests.
\describecaption{(1)~Adjail uses a separate \html{iframe} to disallows guests
from executing in the host's context.
%
(2)~Some wrapper-based solutions~\cite{wrappers:esorics09} restrict \js\
constructs allowed in guests.
%
(3)~Subsetting is a static technique and its policies are not enforced at
runtime.}}{\label{figure:relatedwork}}
}
%\vspace*{-3.5ex}
%\end{wrapfigure}
\end{figure*}

\myparagraph{Language restriction} 
%
Several projects have defined subsets of \js\ that omit dynamic constructs, such
as \code{eval}, \code{with} and \code{this}, to make it amenable to static
analysis \cite{adsafe,fbjs,caja,gatekeeper:sec09}. However, designing
safe subsets of \js\ is non-trivial
\cite{language:csf09,semantics:aplas08,mmt10,leaks:ndss10}, and also prevents
code developers from using arbitrary constructs of the language in their
applications. \txjs\ places no such restrictions on guest code.

\myparagraph{Object capabilities, wrappers, and code rewriting} 
%
Object capability and wrapper-based solutions
(\eg~\cite{views:www10,mmt10,wrappers:esorics09}) create wrapped versions of
\js\ objects to be protected, and ensure that such objects can only be accessed
by code that has the capability to do so. In contrast to these techniques,
which provide isolation by wrapping the host's objects, \txjs\ wraps guest code
using  transactions, and mediates its actions with the host via
iblocks.  Prior research has also developed solutions to inline
runtime checks into untrusted guests. These include
BrowserShield~\cite{browsershield}, CoreScript~\cite{corescript:popl07}, and
the work of Phung \etal~\cite{phung:asiaccs09}. Unlike these works, \txjs\
simply wraps untrusted code in a transaction, and does not modify it. These
works also do not explicitly address recovery.


\myparagraph{Aspect-oriented policy enforcement}
%
Aspect-oriented programming (AOP) techniques have previously been used to
enforce cross-cutting security
policies~\cite{evans:oak99,polymer:pldi05,ulfarthesis}. Among the AOP-based
frameworks for \js~\cite{conscript:oak10,washizaki}, our work is most closely
related to Conscript~\cite{conscript:oak10}, which uses runtime aspect-weaving
to enforce policies on untrusted guests.  Both Conscript and \txjs\ require
changes to the browser to support their policy enforcement mechanisms. However,
unlike \txjs, Conscript does not address recovery from malicious guests, and
also requires guests to be written in a subset of \js.  While recovery may also
be possible in hosts that use Conscript, the hosts would have to encode these
recovery policies explicitly.  In contrast, hosts that use \txjs\ can simply
discard the speculative changes made by a policy-violating guest.

\myparagraph{Browser-based sandboxing}
%
Both BEEP~\cite{beep} and Mashup\-OS~\cite{mashupos} enhance the browser with
new HTML constructs. BEEP's constructs allow the browser to detect
script-injection attacks, while MashupOS provides sandboxing constructs to
improve the security of client-side mashups. While \txjs\ requires modified
\html{script} tags as well, it provides the ability to speculatively execute
and observe the actions of untrusted code, which neither BEEP nor MashupOS
provide. 

AdJail aims to protect hosts from malicious advertisements~\cite{adsandbox}. It
confines advertisements by executing them in a separate \html{iframe}, and uses
\code{postMessage} to allow the \html{iframe} to communicate with the host.
Hosts use access control policies to determine the set of DOM modifications
allowed by an advertisement. AdJail is effective at confining advertisements,
which cannot affect the host's heap.  However, it is unclear whether this
approach will work in scenarios where hosts and guests need to interact
extensively, \eg~in the case where the guest is a library that the host wishes
to use. \newadd{The forthcoming EcmaScript 6 / Harmony modules
\cite{harmony_modules} and HTML5 \html{iframe sandbox} attribute
\cite{html5_sandbox} also enable new isolation mechanisms by constraining the
way guest code interacts with the host, but unlike \txjs\ they do not address
recovery.}

\myparagraph{Sandboxing through speculation}
\newadd{
Blueprint~\cite{blueprint:oak09} and Virtual Browser~\cite{virtualbrowser:ccs10}
confine guests by setting up a virtual environment for their execution. This
environment is itself written in \js\ and parses HTML and script content,
thereby mediating the execution of guests on unmodified browsers. However,
unlike \txjs, they do not address recovery. \txjs\ is most closely related
to Worlds \cite{warth:ecoop11} in its motivation to provide first-class
primitives that enable programmers to contain side-effects. However, there are
major design and implementation differences including \txjs's ability to enforce
fine-grained security policies and its implementation in SpiderMonkey.}

\myparagraph{Using transactions for performance}
\newadd{
Crom \cite{crom:nsdi10} applies speculation to event handlers and takes
non-speculative event handlers to create speculative versions, running them in a
cloned browser context. ParaScript \cite{parascript:hpca11} implements a
selective checkpointing scheme which avoids JavaScript constructs that allow
code injection like document.write, innerHTML, etc., and stops speculation if
checkpointing becomes expensive. Both, Crom and ParaScript use speculation to
improve performance. In contrast, \txjs\ addresses all scenarios in the design
and implementation of a fully speculative JavaScript engine and required several
new contributions, such as the ability to suspend/resume transactions and wrap
event handlers.
}

%\myparagraph{Using transactions for security}
%
%Transactions and speculative execution mechanisms have previously been used to
%improve software security and reliability
%(\eg~\cite{Seltzer:osdi96,porter:operating:sosp:2009,tmi-ccs08,txbox}) and Web
%browser performance~\cite{crom:nsdi10}. The work most closely related to \txjs\
%is the one by Sun \etal~\cite{see:ndss05} on one-way isolation. This work
%describes a sandboxing mechanism that allows isolated execution of untrusted
%code. As in \txjs, code within the sandbox cannot modify the state of code
%outside, but the reverse is possible. However, their work focused on
%implementing such a sandbox at the granularity of OS-level artifacts, such as
%processes and files. In contrast, \txjs\ applies a similar approach to the
%problem of isolating \js\ code. Our work therefore required several new
%contributions, such as the ability to suspend/resume transactions and wrap
%event handlers.


% Erlingsson \cite{end-to-endweb} et al proposed inlined reference monitors in
% the browser for monitoring security sensitive events. Their solution also
% required the web application to first declare its policies before the rest of
% the web application code executes. Though \txjs also needs to modify the web
% browser to support transactions, the web application developer is free to
% specify the policies at any point following the transaction construct.
% Further, \txjs\ is flexible enough to enforce different security policies for
% different transactions within the same web application.
