\section{Evaluation}
\label{section:evaluation}
\begin{table*}[t]
\centering
\footnotesize{
\begin{tabular}{|l|c|}
\hline
 \bf Components                    & \bf Lines of code \\
\hline
Code Provisioning & 270\\
\hline
Loading and Relocating & 188\\
\hline
Checking Executables linked against musl-libc & 1949\\
\hline
Checking Executables Compiled with Stack Protection & 109\\
\hline
Checking Executables Containing Indirect Function-Call Checks & 129\\
\hline
Client's side program & 349\\
\hline
Musl-libc & 90728\\
\hline
Lib crypto (openssl) & 287985\\
\hline
Lib ssl (openssl) & 63566\\
\hline
Total & 453349\\
\hline
\end{tabular}
}
\caption{The lines of code for each component in the loader}
{\label{table:tcb}}
\end{table*}

\begin{table*}[t]
\centering
\footnotesize{
\begin{tabular}{|l|c|c|c|c|}
\hline
 \bf Benchmarks                    & \bf Number of Instructions & \bf CPU cycles of  Disassembling & \bf CPU cycles of Policy Checking & \bf CPU cycles of Loading and Relocating\\
\hline
Nginx & 262228 & 694405019 & 1307411662 & 128696\\
\hline
401.bzip2 & 24112 & 34071240 & 148922245 & 4239\\
\hline
Graph-500 & 100411 & 140307017 & 246669796 & 4582\\
\hline
429.mcf & 12903 & 18242127 & 123895553 & 4363\\
\hline
Memcached & 71437 & 137372517 & 489914732 & 8115\\
\hline
Netperf & 51403 & 90616563 & 367356878 & 18090\\
\hline
Otp-gen & 28125 & 42823024 & 198587525 & 5388\\
\hline
\end{tabular}
}
\caption{Overhead of the system when checking different benchmarks linked against musl-libc}{\label{table:checkinglinkedlib}}
\end{table*}

\begin{table*}[t]
\centering
\footnotesize{
\begin{tabular}{|l|c|c|c|c|}
\hline
 \bf Benchmarks                    & \bf Number of Instructions & \bf CPU Cycles of  Disassembling & \bf CPU Cycles of Policy Checking & \bf CPU Cycles of Loading and Relocating\\
\hline
Nginx & 271106 & 719360640 & 713772098 & 128662\\
\hline
401.bzip2 & 24226 & 34292136 & 1751276 & 4206\\
\hline
Graph-500 & 100488 & 140588361 & 195218892 & 4548\\
\hline
429.mcf & 12985 & 18288921 & 31459881 & 4330\\
\hline
Memcached & 71677 & 137877497 & 325442403 & 8081\\
\hline
Netperf & 51868 & 91577335 & 183274713 & 18057\\

Otp-gen & 28217 & 43053386 & 217302816 & 5355\\
\hline
\end{tabular}
}
\caption{Overhead of the system when checking different benchmarks compiled with stack protection}{\label{table:checkingstackprotection}}
\end{table*}

\begin{table*}[t]
\centering
\footnotesize{
\begin{tabular}{|l|c|c|c|c|}
\hline
 \bf Benchmarks                    & \bf Number of Instructions & \bf CPU Cycles of  Disassembling & \bf CPU Cycles of Policy Checking & \bf CPU Cycles of Loading and Relocating\\
\hline
Nginx & 267669 & 821734999 & 20843253 & 128668\\
\hline
401.bzip2 & 24201 & 34235817 & 1751276 & 4206\\
\hline
Graph-500 & 100424 & 140429738 & 7014913 & 4548\\
\hline
429.mcf & 12903 & 18242127 & 1177429 & 4330\\
\hline
Memcached & 71508 & 138231446 & 5301168 & 8081\\
\hline
Netperf & 51431 & 91161601 & 3775318 & 18057\\

Otp-gen & 28132 & 42829680 & 2334847 & 5355\\
\hline
\end{tabular}
}
\caption{Overhead of the system when checking different benchmarks compiled with indirect function-call checks}{\label{table:checkingindirectfunccall}}
\end{table*}
In evaluating the system, our main goals were 
\begin{itemize}
\item To demonstrate various policy checks that are enabled by the system; and
\item To understand the performance overhead introduced by the components of the system. 
\end{itemize}
Our setup consisted of running OpenSGX atop of Ubuntu 14.04 on a physical machine  equipped 
with an Intel Core i5 CPU and 16GB of memory. We use clang and llvm version 3.6 to compile 
and instrument many real world applications: nginx, memcached, netperf, otp-gen, graph-500 
and two benchmarks 401.bzip2 and 429.mcf. In all experiments, all the applications are compiled 
as position independent executables (PIEs) and are statically linked. Also, to keep the size of 
the executables small all applications are linked against musl-libc ~\cite{musl-libc} instead 
of GNU libc ~\cite{GNU libc}. \tabref{table:tcb} shows the lines of code of all the components of the
loader.

\subsection{Linked against a Required Library}
When a cloud provider allows a client to run code on her platform, she often expects the client 
to run some particular versions of the code, for example a version of the code with security updates. 
As a result, there is a need for the cloud provider to checks if the client has provided a required 
version of the code.

We implemented a policy check that verifies whether an executable is linked against musl-libc 
~\cite{musl-libc} version 1.0.5. To perform this check, we first generate the SHA-256 hashes of all 
the functions of the required version of musl-libc. The policy check starts by iterating through the 
instruction buffer and looking for all direct function calls. For each direct function call, the 
policy check computes the target of the call and then looks up the symbol hash table to get the 
function name of the target. If the target does not exist in the symbol hash table the check will 
mark the function call as invalid; otherwise, it will computes the SHA-256 hash of all the 
instructions starting from the beginning to the end of the function. To determine whether an 
instruction is the end of the function, we rely on the fact that all the instructions after the 
beginning of the function to the end of the function are not at the beginning of a function and 
therefore do not exist in the symbol hash table. The policy check next compares the hash of the 
function in the executable with its hash in musl-libc. If the two hashes do not match, the client 
has not provided the required musl-libc; otherwise, the policy check continues with the next 
iteration until it reaches the end of the instruction buffer.

Similar to ~\cite{OpenSGX}, we assume that each SGX instruction takes 10K CPU cycles and non-SGX 
instructions run at native speed within the enclave. We leverage OpenSGX's performance counter and 
QEMU's instruction count ~\cite{http://wiki.qemu.org/download/qemu-doc.html} to count SGX and non-SGX 
instructions. We calculate the CPU cycles of non-SGX instructions by measuring the instructions per 
cycle by executing the loader natively without OpenSGX. ~\tabref{table:checkinglinkedlib} presents 
the results of our experiments when running this policy check against different benchmarks.  

\subsection{Compiled with Stack Protection}

clang gives the option to emit extra code for checking potential buffer overflows similar to that of 
GCC ~\cite{https://gcc.gnu.org/onlinedocs/gcc-4.9.2/gcc/Optimize-Options.html}. Clang's 
\textit{-fstack-protector} flag lets the compiler to add a guard variable when a function starts and 
checks the variable when a function exits. For example, when compiling with the flag, the following 
extra code will be emitted:
\begin{lstlisting}
19311: mov    %fs:0x28,%rax
1931a: mov    %rax,(%rsp)

193fe: mov    %fs:0x28,%rax
19407: cmp    (%rsp),%rax
1940b: jne    1941f
1941f: callq  8d5bf <__stack_chk_fail>
\end{lstlisting}

The two instructions at addresses 193fe and 19407 check if the variable at the top of the stack is 
the same as the variable at \%fs:0x28. If the values do not match, control will be transfered to the \_\_stack\_chk\_fail function.

clang also provides \textit{-fstack-protector-all} which is similar to \textit{-fstack-protector} 
except that all functions are protected. To check whether an executable is compiled with this flag, 
the policy check iterates through the instruction buffer and identifies the start of a function using 
the symbol hash table. Within each function, the policy check looks for instructions that affect the 
stack's variables (mov \%rax,(\%rsp) in the above example). It then identifies the source operand of 
the instruction (\%rax) and figures out the value of the source operand (mov \%fs:0x28,\%rax). As a 
next step, it checks if the function contains a \textit{cmp} instruction with the source and destination 
are the stack's variable and the previous source operand respectively. It also has to checks that right 
before the \textit{cmp} instruction, there is an instruction that computes the original value of the 
source operand (mov \%fs:0x28,\%rax). Finally, the policy looks for the \textit{jne} and \textit{callq} 
instructions. It computes the target of the \textit{callq} instruction and checks the symbol hash table 
to verify that the target corresponds to the the \_\_stack\_chk\_fail function.

\tabref{table:checkingstackprotection} presents the results of our experiments when running this policy 
check against different benchmarks.

\subsection{Compiled with Indirect Function-Call Checks}
Protecting applications against control flow attacks is one of the emerging concern due to the fact 
that attackers have recently focused on taking advantage of heap-based corruptions to overwrite function 
pointers to change the flow of a program. Control Flow Integrity (CFI)is a measure that guards against 
these attacks by restricting the targets of indirect control transfers to a set of precomputed locations.

We implemented a policy check to verify that executables are compiled with Indirect Function-Call Checks 
(IFCC) ~\citep{Edge CFI}. IFCC protects indirect calls by generating for the targets of indirect calls. 
It adds code at indirect call sites to ensure that function pointers point to a jump table entry. For 
example, IFCC emits the following code for an indirect function call:

\begin{lstlisting}
1b459: lea  0x85c70(%rip),%rax #<__llvm_
#jump_instr_table_0_1>
1b460: sub  %eax,%ecx
1b462: and  $0x1ff8,%rcx
1b469: add  %rax,%rcx
1b475: callq *%rcx
\end{lstlisting}

To instrument executables with these checks, we apply an IFCC patch ~\cite{https://reviews.llvm.org/D4167} 
to LLVM and clang. To check whether an executable is compiled with FFCI checks, the policy first figures 
out the range of the jump table by relying on the fact that all FFCI jump table entries have the following 
format:
\begin{lstlisting}
a19d0 <__llvm_jump_instr_table_0_289>:
a19d0:       jmpq   41090 <ngx_execute_proc>
a19d5:       nopl   (%rax)
\end{lstlisting}

The policy check continues by iterating through the instruction buffer and looking for
indirect function calls. It then verifies that before the indirect function calls, there is a sequence of \textit{lea} (lea 0x85c70(\%rip),\%rax in the above example), \textit{sub} (sub \%eax,\%ecx), \textit{and} 
(and    \$0x1ff8,\%rcx) and \textit{add} (add \%rax,\%rcx). It then computes the target of the indirect call 
and verifies that the target is within the range of the jump table.

\tabref{table:checkingindirectfunccall} presents the results of our experiments when running this policy check against different benchmarks.
