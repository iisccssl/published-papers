\section{Related Work}
\label{section:related}

\myparagraphsquish{Intel SGX.} A number of recent projects have applied Intel
SGX for trusted computation on cloud platforms. Microsoft's
Haven~\citep{haven:tocs15} was the first project that leveraged Intel SGX to
enable unmodified Windows binaries to run on Intel SGX-based cloud platforms.
Haven allows an application to be linked with a runtime library OS variant of
Windows 8 and loaded into an enclave. The confidentiality and integrity of this
code and data is protected from the cloud provider.  VC3~\cite{vc3:oak15} is
another effort to leverage SGX to provide security for enclaves that perform
MapReduce-style computations.  VC3 also recognized that enclave code with
memory safety errors could pose a threat to confidentiality of client data, and
proposed instrumenting client code with a form of control-flow integrity
instrumentation. 

SecureKeeper~\cite{securekeeper:middleware16} leverages Intel SGX to keep
ZooKeeper-style computations confidential. S-NFV~\cite{snfv:sdn16} uses Intel
SGX to address security issues of today's NetWork Function Virtualization (NFV)
infrastructures by securely move the states of NFV applications in enclaves. SGX
processors are also used to improve the performance of privacy preserving
multi-party machine learning~\cite{machinelearningsgx:usenixsec16}.  

While SGX provides attractive hardware-based security guarantees, it places
considerable burden on the enclave code programmer to ensure that computations
executing within the enclave do not accidentally leak information.  Similarly,
vulnerabilities such as memory safety errors in enclave code can lead to
exploits that leak confidential data.  Moat~\cite{moat:ccs15} takes a first
step towards this goal by statically analyzing x86 machine code to be loaded
within the enclave to check for information-flow violations.
\code{/CONFIDENTIAL}~\cite{moatplus:pldi16} extends the approach proposed in
Moat, and provides enclave authors with a library that they can link their
enclave code against. As long as code is linked against the
\code{/CONFIDENTIAL} library, and sensitive data sources are identified, the
library ensures that sensitive data does not accidentally leak from enclaves
(a property called information-release confinement).
\code{/CONFIDENTIAL} achieves this goal by restricting enclave communication
with external memory through a narrow interface. 

The key difference between \code{/CONFIDENTIAL} (and also Moat and VC3) and
\tool\ is in the threat model---in \code{/CONFIDENTIAL}, the client compiles
his code against the the \code{/CONFIDENTIAL} library, but the cloud provider
does not check that the code has been compiled against this library. Thus,
\code{/CONFIDENTIAL} is developed for the benefit of the client.  In contrast,
\tool\ focuses on mutual trust and SLA compliance. With \tool, the cloud
provider can check that the client has compiled his code against a library such
as \code{/CONFIDENTIAL}. The cloud provider therefore obtains an assurance that
the client's code is policy-compliant. However, he does not learn any further
facts about the client's code, thereby protecting client confidentiality.

Intel SGX does not protect applications against side-channel attacks and \tool\
also does not attempt to eliminate this attack vector. Yuanzhong
\etal~\cite{controlledchannel:oak15} demonstrate that by exploiting the fact
that page table management in SGX is under the control of the OS, a malicious
OS can manipulate page tables and page faults to learn memory access patterns
of an enclave and therefore can infer private information of that enclave.
Similarly, AsynShock~\cite{asyncshock:esorics16} controls page access
permissions of a multi-threaded enclave-based application to exploit
synchronization bugs that might lead to memory corruption or crashes. It offers
a tool to widen the attack window in synchronization bugs by interrupting a
thread by removing the read and execute permissions from enclave pages and then
scheduling another thread whose execution causes synchronization bugs.

Finally, recent work on Ryoan~\cite{hunt:osdi16} has leveraged the Intel SGX to
build a sandbox for distributed applications. Like \tool, Ryoan also relies on
NaCl~\cite{nativeclient:oak09} to enforce restrictions on code loaded inside an
SGX sandbox, but does so for an entirely different purpose. While Ryoan uses
NaCl to ensure that code loaded into an enclave only has restricted control
transfers, \tool\ uses NaCl only for reliable disassembly.

\myparagraph{Recognizing Functions in Binary Code.} \tool\ assumes that client
binaries are shipped with symbol-table information (binaries that do not
contain this information are auto-rejected by \tool). This helps identify
functions in binary code which might be needed by the policies on verifying the
binaries. Recognizing functions in COTS programs which do not contain debug
information has become a growing interest in recent years.  For instance, both 
supervised machine learning~\cite{binaryanalysis:AAAI08} and neural
network-based algorithms~\cite{functionrecognizing:usenixsec15} have been
applied to recognize functions in stripped binary executables. However, these
approaches are still in their infancy, and cannot guarantee 100\%
accuracy~\cite{functionrecognizing:usenixsec15}. As these techniques develop
and improve in their accuracy and performance, \tool\ can be enhanced to even
consider stripped binaries as enclave code.

\myparagraph{Instrumenting Code to Thwart Attacks.} For years, various
solutions have been proposed to defend against control flow hijack attacks due
to software bugs. These include stack canaries to protect return addresses and
other control data on the stack~\cite{stackcanaries:handbookofinfosec} uses
stack canaries, and various forms of control-flow integrity protection
(\eg~\cite{xfi:osdi06, mip:ccs13, hypersafe:oak10,edgecfi:sec14}) use binary
rewriting to enforce CFI protection. Cloud providers may require clients to
compile their code with such instrumentation. As we saw in this paper, \tool\
can accommodate a variety of policy modules that check that enclave code has
been instrumented as agreed-upon by the cloud provider and the client.

% ~\cite{edgecfi:sec14} applies compiler-based instrumentation to
% provide CFI protection and to prevent attacks that overwrite the virtual table
% (vtable) pointer to hijack virtual calls. For CFI defense, it modifies compiler
% toolchains to generate jump tables for indirect call targets and insert code at
% indirect call sites to make function pointers point to an entry in the jump
% tables. Similarly, to protect virtual calls it adds a verification call after
% getting the vtable pointer and before deferencing it. The verification call
% receives as arguments the vtable pointer and a set of valid vtable pointers for
% the call site. If the vtable pointer is not in the valid set the verification
% call aborts execution.
