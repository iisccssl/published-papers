\section{Implementation}
\label{section:implementation}

% ---------------------------------------------------------------------------------
% TODO: This text to be added in this section. Modify and fix any
% cross-references.

% TODO: Suitably modify the stuff below and move it to the implementation
% section. Not really necessary to introduce OpenSGX at this early stage.

%Features of the SGX are now commercially deployed in Intel's Skylake series of
%processors. Despite the availability of commodity hardware, for this paper, we
%chose to develop \tool\ atop OpenSGX~\cite{opensgx:ndss16}, a QEMU-based SGX
%emulation infrastructure that supports x86-64 executables, compiled as statically-linked, 
%position independent executables (PIEs). Our choice was motivated by two factors.

Features of the SGX are now commercially deployed in Intel's Skylake series of
processors. Despite the availability of commodity hardware, for this paper, we
chose to develop \tool\ atop OpenSGX~\cite{opensgx:ndss16}, a QEMU-based SGX
emulation infrastructure. Two factors governed our choice.

First, open-source software support for SGX enclave development is still
rudimentary. To create a system that fully realizes the power of enclaves, we
need support for in-enclave bootstrap code and supporting libraries, drivers
within the OS, and compiler support to emit SGX code. Although Intel provides
SDKs for Windows 10, these SDKs are closed-source, which complicates extension
and modification~\cite{win10:sdk}. An open-source Linux
SDK~\cite{linuxsdk:june16}, which we could have extended, was released only in
June 2016 when we were already underway with our \tool\ prototype.  While
Intel's programming references~\cite{intelsgx:sep13,intelsgx:oct14} specify the
semantics of instructions, they offer considerable freedom to end-developers to
choose their enclave programming model. Community consensus has yet to emerge
on these programming models, and rather than define one of our own, we chose to
use the programming model defined by OpenSGX. Moreover, OpenSGX incorporates
driver support for SGX, and has ported various utilities and libraries to work
in enclave mode, which we could readily utilize and extend for \tool.

Second, even the SGX architecture itself is evolving. Skylake processors
currently implement version~1 of the instruction set. This instruction set
poses a number of restrictions~\cite{dynmgmt:hasp16,dynalloc:hasp16}, the chief
of which is that it does not permit page protections to be changed at the
hardware level for pages in the EPC. Page protections can still be changed at
the level of page tables, and SGX performs a two-level page protection check
prior to writing or modifying a page: at the page-table level and at the
hardware level. However, recent work has shown that lack of support for page
protection modifications at the EPC level can be
exploited~\cite{asyncshock:esorics16}. As already discussed, \tool\ relies on
the ability to change EPC page protections. In addition, SGX hardware currently
requires all enclave memory to be committed at enclave build time (therefore
requiring the developer to predict and use the maximum stack/heap sizes during
enclave build) and does not allow additional code modules to be dynamically
loaded into the enclave after it has been built.  While these changes have been
proposed in version~2 of the instruction
set~\cite{dynmgmt:hasp16,dynalloc:hasp16}, it is not yet commercially
available~\cite{sgx:v2:na}. In contrast, it is easy to explore such changes
within the context of a software-based SGX emulator such as OpenSGX.

%------------------------------------------------------------------------------


% ---------------------------------------------------------------------------------

%We implemented a proof-of-concept of the system on top of OpenSGX
%~\cite{opensgx:ndss16}, an open source Intel SGX emulator. The system supports x86-64
%executables that use Executable and 
%Linkable Format (ELF) ~\cite{elfsharedlib, elfsystemV} which are compiled as position independent executables (PIEs)
%and are statically linked.  In this section, we first describe the modification
%we made to OpenSGX. We next discuss the implementation of the bootstrap, the
%decoder and the loader. 

Our \tool\ prototype supports x86-64 executables that use ELF
format~\cite{elfsharedlib, elfsystemV}, are compiled as position independent
executables and are statically linked.  In this section, we first describe our
modifications to OpenSGX. We then discuss the components of \tool.

\myparagraph{Modifications to OpenSGX.} The client enclave holds the client
executable as well as its decoded instructions. As a result, the number of EPC
pages should be large enough to meet the memory requirements of the client
enclave. OpenSGX restricts the number of EPC pages to 2000. We modified OpenSGX
to increase the default number of EPC pages to 32000 which translates to 128 MB
for the physical protected memory region. On OpenSGX, this size can be extended
to meet further memory requirements. We also change the number of initial page
frames for the heap region from the default value of 300 to 5000.

\myparagraph{Binary Disassembly.} The executable file provided by the client is
in 64-bit ELF format.  An ELF binary comprises of several segments and each
segment has one or more sections. Each section contains information of similar
type, for instance the \code{.text} section contains the executable code, all
writable data is stored in the \code{.data} section and uninitilized data is
kept in the \code{.bss} section. The ELF format also features an ELF header
located at the beginning of the file and is used to recognize other parts of
the file.

One common challenge in disassembling a binary is mixing of code and data
within the code section. Our \tool\ prototype assumes that the client's
executable is compiled with separated code and data sections. Before
disassembling the code sections of the executable, the loader checks its header
to verify that the executable is correctly formatted. The checks include
checking the signature as well as the ELF class of the executable. The loader
next reads the program header of the executable to extract all text sections.
We implement the disassembler based on the 64-bit binary disassembler of NaCl,
an open source sandbox for native code. Using prefix and opcode tables for
x86-64 bit instruction set, the disassembler parses the byte sequence of the
text sections into instructions and associated metadata
information, \eg~the number of prefix bytes, number of opcode bytes and number
of displacement bytes~\cite{intelinstructionset:june16}.

NaCl's disassembler  does not track all disassembled instructions. Instead,
during the disassembly it uses a buffer that stores the most recently
disassembled instructions. This stems from the fact that the NaCl validates
each instruction right after it is disassembled.  We instead use a dynamically
allocated buffer that can hold all the instructions and use that buffer as the
input to policy checks. Since dynamic memory allocation involves exiting the
enclave mode and invoking a trampoline, we reduce the involved overhead by
restricting the calls to \code{malloc} by allocating a memory page at a time
instead of just a memory region for an instruction.

Along with disassembling the executable, the loader also reads the symbol
tables to keep track of the address and name of all the functions in the
executable. It constructs a symbol hash table whose key is the address of a
function and value is the name of the function. This symbol hash table could be
used by the policy checking component when it perform policy checks.

\myparagraph{Loading.} After the executable has been checked and confirmed to
follow certain policies the loader takes over. The loader maps the \code{text},
\code{data} and \code{bss} segments to the enclave memory, making the
\code{text} segment be executable but read-only, the \code{data} segment and
\code{bss} segment be writable but non-executable. It then locates the sections
that require relocations and the locations at which the relocations should be
applied. The loader acquires all the information that it needs for relocations
from the \code{.dynamic} section of the executable. In particular, the loader
determines the address and the size of relocation tables which contain detailed
information for relocations by reading appropriated entries of the
\code{.dynamic} section. Upon completing relocation, the loader sets up a call
stack and transfers control to the executable.
