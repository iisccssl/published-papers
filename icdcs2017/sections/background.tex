\section{SGX Background}
\label{section:background}

%------------------------------------------------------------------------------
\myparagraph{Enclaves.} 
%
The main feature of SGX is its support for enclaves. An enclave is a linear
span of a process's virtual address space whose physical pages are drawn from a
region of physical memory called the encrypted page cache (EPC). The contents
of EPC pages are protected cryptographically by the hardware, which does not
reveal the encryption key even to the most privileged software layer on the
system (\eg~the OS or the hypervisor). A process can have multiple enclaves in
its address space.

A process enters an enclave via an instruction (\code{EENTER}). Within an
enclave, the process can have multiple threads of execution. Each such thread
can freely access the memory contents of both the enclave as well as the rest
of the process address space. If an enclave thread references an address within
the enclave, the hardware fetches corresponding memory page from the EPC and
decrypts it within the hardware cache hierarchy, thereby offering the process a
view of the plaintext content of the page.  An adversary outside the enclave
(\eg~observing the memory bus) will only see encrypted traffic to the EPC page,
thereby preserving the confidentiality and integrity of the EPC page.  SGX
imposes a few restrictions on the code that can execute within an enclave. An
enclave can only execute user-mode code and cannot invoke any OS services,
\eg~via system calls. If the enclave code needs to avail of such services, it
must save the enclave state, exit the enclave (via an instruction called
\code{EEXIT}), and have the non-enclave code of the process access such
services on its behalf. SGX offers various data structures to save enclave
state in an encrypted fashion, thereby protecting it from adversaries outside
the enclave. SGX hardware ensures that code executing outside the enclave,
whether in user-mode context in the process address space or in kernel-mode
context within the OS (or hypervisor), cannot access the plaintext enclave 
content.

Although an OS (or hypervisor) cannot view the plaintext contents of a
process's enclaves, it is still responsible for various aspects of enclave
management. The OS creates enclaves for processes (using \code{ECREATE}), adds
or removes pages from a process's enclaves (using \code{EADD} and
\code{EREMOVE}, respectively), and manages the process's page tables. Page
table entries corresponding to the virtual address range of an enclave will be
mapped to the EPC. Although we have only introduced a handful of instructions,
the SGX supports a total of 24 new enclave management
instructions~\cite{intelsgx:sep13,intelsgx:oct14}.

%------------------------------------------------------------------------------
\myparagraph{Attesting and Provisioning Enclaves.}
%
When an enclave is newly created within a process's address space, it is
initialized with some generic bootstrap code. The exact nature of this
bootstrap code differs based on the software vendor who offers this code.
However, at the very minimum, this bootstrap code implements basic cryptographic
functionality (\eg~for SSL/TLS), wrappers for system calls and other popular
libraries that the client's enclave code may wish to use.  SGX offers support
for \textit{attestation}~\cite{sgx:attest:hasp13}, which allows remote clients
of an SGX-based cloud platform to ensure that enclaves are initialized
securely. 

Remote attestation on SGX platforms follows a standard challenge/response
scheme as in TPM-based attestation protocols~\cite{sailer:tpm:security04}. The
client sends a challenge to the SGX-based machine on the cloud platform.  Each
SGX-based machine is endowed with a dedicated, Intel-provided \textit{quoting
enclave}. The quoting enclave obtains a measurement (a SHA-256 digest of a log
of all activities during enclave initialization~\cite{sgx:attest:hasp13},
obtained via the \code{EREPORT} instruction) of each newly-created enclave, and
signs the measurement using a device-specific private key, called the Intel
EPID key. The SGX hardware ensures that only the quoting enclave has access to
the EPID key.  The client can then verify the signed measurements, thereby
obtaining a guarantee, rooted in SGX hardware, that the enclave was initialized
correctly.

Following attestation, the client \textit{provisions} the enclave with
sensitive content. Thus, the client needs an encrypted, authenticated channel
between its server and the newly-created enclave on the cloud platform. On SGX
systems, this problem is addressed by generating an ephemeral public/private
key pair during enclave creation and initialization. The value of this
ephemeral public key is included in the attestation quote that is signed by the
quoting enclave, thereby providing the client a hardware-rooted guarantee that
binds the public-key to the enclave.  The client can then use this public-key
to bootstrap an SSL/TLS handshake, thereby establishing a secure channel to the
enclave. The client then transmits all sensitive content to the enclave over
this encrypted channel.


