\section{Introduction}
\label{section:introduction}

In recent years, the research community has devoted much attention to security
and privacy threats that arise in public cloud computing platforms, such as
Amazon EC2 and Microsoft Azure. From the perspective of cloud clients, one of
the chief security concerns is that the computing infrastructure is not under
the client's control. While relinquishing control frees the client from having
to procure and manage computing infrastructure, it also exposes the client's
code and data to cloud providers and administrators. Malicious cloud
administrators can compromise client confidentiality by reading sensitive code
and data directly from memory images of the client's virtual machines (VM).
They could also inject malicious code into client VMs, \eg~to insert backdoors
or log keystrokes, thereby compromising integrity. Even otherwise honest cloud
providers could be forced to violate client trust because of subpoenas.

Intel's SGX
architecture~\cite{sgx:attest:hasp13,sgx:instructions:hasp13,sgx:isolated:hasp13,intelsgx:sep13,intelsgx:oct14}
offers hardware support to alleviate such client security and privacy concerns.
SGX allows client processes and VMs to create \textit{enclaves}, within which
they can store and compute on sensitive data.  Enclaves are encrypted at the
hardware level using hardware-managed keys. SGX guarantees that enclave content that includes enclave code and data is not visible in the clear outside the enclave, even to the most privileged
software layer running on the system, \ie~the operating system (OS) or the
hypervisor. SGX also offers support for enclave attestation, thereby providing
assurances rooted in hardware that an enclave was created and bootstrapped
securely, without interference from the cloud provider.  With SGX, clients can
therefore protect the confidentiality and integrity of their code and data even
from a malicious cloud provider or administrator, so long as they are willing
to trust the hardware.

Despite these benefits, SGX has the unfortunate consequence of flipping the
trust model that is prevalent on contemporary cloud platforms. On non-SGX
platforms, a benign cloud provider benefits from the ability to inspect client
code and data. The cloud provider can provide clients with services such as
malware detection, vulnerability scanning, and memory deduplication. Such
services are also beneficial to benign clients.  The cloud provider can check
client VMs for service-level agreement (SLA) compliance, thereby catching
malicious clients who may misuse the cloud platform in various ways, \eg~by
using it to host a botnet command and control server. In contrast, on an SGX
platform, the cloud provider can no longer inspect the content of a client's
enclaves. This affects benign clients, who can no longer avail of cloud-based
services for their enclaves. It also benefits malicious clients by giving them
free reign to perform a variety of SLA-violating activities within enclaves.
Researchers have discussed the possibility of such ``detection-proof'' SGX
malware~\cite{davenportford:vbtn,rutkowska:aug13,rutkowska:sep13}.  Without the
ability to inspect the client's code, the cloud provider is left to using
other, indirect means to infer the presence of such malicious activities. For
example, minibox ~\cite{minibox:atc14} verifies that an application behaves properly
by checking system call parameters of the application for malicious activities such as
accessing sensitive files that do not belong to the application.

Can a benign client benefit from the security offered by the SGX while still
allowing the cloud provider to exert some control over the content of the
client's enclaves? One strawman solution to achieve this goal is to use a
trusted-third party (TTP). Both the cloud provider and client would agree upon
a certain set of policies/constraints that the enclave content must satisfy (as
is done in SLAs). For example, the cloud provider could specify that the
enclave code must be certified as clean by a certain anti-malware tool, or that
the enclave code be produced by a compiler that inserts security checks, \eg~to
enforce control-flow integrity or check for other memory access violations.
They inform the TTP about these policies, following which the client discloses
its sensitive content to the TTP, which checks for policy compliance.  The
cloud provider then allows the client to provision the enclave with this
content.

However, the main drawback of this strawman solution is the need for a TTP.
Finding such a TTP that is acceptable to both the cloud provider and the client
is challenging in real-world settings, thereby hampering deployability. TTPs
could themselves be subject to government subpoenas that force them to hand
over the client's sensitive content. From the client's perspective, this
solution provides no more security than handing over sensitive
content to the cloud provider.  

\myparagraph{Contributions.} In this paper, we present \tool, an enclave
inspection library that achieves the above goal without TTPs.  Cloud providers
and clients agree upon the policies that enclave code must satisfy and encode
it in \tool. Thus, cloud providers and clients mutually trust \tool\ with
policy enforcement. The cloud provider creates a fresh enclave provisioned with
\tool, and proves to the client, using SGX's hardware attestation, that the
enclave was created securely. The client then hands its sensitive content to
\tool\ over an encrypted channel. It provisions the enclave with the client's
content only if the content is policy-compliant. If not, it informs the cloud
provider, who can prevent the client from creating the enclave. 

\tool's approach combines the security benefits of non-SGX and SGX platforms.
From the cloud provider's perspective, it is able to check client computations
for policy-compliance, as in non-SGX platforms. From the client's perspective,
its sensitive content is not revealed to the cloud provider, preserving the
security guarantee as offered by SGX platforms. Moreover, \tool\ statically
inspects the client's enclave content only once---when the enclave is first
provisioned with that content. One can also imagine an extension of \tool\ that
instruments client code to enforce policies at runtime, but our current
implementation only implements support for static code inspection.  Thus,
except for a small increase in enclave-provisioning time, \tool\ does not
impose any runtime performance penalty on the client's enclave computations.
We have implemented a prototype of \tool\ and have used it to check a variety
of security policies on a number of popular open-source programs running within
enclaves.
