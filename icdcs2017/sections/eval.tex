\section{Evaluation}
\label{section:evaluation}

In evaluating \tool, our main goals were to demonstrate the flexibility of
\tool\ by showing that it can check compliance against a variety of policies,
and understand the performance costs of various components of \tool.

Our setup consisted of running OpenSGX atop of Ubuntu 14.04 on a physical
machine  equipped with an Intel Core i5 CPU and 16GB of memory. We use clang
and llvm version-3.6 to compile and instrument many real world applications to
run within enclaves: Nginx (an HTTP server), Memcached (a popular key-value
store), Netperf (a networking benchmark), otp-gen (a password generator),
graph-500 (a graph data benchmark) and two SPEC benchmarks (401.bzip2 and
429.mcf). In all experiments, all the applications are compiled as position
independent executables and are statically linked. To keep the size of the
executables small all applications are linked against musl-libc~\cite{musllibc}
instead of GNU libc~\cite{gnulibc}. \figref{table:tcb} shows the lines of code
of all the components of \tool's implementation. In the following sections, we
describe the performance costs of three policy modules that we implemented in
\tool. For each benchmark, we assume that the benchmark executes within the
enclave, and we evaluate the cost of \tool\ as it loads the benchmark within
the enclave for execution, and checks for policy compliance.

\begin{figure}[t!]
\centering
\footnotesize{
\begin{tabular}{|p{0.35\textwidth}|r|}
\hline
 \bf Components                    & \bf LOC \\
\hline
Code Provisioning & 270\\
\hline
Loading and Relocating & 188\\
\hline
Checking Executables linked against musl-libc & 1,949\\
\hline
Checking Executables Compiled with Stack Protection & 109\\
\hline
Checking Executables Containing Indirect Function-Call Checks & 129\\
\hline
Client's side program & 349\\
\hline
Musl-libc & 90,728\\
\hline
Lib crypto (openssl) & 287,985\\
\hline
Lib ssl (openssl) & 63,566\\
\hline
Total & 453,349\\
\hline
\end{tabular}}
\mycaption{Sizes of various components of \tool. Some of these components
(\eg~the cryptographic libraries) are part of the default loader in all enclave
implementations.}
{\label{table:tcb}}
\indent\vspace{-0.5cm}
\end{figure}

\myparagraph{Compliance for Library Linking.}
%
When a cloud provider allows a client to run code on its platform, it often
expects the client to run a particular version of the code. For example, the
cloud provider may require that the client execute a version of the code that
has been patched with the latest security updates. As a special case of this,
the cloud provider may wish to check whether the client's code has been linked
against specific versions of certain libraries. For example, the cloud provider
may wish to ensure that if the client's code uses OpenSSL, then the version of
OpenSSL that is used is free of the vulnerability that caused the HeartBleed
exploit. As another example, consider
\code{/CONFIDENTIAL}~\cite{moatplus:pldi16}, a library that ensures that
enclave code satisfies certain information-flow confinement properties,
\ie~enclave code that is linked against this library will not accidentally leak
sensitive information. To prevent liability issues arising from any accidental
data leaks in the client's code, the cloud provider may wish to ensure that the
client's code is linked with the \code{/CONFIDENTIAL} library. 

To illustrate the power of \tool\ at enforcing such library-linking policies,
we implemented a policy module that verifies whether an executable is linked
against musl-libc~\cite{musllibc} version 1.0.5. To perform this check, we
first generate the SHA-256 hashes of all the functions of musl-libc v1.0.5.
For enforcement, the policy module iterates through the instruction buffer of
the code to be loaded in the enclave, and looks for all direct function calls.
For each direct function call, the policy check computes the target of the call
and then looks up the symbol hash table to get the function name of the target.
If the target does not exist in the symbol hash table the check will mark the
function call as invalid; otherwise, it will compute the SHA-256 hash of all
the instructions of the function. Specifically, the policy module sequentially reads instructions starting from 
the computed target address and stops when it comes across an instruction that 
is at the beginning of another function. The policy module relies on the symbol hash
table to identify whether an instruction address is at the beginning of a function. The policy check next compares the hash of the
function in the executable with its hash in musl-libc. If the two hashes do not
match, the client has not provided the required musl-libc; otherwise, the
policy check continues with the next iteration until it reaches the end of the
instruction buffer.

\begin{figure}[t]
\centering
\scriptsize{
\begin{tabular}{|l|r|r|r|p{0.1\linewidth}|}
\hline
 \bf Benchmark & 
     \bf \#Inst. & 
    \bf Disassembly & 
    \bf Policy Checking & 
    \bf Loading and Relocation\\
\hline
Nginx & 262,228 & 694,405,019 & 1,307,411,662 & 128,696\\
\hline
401.bzip2 & 24,112 & 34,071,240 & 148,922,245 & 4,239\\
\hline
Graph-500 & 100,411 & 140,307,017 & 246,669,796 & 4,582\\
\hline
429.mcf & 12,903 & 18,242,127 & 123,895,553 & 4,363\\
\hline
Memcached & 71,437 & 137,372,517 & 489,914,732 & 8,115\\
\hline
Netperf & 51,403 & 90,616,563 & 367,356,878 & 18,090\\
\hline
Otp-gen & 28,125 & 42,823,024 & 198,587,525 & 5,388\\
\hline
\end{tabular}}
\mycaption{Performance of \tool\ to check the \textit{Library-linking} policy.
Here \tool\ checks whether each benchmark has been linked against musl-libc.
The figure shows the size of each benchmark, measured as the number of
instructions in the code to be loaded in the enclave, and the time taken to
execute each step of \tool, reported as CPU cycles. Wall-clock time can be
obtained by multiplying CPU clock cycles with the clock cycle time. A CPU with
a clock rate of 3.5GHz as used in our experiments has 1/3.5 nanoseconds cycle
time. Therefore, the 694,405,019 cycles it takes to disassemble Nginx, for
example, consumes 198.4 milliseconds.}
{\label{table:checkinglinkedlib}}
\indent\vspace{-0.5cm}
\end{figure}

To compute the performance cost, we adopt the approach suggested in the OpenSGX
paper~\cite{opensgx:ndss16} and assume that each SGX instruction takes 10K CPU
cycles and non-SGX instructions run at native speed within the enclave.  We
leverage OpenSGX's performance counter and QEMU's instruction count~\cite{qemu}
to count SGX and non-SGX instructions. We calculate the CPU cycles of non-SGX
instructions by measuring the instructions per cycle by executing the loader
natively without OpenSGX.  \figref{table:checkinglinkedlib} presents the
results of our experiments when running this policy check against different
benchmarks.  

\renewcommand{\baselinestretch}{0.9}
\myparagraph{Compliance for Stack Protection.}
%
Given the prevalence of buffer overflow vulnerabilities in low-level code, a
number of modern compilers now give the option of emitting extra code to
protect loads and stores to memory locations. Clang's \code{-fstack-protector}
flag lets the LLVM compiler add a guard variable when a function starts and
checks the variable when a function exits. For instance, when compiled with the
flag, the following extra code is emitted:

\begin{center}
\footnotesize{
\begin{tabular}{ll}
\code{19311:} & \code{mov    \%fs:0x28, \%rax}\\
\code{1931a:} & \code{mov    \%rax, (\%rsp)}\\
\code{193fe:} & \code{mov    \%fs:0x28, \%rax}\\
\code{19407:} & \code{cmp    (\%rsp), \%rax}\\
\code{1940b:} & \code{jne    1941f}\\
\code{1941f:} & \code{callq  8d5bf <\_\_stack\_chk\_fail>}\\
\end{tabular}}
\end{center}

The two instructions at addresses \code{193fe} and \code{19407} check if the
variable at the top of the stack is the same as the variable at
\code{\%fs:0x28}. If the values do not match, control will be transfered to the
\code{\_\_stack\_chk\_fail} function.

Clang also provides the \code{-fstack-protector-all} option which is similar to
\code{-fstack-protector} except that \textit{all} functions are protected. To
check whether an executable is compiled with this flag, the policy module
iterates through the instruction buffer and identifies the start of a function
using the symbol hash table. Within each function, the policy check looks for
instructions that affect the stack's variables (\eg~\code{mov \%rax,(\%rsp)} in
the above example). It then identifies the source operand of the instruction
(\code{\%rax}) and figures out the value of the source operand (\code{mov
\%fs:0x28,\%rax}).  As a next step, it checks if the function contains a
\code{cmp} instruction with the source and destination are the stack's variable
and the previous source operand, respectively. It also has to check that just
preceding the \code{cmp} instruction, there is an instruction that computes the
original value of the source operand (\code{mov \%fs:0x28,\%rax}). Finally, the
policy looks for the \code{jne} and \code{callq} instructions. It computes the
target of the \code{callq} instruction and checks the symbol hash table to
verify that the target corresponds to the the \code{\_\_stack\_chk\_fail}
function.

\begin{figure}[t]
\centering
\scriptsize{
\begin{tabular}{|l|r|r|r|p{0.1\linewidth}|}
\hline
 \bf Benchmark         & 
     \bf \#Inst. & 
    \bf Disassembly & 
    \bf Policy Checking & 
    \bf Loading and Relocation\\
\hline
Nginx & 271,106 & 719,360,640 & 713,772,098 & 128,662\\
\hline
401.bzip2 & 24,226 & 34,292,136 & 862,023,613 & 4,206\\
\hline
Graph-500 & 100,488 & 140,588,361 & 195,218,892 & 4,548\\
\hline
429.mcf & 12,985 & 18,288,921 & 31,459,881 & 4,330\\
\hline
Memcached & 71,677 & 137,877,497 & 325,442,403 & 8,081\\
\hline
Netperf & 51,868 & 91,577,335 & 183,274,713 & 18,057\\
\hline
Otp-gen & 28,217 & 43,053,386 & 217,302,816 & 5,355\\
\hline
\end{tabular}}
\mycaption{Performance of \tool\ to check the \textit{Stack protection}
policy.  As before, for performance numbers, we report the CPU cycles.}
{\label{table:checkingstackprotection}}
\indent\vspace{-0.5cm}
\end{figure}

Of course, our implementation of \tool's policy module is customized for
Clang's stack protection instrumentation as emitted by the
\code{-fstack-protector} flag. It can easily be customized to check stack
protection instrumentation inserted by other tools, such as Google's
AddressSanitizer~\cite{google:asan:usenix:2012}, LLVM
SoftBound~\cite{softbound:pldi09}, etc.  \figref{table:checkingstackprotection}
presents the results of our experiments when running this policy check against
different benchmarks executing in enclaves.

\myparagraph{Restricting Indirect Function Calls.}
%
Protecting applications against control-flow hijacking attacks is one of the
emerging concern due to the fact that attackers have recently focused on taking
advantage of heap-based corruptions to overwrite function pointers to change
the flow of a program. Control-flow Integrity (CFI)is a measure that guards
against these attacks by restricting the targets of indirect control transfers
to a set of precomputed locations.

We implemented a policy check to verify that executables are compiled with
indirect function-call checks as proposed in recent work by Google
(IFCC)~\cite{edgecfi:sec14}. IFCC protects indirect calls by generating
instrumentation for the targets of indirect calls. It adds code at indirect
call sites to ensure that function pointers point to a jump table entry. For
example, the LLVM implementation of IFCC emits the following code for an
indirect function call:

\begin{center}
\footnotesize{
\begin{tabular}{ll}
\code{1b459:} & \code{lea  0x85c70(\%rip), \%rax} \\
              & \code{\#<\_\_llvm\_\#jump\_instr\_table\_0\_1>}\\
\code{1b460:} & \code{sub  \%eax, \%ecx}\\
\code{1b462:} & \code{and  \$0x1ff8, \%rcx}\\
\code{1b469:} & \code{add  \%rax, \%rcx}\\
\code{1b475:} & \code{callq *\%rcx}\\
\end{tabular}}
\end{center}

To instrument executables with these checks, we use the LLVM/clang
toolchain enhanced with the IFCC patch~\cite{llvmforwardedgecfi}. To check
whether an executable is compiled with IFCC checks, \tool's policy module first
figures out the range of the jump table by relying on the fact that all jump
table entries have the following format:

\begin{center}
\footnotesize{
\begin{tabular}{ll}
\multicolumn{2}{l}{\code{a19d0 <\_\_llvm\_jump\_instr\_table\_0\_289>:}}\\
\code{a19d0:} & \code{~~~~jmpq   41090 <ngx\_execute\_proc>}\\
\code{a19d5:} & \code{~~~~nopl   (\%rax)}\\
\end{tabular}}
\end{center}

\begin{figure}[t]
\centering
\scriptsize{
\begin{tabular}{|l|r|r|r|p{0.1\linewidth}|}
\hline
 \bf Benchmark         & 
    \bf \#Inst. & 
    \bf Disassembly & 
    \bf Policy Checking & 
    \bf Loading and Relocation\\
\hline
Nginx & 267,669 & 821,734,999 & 20,843,253 & 128,668\\
\hline
401.bzip2 & 24,201 & 34,235,817 & 1,751,276 & 4,206\\
\hline
Graph-500 & 100,424 & 140,429,738 & 7,014,913 & 4,548\\
\hline
429.mcf & 12,903 & 18,242,127 & 1,177,429 & 4,330\\
\hline
Memcached & 71,508 & 138,231,446 & 5,301,168 & 8,081\\
\hline
Netperf & 51,431 & 91,161,601 & 3,775,318 & 18,057\\
\hline
Otp-gen & 28,132 & 42,829,680 & 2,334,847 & 5,355\\
\hline
\end{tabular}}
\mycaption{Performance of \tool\ to check the \textit{Indirect Function-Call}
policy.  As before, for performance numbers, we report the CPU cycles.}
{\label{table:checkingindirectfunccall}}
\indent\vspace{-0.4cm}
\end{figure}

\tool's policy module for this check iterates through the instruction buffer
and looking for indirect function calls. It then verifies that before the
indirect function calls, there is a sequence of instructions \code{lea},
\code{sub}, \code{and} and \code{add}, with data dependence between registers
as shown in the code snippet above. It then computes the target of the indirect
call and verifies that the target is within the range of the jump table.
\figref{table:checkingindirectfunccall} presents the results of our experiments
when running this policy check against different benchmarks.
