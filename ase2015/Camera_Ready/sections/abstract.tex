Mobile app developers often wish to make their apps available on a wide variety
of platforms, \eg~Android, iOS, and Windows devices. Each of these platforms
uses a different programming environment, each with its own language and APIs
for app development. Small app development teams lack the resources and the
expertise to build and maintain separate code bases of the app customized for
each platform. As a result, we are beginning to see a number of cross-platform
mobile app development frameworks. These frameworks allow the app developers to
specify the business logic of the app once, using the language and APIs of a
home platform (\eg~Windows Phone), and automatically produce versions of the
app for multiple target platforms (\eg~iOS and Android). 

% \fixme{We should emphasize the challenges below in this paragraph. What was hard in applying differential testing? What is the novelty?}

In this paper, we focus on the problem of testing cross-platform app
development frameworks. Such frameworks are challenging to develop
because they must correctly translate the home platform API to the
(possibly disparate) target platform API while providing the same
behavior. We develop a differential testing methodology to identify
inconsistencies in the way that these frameworks handle the APIs of
the home and target platforms.  We have built a prototype testing
tool, called \tool, and have applied it to test Xamarin, a popular
framework that allows Windows Phone apps to be cross-compiled into
native Android (and iOS) apps.  To date, \tool\ has found 47 bugs in
Xamarin, corresponding to inconsistencies in the way that Xamarin
translates between the semantics of the Windows Phone and the Android
APIs. We have reported these bugs to the Xamarin developers, who have
already committed patches for \checkme{twelve of them}.
