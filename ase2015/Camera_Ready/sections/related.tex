\section{Related Work}
\label{section:related}

\myparagraph{Testing Cross-platform Apps}
%
To our knowledge, our work is the first on testing cross-platform mobile app
development frameworks. However, there has been prior work on testing
cross-platform apps themselves. The most relevant projects in this area are
X-PERT~\cite{xpert:icse13} and FMAP~\cite{fmap:issta14}. Both projects start
with the observation that an increasing number of Web applications create
customized views of Web pages, each optimized for different platforms, \eg~form
factors, mobile platforms, and Web browsers. Yet, end-users expect Web
applications to behave consistently across these platforms.  The X-PERT project
aims to detect inconsistencies in the way that Web apps are displayed by these
platforms. Dually, FMAP attempts to identify similar elements on Web pages that
may be rendered differently on different platforms.

Our work differs from these projects in that it uses inconsistencies in apps to
identify problems in the app development frameworks. While our work has
primarily targeted APIs used to support application logic, future work on
testing mobile apps created using Web-based frameworks
(\eg~\cite{phonegap,sencha,worklight}) can possibly use the techniques from
X-PERT and FMAP to identify inconsistencies in the way UI elements are
displayed across platforms.

Aside from testing techniques for cross-platform apps, a number of recent
projects~\cite{dynodroid:fse13,concolicandroid:fse12,eventbreak:oopsla14,azim:oopsla13,memon:ase07,guitar:memon}
have been developing techniques to test mobile apps. The main goal of these
techniques is to devise effective input generation techniques for mobile apps,
which is challenging because mobile apps are UI-based and event-driven.
Although these projects are not directly related to our work, the input
generation methods that they develop can potentially be used to identify
inconsistencies in the UIs and UI-handling code of cross-platform apps.

Zhong \etal~\cite{zhong:fase13} consider the related problem of testing
cross-language API mapping relations. Such a relation $<f_S, f_T>$ encodes that
a method $f_S$ implemented in a library written in a source language implements
the same features as the method $f_T$ written for an equivalent library in a
target language. Zhong \etal\ also use random differential testing as the
strategy to identify inconsistencies in these relations. Their findings are
similar to ours, and showcase the difficulty of correctly translating
functionality across different platforms and languages.

\myparagraph{Random and Differential Testing}
%
There is a rich literature on both random testing and differential testing, and
both methods have extensively been used for bug-finding. Fuzz-random testing,
for example, feeds random inputs to applications under test. Crashing
applications are most likely buggy because they do not handle ill-formed random
inputs properly. This method has been used to find bugs in UNIX
utilities~\cite{fuzz:unix} and GUI-based Windows applications~\cite{fuzz:nt}.
For object-oriented code, JCrasher~\cite{jcrasher} generates random unit tests,
and uses crashes to identify buggy class implementations. Differential testing,
originally introduced by McKeeman~\cite{mckeeman:difftest:1998}  has recently
found a number of interesting applications in security
(\eg~\cite{brumley:sec07,franken:oak14,oracle:pldi11}).

Random and differential testing can be usefully combined as is evident from our
results. This method has previously been used successfully, for instance, to
find bugs in compilers~\cite{csmith:pldi11}. The authors of Randoop also used
this method to test two versions of the Java development kit, finding a number
of bugs along the way.


\myparagraph{Implementing Cross-platform App Frameworks}
%
As already discussed in \sectref{section:introduction}, there is significant
recent interest in techniques to develop cross-platform mobile apps. For this
task, the dominant methods are the use of Web-based frameworks, which support
app development in Web-based languages, and native frameworks, which create
apps that can natively execute on the platform. These frameworks do much of the
leg-work necessary to translate API calls across platforms. To our knowledge,
these translations are created manually by domain experts. The software
engineering research community has proposed methods to automatically harvest
cross-platform API mappings by mining existing code bases
(\eg~\cite{mam:icse10,rosetta:icse13,robillard:tse}). Such techniques could
potentially be used to improve the way that cross-platform app development
frameworks are built.


% See also the related work of the MD2 paper for additional related work on
% implementing cross-platform apps.

