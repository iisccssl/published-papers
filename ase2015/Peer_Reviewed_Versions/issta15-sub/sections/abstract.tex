% Technical and experience paper submissions: January 23, 2015
%
% Technical papers and experience papers must be prepared in ACM conference
% format and must not exceed 10 pages (including figures and appendices but NOT
% including references).  That is, any pages after the tenth must contain only
% references.  All submissions must be in English.  The submissions should list
% the paper authors recognizably not anonymously (i.e., ISSTA does not use
% double blind reviews).

Mobile app developers often wish to make their apps available on a wide variety
of platforms, \eg~Android, iOS, and Windows devices. Each of these platforms
uses a different programming environment, each with its own language and APIs
for app development. Small app development teams lack the resources and the
expertise to build and maintain separate code bases of the app customized for
each platform. As a result, we are beginning to see a number of cross-platform
mobile app development frameworks. These frameworks allow the app developers to
specify the business logic of the app once, using the language and APIs of a
home platform (\eg~Windows Phone), and automatically produce versions of the
app for multiple target platforms (\eg~iOS and Android). 

In this paper, we focus on the problem of testing cross-platform app
development frameworks. Such frameworks are challenging to develop because
they must correctly translate the home platform API to the (possibly disparate)
target platform API while providing the same behavior. We develop a
differential testing methodology to identify inconsistencies in the way that these
frameworks handle the APIs of the home and target platforms.  We have built a
prototype testing tool, called \tool, and have applied it to test Xamarin, a
popular framework that allows Windows Phone apps to be cross-compiled into
native Android (and iOS) apps.  To date, \tool\ has found \checkme{47}
inconsistencies in the way that Xamarin translates between the semantics of the
Windows Phone and the Android APIs.
