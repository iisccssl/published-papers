\section{Summary and Future Work}
\label{section:conclusions}

Developers are eager to deploy their mobile apps on as many platforms as
possible, and cross-platform mobile app development frameworks are emerging as
a popular vehicle to do so. However, the frameworks themselves are complex and
difficult to develop. Using \tool\ to test Xamarin, we showed that differential
testing is an effective method to identify inconsistencies in the way that
these frameworks handle the APIs of the home and target platforms.

While \tool\ has been highly effective, it suffers from a number of limitations
that we plan to remedy in future work. First, while \tool\ uses random method
sequences as test cases, the arguments to these methods are drawn at random
from a fixed, manually-defined pool. We plan to investigate techniques to
invoke methods with random, yet meaningful arguments, which would further
increase the coverage of the API during testing. Second, \tool\ has primarily
focused on testing the framework libraries that provide support for the
platform-independent part of cross-platform apps. However, when end-users
interact with apps that have been cross-compiled, they also expect a similar
end-user experience when interacting with the app's UI. To ensure this
property, we must test that semantically-similar UI elements on different
platforms elicit the same behavior within the apps on the corresponding
platforms. This will likely require an analysis of the elements of the UI
itself, and recent work on cross-platform feature matching~\cite{fmap:issta14}
may help in this regard. Finally, we plan to extend \tool\ to work with other
target platforms (\eg~Xamarin for iOS) and with other cross-platform
app-development tools.
