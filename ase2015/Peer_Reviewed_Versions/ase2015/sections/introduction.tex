\section{Introduction}
\label{section:introduction}

Over the last several years, we have witnessed a number of advances in mobile
computing technology. Mobile devices are now available in a variety of form
factors, such as glasses, watches, smartphones, tablets, personal robots, and
even cars. These devices come equipped with powerful processors, ample storage,
and a diverse array of sensors. Coupled with advances in operating systems and
middleware for mobile devices, programmers can now avail rich programming APIs
to build software (``\textit{apps}'') that leverage these advances in hardware.
Modern app markets contain hundreds of thousands of apps, and the number and
diversity of apps available to end-users has further contributed to the
popularity of mobile devices. These advances in hardware and software have made
mobile devices viable replacements for desktop computers.

At the same time, we are also witnessing a fundamental shift in the practice of
software development due largely to the dynamics of mobile app development.
Until a few years ago, the task of developing software (targeting mainly
desktop computers) was mostly confined to teams of software engineers, either
in the open-source community or at IT companies. In contrast, it is common
today for small teams or even individuals to build and distribute software via
mobile app markets. Such teams, or individuals, may lack the expertise and
experience of a large team of developers and often face economic and time
constraints during app development.  Nevertheless, mobile app development teams
aim to maximize revenue by making their apps available on a wide variety of
mobile devices, \ie~those running software stacks such as Android, iOS, and
Windows. Apps that are available for a wide variety of mobile devices can reach
a large user base, and can therefore generate more revenue either through app
purchases or via in-app advertisements.

One way to build apps for different mobile platforms is to create customized
versions of apps for each platform, \eg~a separate version of the app for
Android, iOS and Windows devices. However, this approach leads to multiple
versions of the app's code-base, which are difficult to maintain and evolve
over time. 
%
% \fixme{We can probably cut the below statement. We are being too defensive
% and are speculative.}
% 
% Moreover, this approach is poorly-suited for small mobile app development
% teams, which must now dedicate resources to create, maintain and evolve
% different versions of the app for each mobile platform.
%
% As a result of these shortcomings, 
Therefore, developers are increasingly adopting \textit{cross-platform mobile
app development frameworks}. These frameworks allow developers to program the
app's logic once in a high-level language, and provide tool-support to allow
the app to execute on a number of mobile platforms. 

\begin{figure*}[t!]
\centering
\includegraphics[keepaspectratio=true,width=0.9\textwidth]{figures/xptools-overview.png}
\mycaption{Overall operation of a cross-platform mobile app development
framework, using Xamarin as a concrete example. Developers build apps as they
would for the Windows Phone, in C\# using calls to the API of the Windows Phone
SDK. This code can directly be compiled to Windows Phone apps using the Visual
Studio toolchain. Xamarin allows developers to use the same source code to
build native Android or iOS apps. Xamarin provides compatibility libraries
that translate Windows SDK API calls in the code to the relevant API calls of
the underlying Android and iOS SDKs.}{\label{figure:xplatform-overview}}
\end{figure*}

There are two broad classes of cross-platform frameworks available
today.  The first class, which we call \textit{Web-based frameworks},
allows developers to build mobile apps using languages popularly used
to build Web applications, such as HTML5, JavaScript, and
CSS. Examples of such frameworks include Adobe
PhoneGap/Cordova~\cite{phonegap}, Sencha~\cite{sencha} and IBM
MobileFirst~\cite{worklight}.  Developers specify the app's logic and
user interface using one or more of the Web-development languages.
However, these languages do not contain primitives to allow apps to
access resources on the phone, \eg~peripherals such as the camera and
microphone, the address book, and phone settings. Thus, Web-based
frameworks provide supporting runtime libraries that end-users must
download and execute on their mobile devices. Mobile apps interface
with these libraries to access resources on the mobile devices---such
mobile apps are also popularly called hybrid mobile apps.  Web-based
frameworks allow developers to rapidly prototype mobile apps.
However, these frameworks are ill-suited for high-performance apps,
such as games or those that use animation. The expressiveness of the
resulting mobile apps is also limited to the interface exported by the
runtime libraries offered by the frameworks.


The second class, which we call \textit{native frameworks}, addresses the above
challenges. Examples of such frameworks include Xamarin~\cite{xamarin},
Apportable~\cite{apportable}, MD$^2$~\cite{md2:sac13,myappconverter},
\checkme{and the recently-announced cross-platform bridges to be available on
Windows~10~\cite{win10,build15}}.  These frameworks generally support a
\textit{home platform} and one or more \textit{target platforms}. Developers
build mobile apps as they normally would for the home platform, and leverage
the framework's support to automatically produce apps for the target platforms
as well. For example, the home platform for Xamarin is Windows Phone, and
developers build apps using C\# and the API of the Windows Phone SDK. The
Xamarin framework allows developers to automatically build Android and iOS apps
using this code base. Likewise, the home platform for Apportable is iOS.
Developers build apps using Objective-C and the iOS SDK, and leverage
Apportable to produce Android apps from this code base. \checkme{One of the
main highlights of the Microsoft Build Developer Conference held in April/May
2015 was the announcement that the upcoming release of Windows~10 will contain
interoperability bridges that allow Android and iOS developers to easily port
their apps to the Windows~10 platform~\cite{build15,win10}. These bridges allow
Android (or iOS) apps written in Java (or Objective-C) and programmed to use
calls from the Android (or iOS) SDK to transparently execute atop Windows~10
devices.  While the technical details of this platform are forthcoming, it is
reasonable to assume that the bridges will incorporate a compatibility library
to bridge the Android (or iOS) SDK and the Windows~10 SDK.  In this paper, we
will focus on native frameworks for cross-platform mobile app development.
\figref{figure:xplatform-overview} shows the typical workflow of app
development using a native framework. We use Xamarin as a concrete example, but
the same general workflow applies to all such native frameworks.}

When an app developer uses native frameworks, he implicitly expects the apps to
behave consistently across the home and target platforms. Realizing this
expectation depends to a large extent on the fidelity with which the native
framework translates the API calls to SDK of the home platform to the
corresponding SDK of the target platform(s). Unfortunately, this translation is
a complex task because the platform must correctly encode the semantics of both
the home platform and target platform SDK and the relationship between them.
This complexity translates into bugs in the frameworks. For example, as of May
2015, Xamarin's Bugzilla database shows a history of about 7,100 bugs that are
directly related to cross-platform issues\footnote{There are a total of about
20,300 bugs in the database, related to various related products offered by
Xamarin, \eg~the profiler, the IDE environment, etc. We do not count those bugs
because they are not directly related to cross-platform issues.}, about 2,900
of which are still unresolved (listed as ``open'' or ``new''). Although initial
development of Xamarin only started in 2011, its code is based on Mono, which
started in 2004 as an open-source implementation of \code{.NET}.  The fact that
such a large number of bugs exist in a mature and heavily-used platform (over
500,000 users) such as Xamarin/Mono point to the complexity of translating
between platforms. Other native frameworks are no exception, either.
Apportable's bug database, for instance, shows a history of 820 bug reports,
449 of which are still unresolved.

% \fixme{Can we say Xamarin/Mono has been around for some time? The bugs are
% not because of the product being new.}

In this paper, we develop an approach to test native cross-platform app
development tools. Specifically, we aim to discover cases where the behavior of
the application on the home platform is inconsistent with the behavior of its
counterpart on a target platform. Our approach is based on \textit{differential
testing}~\cite{mckeeman:difftest:1998}. We generate random test cases (using
methods described in prior work~\cite{randoop:icse07}), which in our case are
mobile apps in the source language of the home platform. We then use this code
to produce two versions of the app, one for the home platform, and one for the
target platform using the native framework. We then execute the apps and
examine the resulting state for inconsistent behavior.  When two versions of
the app are produced from the same source code, any differences in the behavior
across the versions are indicative of a problem either in the home platform's
SDK, the target platform's SDK, or the way the native framework translates
between the two SDKs.

To realize this approach, we must address two issues:
%
\begin{mylist}
%
\item \textbf{Test Case Generation.} The key research challenge in generating
effective test cases is that the space of valid programs that we can generate
as test cases is essentially unbounded. While we could sample from this space,
the probability that these test cases will induce inconsistent behavior is low.

To address this challenge, we observe that the main difficulty in building
cross-platform mobile app development tools is \textit{translating between the
semantics of the SDKs of the home and target platforms}. Our test-case
generator therefore produces programs that contain random sequences of
invocations to the home platform's SDK. We then observe whether the resulting
apps on the home and target platforms behave consistently. By focusing on the
SDK alone, our approach narrows testing to the most error-prone components of
the cross-platform frameworks.

\item \textbf{Test Oracle Design.} Each of our test cases is compiled into a
full-fledged app, one each for the home and target platforms. When we run the
corresponding apps, the test oracle must observe their behaviors to identify
inconsistencies. The main research challenge here is in defining a suitable
notion of ``behavior'' that can be incorporated into our test oracle.

We address this challenge by observing all data structures that are reachable
from the variables defined in the test cases. We serialize these data
structures into a standard format, and compare the serialized versions on the
home and target platforms. Assuming that the state of the home and target
platforms is the same before the test cases are executed, the final state in
each platform after the test cases have been executed must also be the same.
If not, we consider this inconsistent behavior and report an error.
%
\end{mylist}

We have prototyped this approach in a tool called \tool, which we have
applied to test the Xamarin framework using Android as the target
platform. Using \tool, we have found 47 inconsistencies, which
corresponded to bugs either in Xamarin or the Microsoft SDK (we have
reported these to Xamarin or Microsoft).  To date, 12 of these bugs
have also been fixed in the development branch of
Xamarin~\cite{xam:bzil:25895,xam:bzil:27901,xam:bzil:27910,xam:bzil:27922,xam:bzil:27982,xam:bzil:28017,xam:bzil:28123,xam:bzil:28134,xam:bzil:28562,xam:bzil:28571,xam:bzil:28572}
and others are still open.

%\fixme{We need to emphasize the contributions. One suggestion is to
%  have the contributions paragraph. We can probably highlight two
%  things: develops methodologies and techniques to apply differential
%  testing in cross platform settings and the following are the hard
%  things in this domain. Second contribution is the fact that a simple
%  differential testing technique finds bugs effectively. }

To summarize, our contributions are:
\begin{mybullet}

\item We initiate the study of cross-platform mobile app development
frameworks, and present an analysis of the kinds of bugs that may arise when
these frameworks translate between the semantics of the programming interfaces
of two different platforms (\sectref{section:example}).

\item We present the design of \tool, a testing tool for cross-platform
frameworks that uses random differential testing to expose bugs in these
frameworks (\sectref{section:design}). We also present a number of practical
challenges that we had to overcome in the implementation of \tool
(\sectref{section:implementation}).

\item We show the effectiveness of \tool\ by applying it to Xamarin.
Specifically, \tool\ tests Xamarin's fidelity as it translates between the
Windows Phone and Android platforms. To date, \tool\ has found 47 bugs, 12 of
which have been fixed after we reported them (\sectref{section:evaluation}).

\end{mybullet}
