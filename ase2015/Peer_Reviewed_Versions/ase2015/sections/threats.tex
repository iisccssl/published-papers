\section{Threats to Validity}
\label{section:threats}

Our results show the effectiveness of using random differential testing at
finding bugs in native cross-platform app development frameworks. We now
summarize the threats to the internal and external validity of our results.

The main threat to internal validity is in determining whether an inconsistency
discovered by \tool\ is indeed a symptom of a bug in Xamarin. Although an
inconsistency manifests itself in one of the three different ways outlined in
\figref{table:inconsistency-sources}, it may be the result of using a method
with a documented difference in behavior across platforms. We addressed this
threat in two ways. First, as discussed in \sectref{section:design}, we created
filters for methods with documented deviations of behavior, so
false-positive-generating test cases are not produced. Second, we studied the
results of \tool\ to understand the cause of the inconsistency. In some cases,
this study lead us to a sentence in the documentation or code comments where
the inconsistency was documented (as discussed in
\sectref{section:evaluation}). We did not include these inconsistencies in our
overall count shown in \figref{table:xcheckresults}, and reported the other
inconsistencies to the Xamarin BugZilla forum.

A second threat to internal validity is the ``seriousness'' of the bugs found
by \tool---\ie~does an inconsistency lead to a serious error in the functioning
of an app, or is it just a minor annoyance? Unfortunately, this aspect is much
harder to evaluate. Our take on the issue is that an inconsistency is indeed a
bug that must be fixed (or suitably documented). However, the fact that 12
(over 25\%) of the inconsistencies that we found lead to bug-fixes within days
of our reports shows that Xamarin developers did perceive the inconsistencies
as being significant.

The main threat to external validity is the ability of our approach to
generalize to other native frameworks, or even other aspects of Xamarin
(\eg~the compatibility libraries used to translate between Windows Phone and
iOS). We currently do not have experimental data to answer such questions.
Nevertheless, our results with Xamarin PCLs for Android indicate that
inconsistencies arise because of the challenges involved in translating the
semantics of two different mobile platforms. In particular, an analysis of our
results does not indicate that the kinds of inconsistencies we found are
symptomatic of problems either in Windows Phone or Android alone. Therefore, we
hypothesize that random differential testing of other native frameworks is
quite likely to find similar bugs in them as well.
